% !TeX root = thesis_draft_4_2.tex



\documentclass{article}

\usepackage[
reflist=true,
natbib=true,
autocite=inline,
style=windycity]
{biblatex}


\usepackage[american]{babel}
\usepackage{csquotes}

\usepackage{url}


\usepackage[multiple]{footmisc}

\usepackage{setspace}
\doublespacing


\title{Do recall elections make incumbents more accountable to their constituents?}
\author{Yifan ``William" Zhu}



\addbibresource{../bibliography/lit_review.bib}
\addbibresource{../bibliography/introduction.bib}
\addbibresource{../bibliography/pbc.bib}
\addbibresource{../bibliography/honeymoon.bib}


\begin{document}
	\maketitle
	\begin{abstract}
		Recall election is designed to enhance politicians' accountability to their constituents,
		but there is only mixed evidence supporting the view that recall achieves this goal,
		espeically when parties are in the picture.
		In this paper,
		I provide a modest theory of recall
		in the legislative arena where parties are influential.
		I postulate that,
		when the \textit{governing party},
		that is, the party that controls a majority
		in the legislature
		attempts to pass extreme policies without costing the members' reelection
		by engineering an electoral cycle of legislation
		such that extreme policies are passed early in the legislative term
		and moderate policies are implemented in the run up to the election,
		recall
		can dampen or even eliminate the electoral cycle of legislation
		and thus \textit{incentivize} the governing party
		to allow its members,
		especially those representing marginal constituencies,
		to cultivate a stronger personal vote
		and hence make them more accountable to their constituents.
	\end{abstract}
	
	
	
	
	
	\section*{Introduction}
	
		Inducing a politician to act according to aggregate constituency preferences
		can be difficult when
		it conflicts with those of the politician's party.
		Electoral pressure may nudge incumbents toward the preferences of the electorate
		\autocite{millerConstituencyInfluenceCongress1963},
		but it is a blunt instrument after all
		and does not alway prevent
		elected officials from choosing to implement policies contrary to the view of their constituents
		\autocite{kirklandIndecisionAmericanLegislatures2018}.
		Therefore, time and time again, the voting public have been trying take things into their own hand
		by resorting to institutions of direct democracy.
		Recall,
		by allowing constituents to trigger an election where voters decide whether to remove incumbents from their office \textit{immediately}, 
		is an institution
		of direct democracy
		that may cost the incumbents' reelection
		which is conventionally assumed to be the only goal of an elected official
		\autocite{mayhewCongressElectoralConnection1974}.
		One may
		then
		expect that 
		when the party and the constituency disagree,
		an incumbent,
		seeking reelection and thus 
		wishing to survive a recall or 
		avoid one altogether,
		will be more likely to side with the constituency instead of the party,
		thus enhancing politicians' accountability to constituents.
		
		However, the current literature on recall has yet to borne out this expectation.
		Recalls in general are rather infrequent to begin with,
		limiting the scholarly interest in recall and,
		as a result,
		the volume of the literature on recall.
		The same infrequency also makes it difficult to argue 
		that recall
		\textit
		{actively}
		constrains 
		the
		behavior of politicians 
		most of the time when it remains \textit{unactivated} for an extended period.
		Furthermore, the vast majority of the literature focuses on how 
		constituents initiate and vote in recall elections and 
		few mention of how recall mechanisms affect politicians' behavior.\footnote
		{
			Recalls in general in the US receives little systematic treatment save Cronin's
			\autocite*{croninDirectDemocracyPolitics1989}
			\textit{Direct Democracy}, despite an inordinately large literature focusing on 2003 recall of California Governor Gray Davis, 
			discussing a wide range of relevant features in this particular recall such as 
			voter turnout
			\autocite{arbourVoterTurnoutCalifornia2005},
			voter choice 
			\autocite{alvarezRationalityRationalisticChoice2009,shawStrategicVotingCalifornia2005},
			candidate choice
			\autocite{mcgheeRoleCandidateChoice2009},
			partisan coordination and mobilization,
			\autocite{masketCaliforniaRecallSprint2016},
			and
			voter polarization along
			racial, and ethnic
			lines
			\autocite{seguraRaceRecallRacial2008}.
			Understandably, the 2003 California gubenatorial recall engenders little scholarly
			discussion of how it affects incumbents, 
			due to not only the immediate removal of the incumbent Governor Davis and 
			the
			installment of new California Governor Arnold Schwarzenegger
			but also 
			how unlikely such a recall attempt
			would ever succeed again in this solidly Democratic state,
			which is further corroborated by the recent failed attempt to recall California Governor Newsom Gavin.
		}\footnote
		{
			For a thorough review of the practice and literature of recalls outside the United States,
			see Welp and Whitehead \autocite*{welpPoliticsRecallElections2020}.
		}
		Those who do attempt to understand their effect on incumbents report mixed results, 
		ranging from evidence for limited accountability
		\autocite{okamotoRecallJapanMeasure2020},
		to
		no evidence for accountability
		\autocite{welpRecallReferendumsPeruvian2016},
		or worse,
		evidence for decline in accountability
		\autocite{hamanRecallElectionsTool2021}.
		
		
		In addition,
		some of these studies rest on theoretical assumptions
		the utility of which are yet to be confirmed,
		preventing us from convincingly asserting
		whether recall improves accountability.
		Since recall elections are rather infrequent,
		researchers often seek to explain
		the frquency and success rates of recalll.
		They sometimes take the further step of appropriating
		the frequency of recall
		as the metrics for inferring 
		the extent to which elected officials are held accountable
		\autocite{hamanRecallElectionsTool2021,okamotoRecallJapanMeasure2020,qvortrupHastaVistaComparative2011}.
		Such attempt,
		though understandable given the rarity of recall elections and the accompanying limitation on data,
		seems to assume that the more often recall mechanisms get triggered,
		the more effectively it holds elected officials accountable,
		which is in direct conflict with
		alternative theories of recall acting as a stopvalve 
		\autocite{serdultHistoryDormantInstitution2015, welpRecallDemocraticAdvance2020},
		or as a way for electoral losers to retaliate
		\autocite{welpRecallReferendumsPeruvian2016},
		implying that frequent use of recall is a sign that elections are failing voters and thus
		not necessarily an indicator that recall itself is improving accountability.
		Even though it may be possible to reconcile these competing theories by 
		categorizing recall attempts and 
		specifying the scope condition of each theory,
		doing so would not only
		negate the usefulness of frequency of all different types of recall attempts and success .
		but also
		invite skepticism as to whether the categorization scheme is proper.
		It appears that the prevailing attention to recall frequency
		might just create more questions than answers.
		
		In lieu of attempting to understand the frequency of recalls and its relationship with accountability,
		Gordon and Yntiso \autocite*{gordonIncentiveEffectsRecall2021a}
		take a different approach
		where they systematically examine
		incentive effects of recall elections on incumbents.
		In fact,
		Gordon and Yntiso's approach is the first of its kind in the literature of recall election
		to
		successfully identify the \textit{threat of recall}
		as the cause of change in behavior of elected officials, 
		the sentencing severity of elected judges to be more specific.
		At the same time,
		Gordon and Yntiso \autocite*{gordonIncentiveEffectsRecall2021a} also recognize that the scope condition of the incentive effect has yet to be determined
		and thus remains a gap in the recall literature,
		especially given that elected judges are different from most other elected politicians in many ways.
		Elected judges compete in a legally and cultrually nonpartisan conditions,
		seeking to be above the fray of politics and are at best somewhat sensitive to what their constitutents want;
		whereas most politicians have partisan affiliatiohns
		and their behavior must account for the preferences of both their consituents and their parties.
		Thus, it is difficult to infer from the result reported by
		Gordon and Yntiso \autocite*{gordonIncentiveEffectsRecall2021a}
		as to
		whether this crosswinds of pressures combined with the threat of recall
		produces any net behavioral change.
		In a nutshell, our understandning of whether recall enhances
		politicians' accountability to their constituents
		is still far from complete.
		
		In summary, though recall \textit{can} improve
		the accountability of partisan officials to their local constituencies,
		it is too early to say that recall \textit{always} promotes accountability
		given that the literature of recall is still developing.
		In this paper,
		I shall attempt to furnish a modest theory of recall
		in the legislative arena,
		where behavior of
		incumbent legislators and parties
		are more observable and inferable from formal rules.
		I assume that members of the legislature
		are elected under first-past-the-post (FPTP) rules
		from single-member districts (SMDs),
		and serve a legislative term of fixed length
		just like members of Congress in the US House of Representatives.
		I further assume that all candidates,
		including the incumbent legislators seeking reelections,
		belong to one of the two major parties.
		These assumptions,
		while somewhat limiting,
		open up the vast literature of electoral behavior and legislative behavior in the US House of Representatives
		to help us begin to understand how parties respond to recall elections.
		Additionally,
		since recall is effectively yet another election
		the incumbent must prevent or, failing that, survive,
		applying the results found in
		electoral and legislative studies on recall elections
		becomes justifiable,
		and thus especially useful for supplementing the nascent recall literature.

%		In summary, though recall \textit{can} improve
%		the accountability of partisan officials to their local constituencies,
%		it is too early to say that recall \textit{always} promotes accountability
%		given that the literature of recall is still developing.
%		In this paper,
%		I shall attempt to furnish a modest theory of recall
%		in the legislative arena,
%		where behavior of
%		incumbent legislators and parties
%		are more observable and inferable from formal rules.
%		More specifically, I shall be studying
%		the recall of members of Taiwan's national legislature
%		which is known as Legislative Yuan.
%		One of the primary reason for
%		studying recall in Taiwan
%		stems from a new law passed in 2016,
%		making recall truly feasible for the first time in Taiwan,
%		which gives us an opportunity to observe how recall's incentive effect,
%		especially that on partisan elected officials.
%		It also provides a good supplement to the current literature of recall
%		by studying its effect on elected officials in \textit{national/federal} government
%		of a country, which is rarely studied given that recall mechanisms targeting national offices
%		are rare in the first place.
%		Furthermore,
%		choosing to study Taiwan's Legislative Yuan
%		which shares a number of institutional features with the US House of Representative,
%		including
%		the use of the first-past-the-post (FPTP) rules 
%		to electe members
%		from single-member districts (SMDs),
%		a legislative term of fixed length (2 years for the US House of Representative, 4 years for Legislative Yuan,
%		holding legislative elections at the same time as presidential election
%		(except legislative elections in Taiwan before 2008 and midterm elections in the US.),
%		a two-party system,
%		etc.,
%		opens up the vast literature of electoral behavior and legislative behavior in the US House of Representatives
%		to help us begin to understand how parties respond to recall elections.
%		Additionally,
%		since recall is effectively yet another election
%		the incumbent must prevent or, failing that, survive,
%		applying the results found in
%		electoral and legislative studies on recall elections
%		becomes justifiable,
%		and thus especially useful for supplementing the nascent recall literature.
		
%		
%	\section*{Recall in Taiwan}
%		
%		To further convince the reader of this approach,
%		I shall furnish a primer
%		on the evolutiuon of recall in Taiwan.
%		While the power to recall elected officials
%		was written into the constitution of the regime
%		long before it took over Taiwan at the conclusion
%		of the World War II.\footnote
%		{The right to recall first appeared during the 1930's
%		in several constitutional texts before the Republic of China's jurisdiction
%		extended to Taiwan at the conclusion of the World War II
%		when Taiwan, previously ceded to Japan during the Qing Dynasty in the 19th century,
%		was reverted to the Republic of China.
%		}
%		Before democratization, recall attempts only sporadically
%		occurred and were occassionally
%		successful in their goal of removing representatives
%		in local councils.
%		Since the 
%		It was only formalized during the 1970's,
%		when Taiwan is yet to be democratized
%		and sporadic recall only occurs rural area.
%		Paradoxically,
%		recall became much more difficult since it democratized
%		as the democratically elected Legislative Yuan
%		introduced a series of new restrictions
%		making recall attempts much more difficult to succeed.
%		It was only in 2016 that Legislative Yuan removed a lot of the restrictions
%		thus creating a truly potent recall device that was
%		theretofore only a 
		
		
	\section*{Accountability to Constituency}
		
		An elected official's accountability for their behavior
		to the constituents,
		often known as dyadic representation
		\autocites
		[][]
		{millerConstituencyInfluenceCongress1963}
		{weissbergCollectiveVsDyadic1978}
		{ansolabehereDyadicRepresentation2011}
		or personal representation
		\autocites
		[][]
		{colomerPersonalRepresentationNeglected2011}
		depends on
		the extent to which elected official's behaviors' impact on the constituency
		produces electoral consequence.
		Then,
		to understand how accountable an incumbent is to the constituents,
		one should examine
		the extent to which an incumbent's electoral support depends on the incumbent's behaviors.
		Since all candidates are also
		assumed to be
		members of one of the two major parties,
		when constituents cast a vote for any one of them,
		one part of the voter choice is based on the candidate's personal tratis,
		and the other part depends on the performance of the candidate's party. 
		One may thus say that there is both
		a \textit{personal vote}
		and a \textit{party vote} for the candidate.
	
		\subsection*{Personal Vote and Party Vote}
		
			Cain, Ferejohn and Fiorina
			\autocite*{cainPersonalVoteConstituency1987}
			define the personal vote as the ``portion of a candidate's electoral support which originates in his or her
			personal qualities,
			qualifications,
			activities,
			and record." 
			\autocite*[9]{cainPersonalVoteConstituency1987}
			The significance of the personal vote has long been recognized,
			though the employment of this term is not so universal as it manifests itself throughf a different name or an analogous concept
			(electoral connection
			\autocites[home style][]
			{fennoHomeStyleHouse1978}
			[electoral connection][]
			{mayhewCongressElectoralConnection1974}
			[personal reputation][]
			{careyIncentivesCultivatePersonal1995}
			[local vote][]
			{pattieWinningLocalVote1995}.
			To clarify their defintion of the personal vote,
			Cain, Ferejohn and Fiorina
			\autocite*[9]{cainPersonalVoteConstituency1987}
			further emphasize the personal vote does not include
			\begin{quotation}
				support for the candidate based on his or her partisan affiliation, fixed voter charactersistics such as class, religion, and ethnicity, reactions to national conditions such as the state of the economy and performance evaluations centered on the head of the governing party
			\end{quotation}
			which will serve as my working defintion of ``party vote" are distinct types of support.
			From the above definitions,
			it should be easy to see
			that a politician can be only accountable to the constituency
			insofar as the variation of personal vote
			with regard to the politician's performance
			could make or break the prospect of reelection.
			It follows that to make politicians more accountable to their constituents,
			they need to work hard to cultivate a personal vote.
			
			
			Naturally, this leads to a question as to
			whether cultivating a strong personal vote
			conflicts with the party vote.
			\citeauthor{cainPersonalVoteConstituency1987}'s
			\autocite*{cainPersonalVoteConstituency1987}
			defintion of personal vote and party vote
			do
			hint at a conflict between those two different kinds of support
			which,
			as
			\citeauthor{carseyRethinkingNormalVote2017}
			\autocite*{carseyRethinkingNormalVote2017}
			dispute,
			need not always be a real one.
			More specifically,
			\citeauthor{carseyRethinkingNormalVote2017}
			\autocite*{carseyRethinkingNormalVote2017}
			argue that,
			though a candidate cultivating a personal vote based on policy orientation
			would almost inevitably conflict with seeking a party vote based on party platform,
			it is theoretically possible to
			seek electoral support through non-policy work like providing constituency service
			without undermining the incumbent's pursuit of party vote.
			Whether such proposition is true,
			\citeauthor{carseyRethinkingNormalVote2017}
			\autocite*{carseyRethinkingNormalVote2017}
			argue,
			should only be determined empirically.
			Though I do believe that such conflict is real, even in non-policy areas,
			it seems prudent
			to recognize this validity of this critique 
			by explicitly dividing both party vote and personal vote into
			a
			policy component
			and a
			catch-all
			non-policy component
			and
			then
			focus on their conflict in policy component for the time being.
			
			
		\subsection*{Institutional Incentives for Seeking Personal Vote}
			
			From the discussion above,
			it follows that providing incumbents with incentives to cultivate a stronger personal vote
			generally makes them more accountable to their constituents policy-wise and
			less likely to toe
			the party line,
			resulting in a higher \textit{policy congruence} between incumbents and their constieuncy's aggregate preference.
			Given that recall elections are essentially
			up or down votes on the incumbents
			and thus greatly ressemble elections held under the
			FPTP electoral system,
			which is known for incentivizing incumbents to cultivate a strong personal vote
			\autocite{careyIncentivesCultivatePersonal1995},
			one should expect recall elections
			to improve incumbents' dyadic accountability to their constituents.
			Unfortunately,
			such prediction is unable to stand on its own
			as the party may well discipline
			its members to prevent a defection induced by a recall election or a threat thereof.
			This is especially likely for the governing party in the legislature
			which
			regularly
			uses the majority it commands to set the agenda
			through discipline often bought with
			side payment for individual members
			\autocite{coxSettingAgendaResponsible2005}.
			
			
			To bypass this difficulty,
			I argue that the governing party may have an incentive
			to help its members cultivate a stronger personal vote
			\textit{because}
			doing so helps \textit{protect} the
			\textit{extreme policies} in the governing party's legislative agenda
			from the assault of recall election.
			The governing party,
			for a variety of reasons
			may want to pass policies that are deemed too extreme
			by voters in marginal constituencies.
			When recall is not an option,
			the governing party can choose to engineer an electoral cycle of legislation
			in a mechanism
			akin to the that of political business cycle.
			However,
			when recall becomes feasible,
			voters can threaten to deprive the governing party of majority status
			if it passes an extreme policy
			by firing the legislators sitting in marginal seats right
			after the policy is passed.
			The governing party,
			seeking to perpetuate the electoral cycle of legislation
			due to the continued need to pass extreme policies
			even when recall is feasible,
			is incentivized to allow its legislators,
			especially those representing marginal constituencies,
			to cultivate a stronger personal vote
			in order to survive a real or potential recall,
			leading to a net increase of the legislators' accountability
			to their constituencies.
			
						
	\section*{Electoral Cycle of Legislation}
		
		
		The literature of policy timing
		that revolves
		around
		the electoral calendar
		traces its root back to
		the model of political business cycle
		first formalized by
		\citeauthor{nordhausPoliticalBusinessCycle1975}
		\autocite*{nordhausPoliticalBusinessCycle1975}
		who
		sought to explain economic conditions
		with political institutions.
		Much as
		it offers a simple and compelling mechanism
		the level of empirical support for the
		political business cycle in its original conception
		is rather mixed and
		the theoretical assumptions
		and
		has received its fair share of criticism
		and inspired lively debates about
		when and in what form political business cycle exists.\footnote{
			See \citeauthor{duboisPoliticalBusinessCycles2016}
			\autocite*{duboisPoliticalBusinessCycles2016}
			and
			\citeauthor{altContextConditionalPolitical2009}
			\autocite*{altContextConditionalPolitical2009}.
			They provide good summaries
			of the empirical support for political business cycle
			and theoretical development of
			political business cycle and other
			studies that focus on electoral cycle of policy instruments
			directly or indirectly inspired by
			\citeauthor{nordhausPoliticalBusinessCycle1975}
			\autocite*{nordhausPoliticalBusinessCycle1975}.
		}
		These debates
		have contributed to the development of
		a more nuanced
		understanding of a \textit{conditional} electoral cycle
		of both \textit{policy instruments}
		and \textit{economic conditions}
		\autocite{altContextConditionalPolitical2009}.
		Given that not all studies in the literature
		examine cyclical patterns of \textit{economic conditions},
		it would be inappropriate to consider all of them as
		part of the political business cycle literature.
		However, for lack of a better term,
		this literature will hereinafter be referred to as PBC literature
		where PBC, though usually an abbreviation of political business cycle
		or political budget cycle,
		is meaningless here except as a label.
		
		
		
		
		
		
		There are two developments in the PBC literature that are especially pertinent to this study.
		One offshoot of
		this
		literature
		trains its aim on
		electoral cycle of legislation,
		which refers to
		the cylical pattern of bill passage
		in general
		\autocite{lagonaOppositeCyclesLaws,
			lagonaPoliticalLegislationCycle2008,
			brechlerPoliticalLegislationCycle2014,
			wittmanMythDemocraticFailure1995}
		instead of focusing on one particular type of policy instruments
		like fiscal policies, budgeting among other policies
		with a strong focuses on
		the volume of legislative activity at different points in time
		throughout the legislative term.
		Another branch of the PBC literature investigates how
		modifying the electoral calendar
		yields changes to the cyclical pattern of the policy or economic outcomes.
		One of the many ways electoral calendar could be modified to affect the cycle,
		is the length of a politician's term,
		as
		\citeauthor{nordhausPoliticalBusinessCycle1975}
		\autocite*{nordhausPoliticalBusinessCycle1975}
		points out that shortening the length of the incumbent's term
		undermine the incumbent's ability to engineer a political business cycle.
		Accordingly other studies find that lengthening the term
		makes it less costly to do so
		\autocite
		{amacherCyclesSenatorialVoting1978,
			macraePoliticalModelBusiness1977}.
		Other studies
		point out that if the next election takes place
		at a random point in time before the full term is up,
		political business cycle would be severely dampened and distorted
		\autocite{ginsburghRandomTimingElections1983,lindbeckStabilizationPolicyOpen1976}.
		This
		study of recall contributes to the these two branches of the
		PBC
		litertature
		as it
		not only seeks to expand the theory
		of electoral cycle of legislation
		but also examines recall's impact on electoral calendar
		which effectively gives voter
		a greater say in when and which legislators
		need to stand for recall elections.
		
		
		
		
		
		
		
		
		
		
		
		
		

		But first, I will outline how the electoral cycle of legislation work
		when recall is not feasible.
		As is conventionally assumed in
		most studies in the PBC traditon since 
		\citeauthor{nordhausPoliticalBusinessCycle1975}
		\autocite*{nordhausPoliticalBusinessCycle1975},
		voters are retrospective.
		and
		attribute blames and praise of policy outcomes
		to the governing party which is
		in charge of implementing them.
		Voters then electorally reward and punish governing party legislators
		based on these policies as well.
		Given that this paper also assumes that
		legislators are elected from SMDs under the rules of FPTP
		and that the governing party is very unlikely to lose safe seats
		but very likely to lose swing seats in marginal constituencies,
		its main goal is to help reelect their members
		standing in the marginal constituency elections,
		to preserve its majority in the legislature.
		Furthermore,
		even without delving deeply into what makes a policy
		extreme or moderate,
		it should be uncontroversial to assume that
		voters in marginal constituencies
		strictly prefer moderate policies to extreme ones
		in aggregate.
		Knowing
		that voters
		retain a recency bias,
		that is,
		they
		discount the utility of policy congruence
		more heavily
		when they
		were passed earlier
		\autocite{healySubstitutingEndWhole2014,
			stroblElectoralCyclesGovernment2021},
		the governing party in the legislature
		can pass extreme policies early in the legislative term,
		and dole out moderate policies in the run up to the next election
		to win the retain their seats in \textit{marginal constituencies}
		needed to maintain a workable majority for governing party's bills
		in the legislature.
		Hence,
		I make the first hypothesis that
		
		\textbf{Hypothesis 0.0}: When recall is not feasible,
		the bills passed by the governing party
		\textit{earlier} in the legislative term are on average  
		more extreme than those passed \textit{later} in the legislative term,
		especially the last year of the legislative term.
		
		Furthermore,
		since
		this electoral cycle of legislation theory is predicated on
		the governing party's ability to pass extreme policies
		and its need to win elections
		in marginal constituencies
		in order to maintain a majority for government bills,
		such behavior is conditional on
		the size of the governing party size in the legislature,
		the number of marginal constituencies
		represented by the members of the governing party.
		\citeauthor{stroblElectoralCyclesGovernment2021}
		\autocite*{stroblElectoralCyclesGovernment2021}
		argue along similar lines that the government is less likely to pass austerity policy
		early in the legislative term
		both when it does not hold a majority,
		which gives the party in opposition the
		power to forestall it or prevent it altogether,
		and when the governing party has a large majority,
		which makes it easier to appease voters.
		Following
		I make the analogous hypothesis that
		
		\textbf{Hypothesis 0.1}: When recall is not feasible,
		the closer the size of the governing party is to the minimum-winning majority,
		the more likely its extreme bills will be passed early in the legislature term.
		
		One word of caution may be required
		to note that
		Hypothesis 0.1 does not predict that
		the governing party will pass the most extreme policy
		when it controls a minimum-winning majority.
		It simply predicts that
		the governing party will most likely to opportunistically
		time their policy implementation
		which intends to maximize the distance between
		policies introduced before and after the election day,
		which helps the governing party to exploit the recency bias
		to the fullest extent.
		
		So far, the electoral cycle of legislation outlined above
		does not bode well for moderate voters.
		It suggest that,
		given sufficient amount of time,
		the governing party can bounce back from
		any dip in its popularity come election day,
		by passing policies that appease voters in the marginal constituencie.
		Consequently,
		legislators, especially those belonging to the governing party,
		are effectively unaccountable for their policies
		passed early in their legislative term.
		
		
		
	
	
	\section*{Dampening the Cycle with Recall}
		
		This is precisely where the recall comes in
		to enhance legislators' accountability to their constituents
		\textit{throughout} the legislative term.
		Voters no longer have to wait until the election day
		to punish the governing party in the legislature for bad policy outcome.
		Instead,
		they can do that right after the governing party passes extreme bills or,
		better yet,
		threaten to punish the governing party if it passes the bills.
		This time around,
		the recency bias works in favor of voters,
		seeking to hold legislators accountable in the middle of their term,
		as voters now make their choice at the recall election
		right when the memory of the extreme policy's implementation is feash,
		constituting a threat the credibility of which decreases in the
		legislator's margin of victory.
		Consequently,
		voters,
		espeically those in the marginal constituency,
		are in a better position to
		demand the governing party to pass bills that
		are closer to their aggregate preferences
		by wielding the threat of recall leading to this first hypothesis
		about recall's incentive effect:
		
		\textbf{Hypothesis 1.0}: When recall becomes feasible,
		the maximum distance between government bills
		and the preferences of the marginal constituencies in the policy space will be lower
		than that before recall becomes feasible.
		
		In addition,
		if recall can be triggered at anytime,
		it makes less sense
		for the governing party
		to pass extreme policies
		in the hope that
		voters can forgive them for passing moderate policies
		that are more congruent with their preferences.
		Instead,
		the governing party now expects the threat of recall 
		of its members representing marginal constituencies
		to materialize very soon if extreme policies are to be passed
		at anytime,
		which somewhat defeats the purpose of engineering a electoral cycle of legislation in the first place -
		to pass extreme policies \textit{without} losing marginal seats.
		This leads to the next hypothesis about legislative recall:
		
		\textbf{Hypothesis 1.1}: When recall becomes feasible,
		the distance between the governing party's bills passed earlier in the legislative term
		and those passed later in the legislative term,
		espeically the last year of the legislative term,
		in the policy space decreases.
		
		
		In addition,
		since
		the threat of recall is more credible in marginal constituencies,
		its effect on the governing party's policies
		will also be the most pronounced when the governing party
		has the strongest need for marginal seats.
		Per
		\citeauthor{stroblElectoralCyclesGovernment2021}
		\autocite*{stroblElectoralCyclesGovernment2021},
		the need for marginal seats is the strongest
		when the government has the minimum-winning majority.
		As a result, recall's ability to undermine the political business cycle
		is at its strongest when the government has the minimum-winning majority.
		This leads to the supplementary hypotheses below:
		
		\textbf{Hypothesis 1.0.0}: When recall is feasible,
		further than that before recall becomes feasible.
		the closer the size of the governing party is the minimum-winning majority,
		the closer
		the maximum distance between government bills
		and the preference of the marginal constituencies will be.
		
		\textbf{Hypothesis 1.1.0}: When recall is feasible,
		the closer the size of the governing party is the minimum-winning majority,
		the lesser the distance is between government bills passed earlier in the legislative term
		and those passed later in the legislative term,
		espeically the last year of the legislative term,
		in the policy space.
		
		Of course,
		recall may not completely eliminate electoral cycle of legislation,
		even if
		the foregoing hypotheses do hold.
		It bears repeating that a governing party
		often has to implement unpopular policies for various reasons,
		and elections may still be capable of generating
		a window of opportunities for passing extreme policies,
		even when recall is in the picture.
		Voters' positive feeling toward the chief executive
		at the beginning of the executive term
		\autocite{beckmannPolicyOpportunitiesPresidential2007,
			bondMarginalTimeVaryingEffect2003,
			castrocornejoElectionDayPresidential2022,
			dewanDynamicGovernmentPerformance2012,
			elgieProximityCandidatesPresidential2014}
		may well spill over into the legislative arena
		if it
		coincide with the beginning of the legislative term,
		thus allowing the governing party in the legislature
		to continue to pass some extreme policies in the legislature during that period
		without losing many of its governing party legislators
		at recall or general elections.
		Therefore, it is entirely possible that
		recall
		may simply dampens the existing electoral cycle of legislation,
		without eliminating it.
		
		If the electoral cycle of legislation remains,
		would recall generate other incentive effect
		that distorts the cycle by encouraging the governing party
		to \textit{intentionally} pass more extreme policies
		toward the middle of the term?
		Naturally,
		negotiation within the governing party and
		that between the governing party and the opposition
		may prevent bill passage to take place as soon as possible,
		but that should remain the goal if the governing party operates on
		the assumption that voter has recency bias.
		Would recall introduce a new ideal timing for passing extreme policies?
		
		While this question deserves
		more detailed treatment in the future,
		my intuition suggests
		the answer is \textit{most likely no}.
		The above argument based on the existence of honeymoon period,
		indicates that providing for recall throughout the legislative term
		would not prevent the governing party from
		considering it ideal to implement extreme policies
		during the honeymoon period at the beginning of the term
		as soon as possible.
		Any institutional arrangements
		that prevent voters from recalling the incumbents
		at the beginning of their term would only serve to reinforce this tendency.
		Suppose voters are forbidden from
		recalling incumbents
		during some idiosyncratic time period in the middle of their term,
		say the second year of a four-year term,
		then it may make sense to predict that
		extreme bills would be more like
		to be passed in the second year of a four-year term.
		The last type of institutional arrangement,
		though,
		is difficult to justify and still more difficult to implement.
		Even if recall is permissible throughout the legislative term.
		it still makes sense to pass extreme policies
		as soon as possible after the election is over.
		Ultimately,
		the ideal electoral cycle of legislation desired by the governing party
		should see the passage of extreme policies
		right after the election day
		and the passge of moderate policies right before the election day.
		
		
		
		
		
		Nevertheless,
		while the ideal timing for passing extreme and moderate policies
		is most likely to remain the same even
		when recall becomes feasible,
		the actual timing for passing extreme policies in the legislature,
		may actually be distorted as a result of particularities of recall institutional arrangements.
		In the hypothesis about legislative recall,
		the governing party with a minimum winning
		has the greatest incentive to pass the extreme policies as soon as possible.
		Will it actually be able to do so?
		This question offers a good opportunity
		to revisit whether recall induces the party
		to permit its members to pursue a stronger personal vote.
		Though there may be a number of ways to
		explain away the protracted bargaining process,
		consideration of personal vote presents some of the simplest explanation.
		When the government with a minimum winning majority
		seeks to pass an extreme bill which conflicts with
		the constituency preferences,
		any legislator would have an incentive
		to kill the bill
		unless the party doles out
		enough side payment for supporting the extreme bill,
		leading to an extremely long bargaining process
		that delays the passage of the bill.
		As the size of the governing party moves away from,
		one should also predict that the bargaining process becomes less cumbersome
		as the governing party does not need every member's support
		and the time it takes to roll out the extreme bill
		shortens as a result.
		
		Here, I
		suggest one way
		a particular
		recall institutional arrangement may affect
		the actual timing of the bill passage.
		Remember that
		when recall is assumed to be feasible throughout the legislative term,
		there is no reason particular reason
		to pass an extreme bill at anytime other than the beginning of the legislative term
		since voters can always recall legislators
		right after they pass it.
		That assumption is violated
		in some jurisdictions, Taiwan included,
		where there exist legal prohibition against recalling incumbents
		until a period of time after the term began.
		While this type of legal protection against early recall
		has the potential to preserve the political legislation cyle,
		it also means that there is a clear disincentive
		against passing extreme bills
		\textit{after} a certain point in time during the legislature.
		As a result,
		it is possible to observe a sharp drop in
		the policy extremeness after that point in time.
		Such drop is especially likely to be observed
		when the size of the governing party
		in the legislature is close to minimum-winning majority
		where means the governing party members' incentives
		to haggle and to avoid recall
		are both at its strongest.
		This offers an important test of recall's
		incentive effect on electoral cycle of legislation.
		
		More importantly,
		consideration of the personal vote
		not only sheds light on the bargaining process within the governing party
		but also other legislative behavior
		that is directly tied to 
		cultivating a personal vote
		and ultimately their accountability to voters.
		Before the recall sets in,
		governing party legislators' need for reelection
		can be somewhat reliably met in part by the
		electoral cycle of legislation
		which generates a more favorable prospect of reelection for them.
		Unfortunately,
		after recall becomes feasible,
		the governing party legislators no longer enjoy this
		benefit when extreme government bills continue to be passed.
		Consequently,
		if the governing party ever wants to pass extreme government bills
		when recall is possible,
		it needs its members in the legislature to
		cultivate a \textit{stronger} personal vote
		to ward off recall or threat thereof.
		
		First, the need for a stronger personal votes
		required to pass extreme bills
		without losing its members representing marginal constituencies
		induces the governing party
		to convince the voters that its members 
		are not merely vehicles
		by allowing them to take dissenting positions
		that is costly to the government
		including votes against party.
		Otherwise,
		the voters represented by the governing party legislator,
		believing that their legislator will never oppose the government bills,
		will always threaten to recall the legislator,
		to prevent extreme government bills from passing,
		ultimately threatening to deprive a workable majority for government bills.
		leading to the hypothesis that
		
		\textbf{Hypothesis 2.0}:
		Recall causes
		the position taken by the governing party members
		that represent marginal constituencies to move away from the party position. 
		
		
		Second,
		the governing party also wants to encourage
		its members in marginal constituencies to cultivate a stronger personal vote
		by giving them even more pork for which they could claim credit than they received before,
		which entails the following hypothesis
		
		\textbf{Hypothesis 2.1}:
		Recall causes
		the marginal constituencies to receive a greater amount of pork
		than it otherwise would have.
		
		
	\section*{Discussion and Tentative Conclusion}
		In this paper, I have demonstrated that
		the governing party responds to recall by
		not only shrinking its agenda
		but also giving its members more leeway to construct a stronger personal vote
		in order to protect its remaing agenda
		through perpetuating
		the electoral cycle of legislation,
		albeit in a less pronounced fashion.
		What this paper has yet to done
		is to investigate how recall and electoral institution of the legislature
		may interact with other institutions,
		which may well lead to a different outcome.
		For instance,
		there was no discussion of other institutions of direct democracy
		like initiatives and refrenda
		which very well can be used instead of recall
		and may render the mechanism described in this paper obsolete
		if all policy changes voters seek to change can be achieved through ballot measures.
		Nonetheless,
		given the complexity of modern politics
		and the attendant difficulty of converting every policy proposal to
		an up and down vote for the electorate to decide,
		I believe that recall still have an important role for
		constituents who need a greater say in policy
		without having to cast a ballot on every issue of the day.
		
		Furthermore,
		the validity of this theory hinges largely on
		the existence of an electoral cycle of legislation.
		As previous studies in the PBC literature have warned us,
		observing electoral cycle of just about any sort of policymaking
		is very much conditional on a number of factors,
		such as strategic considerations of the incumbents
		and institutional arrangements.
		On the flip side,
		it may be possible to consider the non-feasibility of recall
		as a prerequisite for observing political legislation cycle
		or other cyclical pattern of policymaking or economic conditions
		around the electoral calendar.
		Future studies can benefit from studying
		the economic indicators and policymaking process
		before and after recall was instituted for US state legislatures
		in the Progressive Era,
		and for Taiwan's national legislature
		before and after it passed recall reform in 2016
		that made recall truly feasible there for the first time.
		Demonstrating that recall dampens the electoral cycle
		of any sort of policymaking process
		could be very important for our understanding of accountability.
		To paraphrase E. E. Schattschneider \autocite*[52]{schattschneiderPartyGovernmentAmerican2017},
		the people \textit{are} a soverign
		whose vocabulary is limited to two words, "Yes"
		and "No," but this sovereign may now speak
		whenever it pleases the sovereign to exact vengence.
		
		
		
		
		
	
	\newpage
	\printbibliography
			
\end{document}

