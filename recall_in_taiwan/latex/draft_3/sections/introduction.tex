% !TeX root = introduction.tex


\documentclass[crop=false]{standalone}



% Stress that recall as an anti-party device comes from Masket, potentially Cronin and Welp and other commentators.
% de Marchi didn't realize it last time, so hammer it home.
% Let ANES also show that this claim is supported.





\begin{document}
	The reputation of parties in general can be very mixed. 
	On the one hand, 
	Political scientists since E. E. Schattschneider, 
%	\cite{}
%	Party Government
	who famously quipped that
	"modern democracy is unthinkable save in terms of parties," have long corroborated this pithy quote by pointing to
	a number of functions that only parties can perform, thereby providing a structure to the political landscape.
	On the other hand, the very structure parties provide has its share of failings.
	Resultantly, despite their persistent attachment to individual parties, voters' attitudes toward parties in general have been far from an unqualified approval.
	American National Electoral Studies 
%	\cite{}
%	ANES
	have reported American voters' increasingly negative view of parties in general.
	Parties have also been charged for polarizing politics which has grown to be one of the top concerns for Americans.
%	\cite{}
%	This result is from fivethirtyeight
	Similar negative attitudes have also been reported across a number of democracies.
%	\cite{}
	In fact, such anti-party attitude has been a recurring theme since Antiquity.
%	I got the source from Masket' s Introduction
	Founding Fathers, Progressive reformers around 1900's were among the most vehement spokespeople 
	for this anti-party cause. 
	
	Consequently, the demerits of the parties, real and imagined, have provided the needed impetus for reforming institutions with a view to fix the parties, so to speak.
	One of the most important achievement of this cause is the institution of direct democracy, including referenda, initiatives, recalls. 
	While referenda and initiatives help translate popular will into public policies, recalls,
	by allowing constituents to trigger an election where voters decide whether to remove incumbents from their office immediately, 
	stand out as the only form of direct democracy that may cost the incumbents' reelection.
	When such great power is given to the constituents, one should expect parties to loosen its grip on their politicians, especially those vying for survival in swing districts where the need to ward off potential and actual recalls looms larger than elsewhere. 
	
	However, the current literature of recall elections have yet to born out this expectation that recalls would undermine the party's control of politicians and return such control to their constituencies. The volume of the literature is already miniscule compared with that focued on conventional elections due to the infrequency of recall elections which put a limit on both the interest of the discipline at large and the methodological approaches and rigor one would like to apply. Even when recalls do undergo scholarly scrutiny, students of recall elections generally agree that parties continue to play an important role not only in coordinating campaign efforts whereas accountability for individual politicians have failed to materialize. This is in direct contradiction to not only one of the major motivations for introducing recall elections and the theory of elections, which both predict that recalls should improve elected officials' accountability to their constitutents.
%	\cite{}
	It almost seems that parties have managed to (re)assert themselves as the main driver for campaigns and outcomes of recall elections.
	
	Furthermore, these results seem to largely sidestepped incumbents' performance in their
	
	Therefore, I seek to revive the contention that parties 
	
	However, aside from the trailblazing work of @gordon2021 on recall elections' effect on elected judges in California, systematic assessment of the effect of recall elections on incumbents' behavior is essentially non-existent. In addition to the methodological challenges mentioned by @gordon2021, perhaps another important reason lies in the lack of credible threat due to prevalent institutional obstacles of recalling an elected official. In the case of judicial recalls in California, the high number of signatures required to trigger one, combined with its rarity and the resultant norm against recalling judges, have historically made the threat of recall very much negligible. It is no wonder that Californian judges only just began to take the threat of recall seriously, even though this mechanism has been on the statute book for decades. In addition, due to the nonpartisan nature of judicial elections and recalls in California, @gordon2021 did not address the role of party of recall elections. Furthermore, unlike judges [@gordon2021, p. 7], most elected officials go out of their way to supply voters with verifiable information on which voters may evaluate their performance.
	
	Fortunately, on the other side of the Pacific Ocean, Taiwan has been experimenting with new legislation passed in 2016 that made it much easier for voters to trigger recall elections of the vast majority of elected officials on every level of government. Recall elections became a recurring event. I attempt to evaluate the extent to which incumbent legislators behave differently as a result of the new recall rules by using difference-in-difference (DiD) approach which leverages Taiwan's mixed-member electoral system. In Taiwan, each voter gets two votes at a legislative election: one vote goes toward electing the legislator representing the local constituency under the system of first-past-the-post (FPTP), and the other one goes toward selecting a party list in a single national at-large constituency under the system of proportional representation (PR). Since dual candidacy in both contests is not permitted, legislators face a distinct set electoral incentives unlike other mixed-member electoral system where legislators fail to win the hearts and minds of local constituents could still be returned to the legislature by the strength of their parties and their positions on the party lists in a larger constituency. Furthermore, perhaps due to the institutional setup where parties rather than the local constituents that choose the legislators in the PR tier, they can only be removed by the party itself and could never be recalled by the voters before and after the passage of the new recall election. This allows us to treat the legislators in the PR tier as the control group and those in the FPTP tier the treatment group. And the difference in behavior between these two groups of legislators could allow us to identify the causal effect of the legislation that made recall easier.
	
	
	
%	How can voters make sure that elected officials are on their best
%	behavior? We can monitor them, and credibly threaten to vote them out
%	when we don't like them anymore. However, the threat isn't always
%	credible. When the election day comes, forces of partisan
%	identification, incumbent effect, among numerous factors, may save their
%	skin after all. Perhaps more importantly, voters may want to punish the
%	officials sooner than the next election. To what other remedies can they
%	resort?
%	
%	The answer given by voters in the city of angels more than a century ago
%	is recall elections. To check the corruption in local government, a
%	ballot measure was in the Los Angeles County passed to allow voters to
%	hold recall municipal officials when the number of eligible voters who
%	demand the recall passes a certain threshold. The same device was later
%	on introduced to the entire California and spread to other US states and
%	countries across the world. However, systematic analysis of recall
%	elections are rare because recall elections themselves are even rarer.
%	Most of the extant literature on recall elections focused on just one
%	recall elections: the 2003 California gubernatorial recall. Furthermore,
%	aside from the trailblazing work of Gordon and Yntiso (2021) on recall
%	elections' effect on elected judges in California, systematic assessment
%	of the effect of recall elections on incumbents' behavior is essentially
%	non-existent. Aside from the methodological challenges mentioned by
%	Gordon and Yntiso (2021), perhaps another important reason lies in the
%	lack of credible threat due to prevalent institutional obstacles of
%	recalling an elected official. In the case of judicial recalls in
%	California, the high number of signatures required to trigger one,
%	combined with its rarity and the resultant norm against recalling
%	judges, have historically made the threat of recall very much
%	negligible. It is small wonder that Californian judges only just began
%	to take the threat of recall seriously, even though this mechanism has
%	been on the statute book for nearly a century.
%	
%	In addition, unlike judges who are supposed to be fair and unswayed by
%	public opnions as much as possible, most elected officials are supposed
%	to represent and internalize the opinions of a number of political
%	actors. Therefore, while it is reasonable to expect judicial recalls to
%	cause judges to consider the opinions of their constituents who are in
%	all likelihood the only group of political actors that matter to their
%	decision process, the extent to which elected officials consider
%	constituents' opinions remains to be seen when they are under the
%	pressure of a number of political actors, say parties, pressure groups,
%	other than their constituents that push them in different directions.
%	
%	Fortunately, for students of direct democracy, a new experiment is under
%	way on the other side of the Pacific Ocean, presenting us a better
%	opportunity to probe the effect of recall elections where incumbents
%	face pressure from a number of different directions. In 2016, Taiwan
%	passed new legislation that made it much easier for voters to trigger
%	recall elections of the vast majority of elected officials on every
%	level of government. Recall elections has become a recurring event since
%	then, and the threat to recall elected officials in Taiwan appears more
%	credible.
%	
%	I propose to evaluate the extent to which incumbent legislators behave
%	differently as a result of the new recall rules by using
%	difference-in-difference (DiD) approach which leverages Taiwan's
%	mixed-member electoral system. In Taiwan, each voter gets two votes at a
%	legislative election: one vote goes toward electing the legislator in a
%	single-member district (SMD) under the system of first-past-the-post
%	(FPTP), and the other one goes toward selecting a party list in a single
%	national multi-member district (MMD) under the system of closed-list
%	proportional representation (CLPR). Since dual candidacy in both
%	contests is not permitted, legislators face a distinct set electoral
%	incentives unlike other mixed-member electoral system where legislators
%	fail to win the hearts and minds of local constituents could still be
%	returned to the legislature by the strength of their parties and their
%	positions on the party lists in a larger constituency. Furthermore,
%	perhaps due to the institutional setup where parties rather than the
%	local constituents that choose the legislators in the PR tier, they can
%	only be removed by the party itself and could never be recalled by the
%	voters before and after the passage of the new recall election. This
%	allows us to treat the legislators in the PR tier as the control group
%	and those in the FPTP tier the treatment group. I propose to choose
%	measure four kinds of legislative behavior: roll call votes, bill
%	co-sponsorship, three-minute speeches, and ministers' question time to
%	detect whether legislators change their behavior because it is easier to
%	recall them.
	

\end{document}

