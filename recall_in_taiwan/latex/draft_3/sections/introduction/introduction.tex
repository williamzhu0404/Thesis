% !TeX root = introduction.tex


\documentclass[hyphens, crop=false]{standalone}



\usepackage[
reflist=true,
%noreprint, 
natbib=true,
autocite=inline,
style=windycity]
{biblatex}


\usepackage[american]{babel}
\usepackage{csquotes}

\usepackage{url}


\usepackage[multiple]{footmisc}


\addbibresource{introduction.bib}


% Stress that recall as an anti-party device comes from Masket, potentially Cronin and Welp and other commentators.
% de Marchi didn't realize it last time, so hammer it home.
% Let ANES also show that this claim is supported.

% There may be ways to motivate this paper by heavily referencing accountability, but
% I REALLY DON'T WANT TO DO THAT!!!

%\title{Research Proposal}
%\author{William Zhu}



\begin{document}
	
%	\maketitle
%%	Motivation 1: Begin with parties' pros and cons

%%	Parties are both good and bad, can be really bad sometimes
%	Political scientists since Schattschneider,
%	who famously quipped that
%	"modern democracy is unthinkable save in terms of parties," 
%	\autocite*[1]{schattschneiderPartyGovernmentAmerican2017}
%	have long supported at least a qualified version of this statement by pointing to
%	a number of parties' functions that are crucial to democracy's proper functioning:
%	they simplify policy options, 
%	organize legislative activities, 
%	recruit and train candidates for political offices, 
%	fund and coordinate their campaigns,
%	\autocite{mungerSignificancePoliticalParties2019},
%	etc.
%	These functions not only organize the activities of politicial elites 
%	but also create useful and stable brands 
%	with which people identify \autocite{coxLegislativeLeviathanParty2007}, 
%	allowing even the politically inattentive voting public at large 
%	to make meaningful and sensible choices at the poll.
%	
%	Yet by the same token, the functions they perform may also draw ire to the parties.
%	Simplifying policy options and screening candidate
%	would also foreclose certain policy choices if major parties agree. 
%	Parties often achieve this function by forming procedural cartels in legislatures, thus polarizing voting in legislatures.
%	\autocite{coxPartyPowerPreferences2010}
%	oftentimes in direct contradiction to the preferences of many constituencies
%	\autocite{kirklandIndecisionAmericanLegislatures2018}.
%	Resultantly, despite their persistent attachment to individual parties 
%	\autocite{campbellAmericanVoter1960}, 
%	voters' attitudes toward parties in general have been far from an unqualified approval.
%	American National Electoral Studies 
%	have reported American voters' increasingly negative view of parties in general.\footnote
%	{
%		American National Electoral Studies (ANES) show that on average
%		voters' feeling toward parties in general hovered around netural until 2000
%		when the question about feelings toward parties in general is dropped. 
%		However, given the later ANES surveys where
%		voters record increasingly less positive feelings toward both parties,
%		it seems reasonable to infer that similar downward trend would be evident
%		in voters' feeling toward parties in general as well.
%		For more details, reader may refer to
%		\url
%		{
%			https://electionstudies.org/resources/anes-guide/top-tables/?id=113
%		}.
%	}
%	Parties have also been charged for polarizing politics which has grown to be one of the top concerns for Americans.\footnote
%	{
%		See \url
%				{
%					https://fivethirtyeight.com/features/3-in-10-americans-named-political-polarization-as-a-top-issue-facing-the-country/
%				}.
%	}
%%	This result is from fivethirtyeight
%	Similar negative attitudes toward parties have also been reported across a number of democracies.\footnote
%	{
%		For attitudes toward individual parties in Europe, see
%		\url
%		{
%			https://www.pewresearch.org/global/2019/10/14/political-parties/
%		};\textit{\textit{}}
%	}
%	In fact, such anti-party attitude has been a recurring theme since Antiquity \autocite{masketInevitablePartyWhy2016a}.
%%	I got the source from Masket' s Introduction
%	Founding Fathers, Progressive reformers around 1900's were among the most vehement spokespeople 
%	for this anti-party cause. 
	
	
%%	Recall as an anti-party mechanism
%	Consequently, the demerits of the parties, real and imagined, have from time to time convinced enough voters and politicians across the world to reform institutions with a view to fix the parties, so to speak.
%	One of the most important achievement of this cause is the institution of direct democracy, including referenda, initiatives, recalls. 
%	Unlike referenda and initiatives, which are meant to help translate popular will into public policies, recalls,
%	by allowing constituents to trigger an election where voters decide whether to remove incumbents from their office immediately, 
%	stand out as the only form of direct democracy that may cost the incumbents' reelection.
%	Given that recall grants such great power to the constituents, 
%	it is perfectly reasonable to expect that 
%	whenever the party and the constituency disagree,
%	an elected official,
%	with great fear of recalls and the threat thereof,
%	will be more likely to side with the constituency instead of the party.

% Motivation 2: Begin with parties and constituencies

%	[I kind of realized that last time when I submitted my literature review
%	I didn't provided a proper puzzle that I aim to solve.
%	So I am submitting my new introduction section of the thesis as my research proposal.
%	Any advice, especially that on how I may generate interest in this paper in the broader political science community,
%	is most appreciated.]
%%	\section*{Introduction}
	
	
	Inducing politicians to act according to constituent preferences can be difficult when
	it conflicts with those of the parties with which the politicians identify.
	Elections may nudge them toward the preferences of the electorate
	\autocite{millerConstituencyInfluenceCongress1963},
	but it is a blunt instrument after all. 
	They oftentimes choose to implement policies contrary to the view of their constituents
	\autocite{kirklandIndecisionAmericanLegislatures2018}.
	Therefore, time and time again, the voting public have been trying take things into their own hand
	by resorting to direct democracy, such as referendum, initiative, and recall. 
	Among
	them,
%	these mechanisms of direct democracy,
%	Unlike referendum and initiative, which are directly translate public preferences into policies, 
	recall,
	by allowing constituents to trigger an election where voters decide whether to remove incumbents from their office immediately, 
	stands out as the only
	mechanism of direct democracy
% one
%	form of direct democracy 
	that may cost the incumbents' reelection.
%	Given that recall grants such great power to the constituents, 
	Hence,
%	it is perfectly reasonable to 
	one may
	expect that 
	when the party and the constituency disagree,
	an incumbent,
	seeking reelection and thus 
	wishing to survive a recall or 
	avoid one altogether,
	will be more likely to side with the constituency instead of the party,
	thus enhancing politicians' accountability to constituents.
	
	However, the current literature on recall has yet to borne out this expectation.
	Recalls in general are rather infrequent to begin with,
	limiting the scholarly interest in recall and, as a result, the volume of the literature on recall.
	The same infrequency also makes it difficult to argue 
	that recall
	\textit
	{
		actively
	}
	constrains 
	the
	behavior of politicians 
	most of the time when it spends a long time being unactivated.
	Furthermore, the vast majority of the literature focuses on how 
	constituents initiate and vote in recall elections and 
	few mention of how recall mechanisms affect politicians' behavior.\footnote
	{
		Recalls in general in the US receives little systematic treatment save Cronin's
		\autocite{croninDirectDemocracyPolitics1989}
		\textit{Direct Democracy}, despite an inordinately large literature focusing on 2003 recall of California Governor Gray Davis, 
		discussing a wide range of relevant features in this particular recall such as 
		voter turnout
		\autocite{arbourVoterTurnoutCalifornia2005},
		voter choice 
		\autocite{alvarezRationalityRationalisticChoice2009,shawStrategicVotingCalifornia2005},
		candidate choice
		\autocite{mcgheeRoleCandidateChoice2009},
		partisan coordination and mobilization,
		\autocite{masketCaliforniaRecallSprint2016},
		and
		voter polarization along
		racial, and ethnic
		lines
		\autocite{seguraRaceRecallRacial2008}.
		Understandably, the 2003 California gubenatorial recall engenders little scholarly
		discussion of how it affects incumbents, 
		due to not only the immediate removal of the incumbent Governor Davis and 
		the
		installment of new California Governor Arnold Schwarzenegger
		but also 
		how unlikely such a recall attempt
		would ever succeed again in this solidly Democratic state,
		which is further corroborated by the recent failed attempt to recall California Governor Newsom Gavin.
	}\footnote
	{
		For a thorough review of the practice and literature of recalls outside the United States,
		see Welp and Whitehead \autocite*{welpPoliticsRecallElections2020}.
	}
	Those who do attempt to understand their effect on incumbents report mixed results, 
	ranging from evidence for limited accountability
	\autocite{okamotoRecallJapanMeasure2020},
	to
	no evidence for accountability
	\autocite{welpRecallReferendumsPeruvian2016},
	or worse,
	evidence for decline in accountability
	\autocite{hamanRecallElectionsTool2021}.
	
	
%	To further complicate matters,
	In addition,
%	a number of methodological challenges used to
	some of these studies rest on theoretical assumptions
	the utility of which are yet to be confirmed,
	preventing us from convincingly asserting
	whether recall improves accountability.
%	the evidence cited in support or opposition to this claim 
%	daunting methodological challenges lie ahead.
	Since recall elections are rather infrequent,
	researchers have not only sought to explain
	the frquency and success rates of recalll
	but also used them
	as the metrics for inferring 
	the extent to which elected officials are held accountable
	\autocite{hamanRecallElectionsTool2021,okamotoRecallJapanMeasure2020,qvortrupHastaVistaComparative2011}.
%	since the conventional metrics th.
	Such attempt,
	though understandable given the rarity of recall elections and the accompanying limitation on data,
	seems to assume that the more often recall mechanisms get triggered,
	the more effectively it holds elected officials accountable,
	which is in direct conflict with
	the theory of recall acting as a stopvalve 
	\autocite{serdultHistoryDormantInstitution2015, welpRecallDemocraticAdvance2020},
	or as a way for electoral losers to retaliate
	\autocite{welpRecallReferendumsPeruvian2016},
	implying that frequent use of recall is a sign that elections are failing voters and thus
	not necessarily an indicator that recall itself is improving accountability.
	Though it may be possible to reconcile these competing theories by 
	categorizing recall attempts and 
	specify the scope condition of each theory,
	doing so would not only
	negate the usefulness of frequency of all different types of recall attempts and success .
	but also
	introduce questions as to the propriety of criteria for categorization. 
	It appears that the prevailing attention to recall frequency
	might just create more answers than questions.
	
%	Although this is a fairly well treaded path for students of other democratic institutions,
%	it has only been recently attempted by taken by 
	In lieu of attempting to understand the frequency of recalls and their significance to accountability,
	Gordon and Yntiso \autocite*{gordonIncentiveEffectsRecall2021a}
	take a different approach
	where they systematically examine
	incentive effects of recall elections on incumbents.
	Though familiar to most neoinstitutionalists,
	Gordon and Yntiso's approach is the first of its kind in the literature of recall election,
	successfully identify the \textit{threat of recall}
	as the cause of change in behavior of elected officials, 
	the sentencing severity of elected judges to be more specific.
	At the same time 
	Gordon and Yntiso \autocite*{gordonIncentiveEffectsRecall2021a} also recognize that the scope condition of the incentive effect has yet to be determined
	and thus remains a gap in the recall literature,
	especially given that elected judges are different from most other elected politicians in many ways.
	Elected judges compete in a legally and cultrually nonpartisan conditions,
	seeking to be above the fray of politics and are at best somewhat sensitive to what their constitutents want;
	whereas most politicians do politics for a living,
	and their behavior must account for the preferences of both their consituents and their parties,
	making it difficult to say for certain 
	whether this crosswinds of pressures combined with the threat of recall
	produces any net behavioral change.
	In a nutshell, our understandning of whether recall enhances
	politicians' accountability
	is still far from complete.
%	Another pertinent difference between judges and other elected officials concerns
%	their respective menu of behavioral options.
%	The normative expectation of impartiality forces judges to
%	resort to adjusting the severity of sentences in response to public pressure;
%	on the other hand, elected politicians are spoiled for choice,
%	which may prevent us from conclusively state that
%	 the attempt to 

	To render a more complete understanding of recall's ability to enhance accountability of incumbents
	beyond the settings of recalls of judges, who are nonpartisan officials. I shall use eviedence from Taiwan's Legislature Yuan which produced a series of political reforms that create two tiers of legislators who were effectively immune from recall before the new legislation that, for reasons unrelated to the composition of each tier of legialtors, made only one tier of legislators vulnerable to credible threat of recall, thus allowing us to use difference-in-difference method to determine whether the incentive effects of the introduction of new recall mechanisms causes the incumbent legislators to 
	(1) be less likely to toe the party line
	and
	(2)
	more likely to behave in accordance with the preferences of their constituents
	thus enhancing their accountability to constituencies..
	
	In addition to the statistical assessment
	I will 
	also present a ``prototypical" recall mechanism.
	and its scope condition which undergirds the theoretical development of the incentive effect of recall.
	This will help isolate the incentive effects of various typical components of recall.
	Though it is my goal to
	sketch a theory of recall elections applicable across various institutional arrangements,
	the diversity of these arrangements is almost at the same level as that of electoral systems around the world.
	Therefore, this ``prototypical" recall mechanism is bound to be quite distinct from that found elsewhere.
	In fact, it will become clear that this prototype is a bit different from the actual recall mechanism found in Taiwan and most US state and local jurisdiction.
	Reader should bear in mind that this prototype only serves a starting point for developing the incentive theory and should be adjusted when the difference between the particular feature of recall under study and the prototype justifies it.
	
	
%	Therefore, rather than focusing on the frequency of recall attempts and successes
%	and
%	their impact on politicians,
%	I shall assess
%	how recall mechanism itself, rather than instances of recall attempts and elections,
%	\textit{directly}
%	changes incumbents' behavior
%	by reshaping their incentive structure.
	%	I shall focus more on how politicians respond to 
	%	as a result of not so much the incentive structures 
	%	rather than instances of recall,
	%	which may provide a more fertile
%	I shall attempt to assess the extent to which introud 
	
	
%	Consequently, there have 
%%	Recalls elections aren't so effective
%	However, this expectation has yet to be corroborated.
%	Not only are recalls rarely triggered in general across the world,
%	which make their .
%	the current literature of recall elections have not born out this expectation that introducing recall makes politicians that are more compliant with their constituencies.
%	Generally speaking, recall is rarely triggered at all, thus introducing
%	The volume of the literature is already miniscule compared with that focued on conventional elections due to the infrequency of recall elections which put a limit on both
%	the interest of the discipline at large and the methodological approaches and rigor one would like to apply.
%	Even when recalls do undergo scholarly scrutiny, 
%	students of recall elections generally agree that parties continue to play an important role
%	not only in coordinating campaign efforts whereas accountability for individual politicians have failed to materialize.
%	This is in direct contradiction to not only one of the major motivations for introducing recall elections and the theory of elections,
%	which both predict that recalls should improve elected officials' accountability to their constitutents.
%%	\cite{}
%	In fact it has been
%	It almost seems that parties have managed to (re)assert themselves as the main driver for campaigns and outcomes of recall elections.
%	
%%	Incentive effect of recalls, first studied in judicial elections, parties' role unknown
%	Nevertheless,  all is not lost for those who entertain hope that recall mechanism would make politicians more accountable to their constituents than their parties. 
%	
%	Therefore, I seek to revive the contention that parties 
	
%	However, aside from the trailblazing work of @gordon2021 on recall elections' effect on elected judges in California,
%	systematic assessment of the effect of recall elections on incumbents' behavior is essentially non-existent.
%	In addition to the methodological challenges mentioned by @gordon2021,
%	perhaps another important reason lies in the lack of credible threat due to prevalent institutional obstacles of recalling an elected official.
%	In the case of judicial recalls in California,
%	the high number of signatures required to trigger one,
%	combined with its rarity and the resultant norm against recalling judges, have historically made the threat of recall very much negligible.
%	It is no wonder that Californian judges only just began to take the threat of recall seriously, even though this mechanism has been on the statute book for decades.
% 	In addition, due to the nonpartisan nature of judicial elections and recalls in California, @gordon2021 did not address the role of party of recall elections.
% 	Furthermore, unlike judges [@gordon2021, p.7], most elected officials go out of their way to supply voters with verifiable information on which voters may evaluate their performance.
 	
 	

	
%	Fortunately, on the other side of the Pacific Ocean,
%	Taiwan has been experimenting with new legislation passed in 2016 that made it much easier for voters to trigger recall elections of the vast majority of elected officials on every level of government.
%	Recall elections became a recurring event.
%	I attempt to evaluate the extent to which incumbent legislators behave differently as a result of the new recall rules by using difference-in-difference (DiD) approach which leverages Taiwan's mixed-member electoral system.
%	In Taiwan, each voter gets two votes at a legislative election:
%	one vote goes toward electing the legislator representing the local constituency under the system of first-past-the-post (FPTP),
%	and the other one goes toward selecting a party list in a single national at-large constituency under the system of proportional representation (PR).
%	Since dual candidacy in both contests is not permitted,
%	legislators face a distinct set electoral incentives unlike other mixed-member electoral system where legislators fail to win the hearts and minds of local constituents could still be returned to the legislature by the strength of their parties and their positions on the party lists in a larger constituency.
%	Furthermore, perhaps due to the institutional setup where parties rather than the local constituents that choose the legislators in the PR tier, they can only be removed by the party itself and could never be recalled by the voters before and after the passage of the new recall election.
%	This allows us to treat the legislators in the PR tier as the control group and those in the FPTP tier the treatment group.
%	And the difference in behavior between these two groups of legislators could allow us to identify the causal effect of the legislation that made recall easier.
	
	
	
%	How can voters make sure that elected officials are on their best
%	behavior? We can monitor them, and credibly threaten to vote them out
%	when we don't like them anymore. However, the threat isn't always
%	credible. When the election day comes, forces of partisan
%	identification, incumbent effect, among numerous factors, may save their
%	skin after all. Perhaps more importantly, voters may want to punish the
%	officials sooner than the next election. To what other remedies can they
%	resort?
%	
%	The answer given by voters in the city of angels more than a century ago
%	is recall elections. To check the corruption in local government, a
%	ballot measure was in the Los Angeles County passed to allow voters to
%	hold recall municipal officials when the number of eligible voters who
%	demand the recall passes a certain threshold. The same device was later
%	on introduced to the entire California and spread to other US states and
%	countries across the world. However, systematic analysis of recall
%	elections are rare because recall elections themselves are even rarer.
%	Most of the extant literature on recall elections focused on just one
%	recall elections: the 2003 California gubernatorial recall. Furthermore,
%	aside from the trailblazing work of Gordon and Yntiso (2021) on recall
%	elections' effect on elected judges in California, systematic assessment
%	of the effect of recall elections on incumbents' behavior is essentially
%	non-existent. Aside from the methodological challenges mentioned by
%	Gordon and Yntiso (2021), perhaps another important reason lies in the
%	lack of credible threat due to prevalent institutional obstacles of
%	recalling an elected official. In the case of judicial recalls in
%	California, the high number of signatures required to trigger one,
%	combined with its rarity and the resultant norm against recalling
%	judges, have historically made the threat of recall very much
%	negligible. It is small wonder that Californian judges only just began
%	to take the threat of recall seriously, even though this mechanism has
%	been on the statute book for nearly a century.
%	
%	In addition, unlike judges who are supposed to be fair and unswayed by
%	public opnions as much as possible, most elected officials are supposed
%	to represent and internalize the opinions of a number of political
%	actors. Therefore, while it is reasonable to expect judicial recalls to
%	cause judges to consider the opinions of their constituents who are in
%	all likelihood the only group of political actors that matter to their
%	decision process, the extent to which elected officials consider
%	constituents' opinions remains to be seen when they are under the
%	pressure of a number of political actors, say parties, pressure groups,
%	other than their constituents that push them in different directions.
%	
%	Fortunately, for students of direct democracy, a new experiment is under
%	way on the other side of the Pacific Ocean, presenting us a better
%	opportunity to probe the effect of recall elections where incumbents
%	face pressure from a number of different directions. In 2016, Taiwan
%	passed new legislation that made it much easier for voters to trigger
%	recall elections of the vast majority of elected officials on every
%	level of government. Recall elections has become a recurring event since
%	then, and the threat to recall elected officials in Taiwan appears more
%	credible.
%	
%	I propose to evaluate the extent to which incumbent legislators behave
%	differently as a result of the new recall rules by using
%	difference-in-difference (DiD) approach which leverages Taiwan's
%	mixed-member electoral system. In Taiwan, each voter gets two votes at a
%	legislative election: one vote goes toward electing the legislator in a
%	single-member district (SMD) under the system of first-past-the-post
%	(FPTP), and the other one goes toward selecting a party list in a single
%	national multi-member district (MMD) under the system of closed-list
%	proportional representation (CLPR). Since dual candidacy in both
%	contests is not permitted, legislators face a distinct set electoral
%	incentives unlike other mixed-member electoral system where legislators
%	fail to win the hearts and minds of local constituents could still be
%	returned to the legislature by the strength of their parties and their
%	positions on the party lists in a larger constituency. Furthermore,
%	perhaps due to the institutional setup where parties rather than the
%	local constituents that choose the legislators in the PR tier, they can
%	only be removed by the party itself and could never be recalled by the
%	voters before and after the passage of the new recall election. This
%	allows us to treat the legislators in the PR tier as the control group
%	and those in the FPTP tier the treatment group. I propose to choose
%	measure four kinds of legislative behavior: roll call votes, bill
%	co-sponsorship, three-minute speeches, and ministers' question time to
%	detect whether legislators change their behavior because it is easier to
%	recall them.
	
%	\newpage
%	\printbibliography
	

\end{document}

