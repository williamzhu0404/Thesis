% !TeX root = lit_review.tex

\documentclass[hyphens, crop=false]{standalone}
\usepackage{import}
\usepackage[subpreambles=true]{standalone}

\usepackage[
reflist=true,
%noreprint, 
natbib=true,
autocite=inline,
style=windycity]
{biblatex}


\usepackage[american]{babel}
\usepackage{csquotes}

\usepackage{url}


\usepackage[multiple]{footmisc}

\usepackage{setspace}
\doublespacing



\addbibresource{lit_review.bib}


\begin{document}
	
	
	
	
	
	\subsection*{Roadmap}
	\begin{enumerate}
		\item 
		General Theory
		
		Recall is just another election
		
		Voting theory in elections
		\begin{enumerate}
			\item 
			Theory of voting behavior
			\item 
			Theory of personal vote and party vote
			\item 
			Theory of recall
		\end{enumerate}
		\item 
		Scope Condition
		\begin{enumerate}
			\item 
			Prediction of individual politician behavior theoretically more straightforward
			\item 
			Observation of incentive effect on individual behavior more likely to be significant
			\item 
			Focus on recall of individuals instead of collectivities
			\item 
			Recall of individuals compatible only with candidate-based electoral system
		\end{enumerate}
		\item 
		Stylized instittutional arrangement of recall
		\begin{enumerate}
			\item 
			Similar to institutional arrangement of recall in Taiwan
			\item 
			No extraneous elements
			\item 
			Separate component
			\item 
			Similar to institutional arrangement of recall elsewehere
		\end{enumerate}
		\item 
		Main Theory
		\begin{enumerate}
			\item 
			Recall causes parties and individuals to agree that personal vote is needed
			\item 
			Parties need policies and seats, individuals need reelection
			\item 
			Recall makes policy more difficult to implement
			as recall could reduce the size of legislative party.
			\item 
			Without recall, parties could maintain their size through buying off voters with policy closer to the election
			\item 
			With recall, members of legislative parties are subject to swift retribution through recall.
		\end{enumerate}
		\item 
		Alternative Theory
		\begin{enumerate}
			\item 
			Recall leads to conflict between parties and individual politicians, relative strengths of parties and individuals would matter
			\item 
			Fear of resource depletion
			\item 
			Fear of credible signal of unpopularity
			\item 
			Fear of losing election
			\item 
			We unfortunately cannot determine which mechanism at work, we may only exclude certain mechanisms if our statistical asssessment may ascertain whether recall has incentive effect
		\end{enumerate}
	\end{enumerate}
	\newpage
	
	\subsection*{Setting the Scope Condition}
		
		\subsubsection*{Personal Vote and Party Vote}
			\import{./modules/scope}{personal_vote_definition}
		
		\subsubsection*{Relationship between Personal and Party Votes}
%			\import{./modules/scope}{personal_vs_party}
		
		\subsubsection*{Personal Vote in Recall Elections}
			
%			\import{./modules/scope}{personal_recall}
		
		\subsubsection*{Personal Vote across Electoral System}
			
%			\import{./modules/scope}{personal_electoral_system}
		
		\subsubsection*{Scope Condition of this Study}
		
%			\import{./modules/scope}{scope_condition}
	
	\subsection*{Stylized institution of Recall}
		
		\subsection*{Description of Stylized Institution}
%			\import{./modules/scope}{personal_vote_definition}
		
	\subsection*{Main Theory: Agreement Model}
	
		\subsection*{A Game of Policy Sequencing}
%			\import{./modules/agree}{personal_vote_definition}
		
		\subsection*{Policy Sequencing without Recall}
%			\import{./modules/agree}{personal_vote_definition}
		
		\subsection*{Policy Sequencing with Recall}
%			\import{./modules/agree}{personal_vote_definition}
	
	\subsection*{Alternative Theory: Conflict Model}
	
		\subsection*{Signal of Unpopularity}
%			\import{./modules/agree}{personal_vote_definition}
	
		\subsection*{Depleteion of Resources}
%			\import{./modules/agree}{personal_vote_definition}

	
		
	\newpage
	\printbibliography
	
	
\end{document}

