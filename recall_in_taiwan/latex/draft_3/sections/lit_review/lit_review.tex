% !TeX root = lit_review.tex

\documentclass[hyphens, crop=false]{standalone}
\usepackage{import}
\usepackage[subpreambles=true]{standalone}

\usepackage[
reflist=true,
%noreprint, 
natbib=true,
autocite=inline,
style=windycity]
{biblatex}


\usepackage[american]{babel}
\usepackage{csquotes}

\usepackage{url}


\usepackage[multiple]{footmisc}

\usepackage{setspace}
\doublespacing



\addbibresource{lit_review.bib}


\begin{document}
	
	
	
	
	
	\subsection*{Roadmap}
	\begin{enumerate}
		\item 
		General Theory
		
		Recall is just another election
		
		Voting theory in elections
		\begin{enumerate}
			\item 
			Theory of voting behavior
			\item 
			Theory of personal vote and party vote
			\item 
			Theory of recall
		\end{enumerate}
		\item 
		Scope Condition
		\begin{enumerate}
			\item 
			Prediction of individual politician behavior theoretically more straightforward
			\item 
			Observation of incentive effect on individual behavior more likely to be significant
			\item 
			Focus on recall of individuals instead of collectivities
			\item 
			Recall of individuals compatible only with candidate-based electoral system
		\end{enumerate}
		\item 
		Stylized instittutional arrangement of recall
		\begin{enumerate}
			\item 
			Similar to institutional arrangement of recall in Taiwan
			\item 
			No extraneous elements
			\item 
			Separate component
			\item 
			Similar to institutional arrangement of recall elsewehere
		\end{enumerate}
		\item 
		Main Theory
		\begin{enumerate}
			\item 
			Recall causes parties and individuals to agree that personal vote is needed
			\item 
			Parties need policies and seats, individuals need reelection
			\item 
			Recall makes policy more difficult to implement
			as recall could reduce the size of legislative party.
			\item 
			Without recall, parties could maintain their size through buying off voters with policy closer to the election
			\item 
			With recall, members of legislative parties are subject to swift retribution through recall.
		\end{enumerate}
		\item 
		Alternative Theory
		\begin{enumerate}
			\item 
			Recall leads to conflict between parties and individual politicians, relative strengths of parties and individuals would matter
			\item 
			Fear of resource depletion
			\item 
			Fear of credible signal of unpopularity
			\item 
			Fear of losing election
			\item 
			We unfortunately cannot determine which mechanism at work, we may only exclude certain mechanisms if our statistical asssessment may ascertain whether recall has incentive effect
		\end{enumerate}
	\end{enumerate}
	\newpage
	
	\subsection*{Setting the Scope Condition}
		Having laid out the context of recall in Taiwan,
		it may help to zoom out for a moment by noting that
		recall mechanisms around the world can be very dissimilar to each other,
		and as such should limit the ambition of any theory of recall.
		One of the most substantial differences concerns
		whether it is a collectivity or an individual that can be recalled.
		Though recall of collectivity, such as an entire legislative or executive council,
		is rather rare indeed
		\autocite{welpRecallReferendumWorld2020}
		and may be simply excluded for this reason alone,
		I would also like to argue that the scope condition should exclude
		recall of collecitivty to create a more tractable theoretical foundation
		for the rest of the paper.
%		I will develop a theory of recall's incentive effect that is
%		specific to the scope condition
%		that only includes mechanisms ofrecall of individuals
%		Although I hope a theory applicable to all recall mechanisms
%		may be formulated and supported,
%		the nature of one of the key motivations for this study
%		makes it the most tractble approach.
		As the reader may recall,
		one of my goals
		is to understand the net incentive effect of recall institution
		on incumbents in a partisan environment,
		the implicit assumption being that
		the incumbent's party and constituency
		may differ in their preferences and that
		the differences matter to the incumbent.
		It will soon become apparent that this assumption does not hold
		for recall of collectivity and thus justifies a different theory
		that cannot be developed here due to the word limit constraint.
		
		To begin understanding how incumbents respond to recall elections
		in a partisan environment,
		I deem it uncontroversial enough to assume that
		the incumbent
		is associated with a party
		in addition to the conventional assumption that the incumbent
		single-mindedly pursues reelection
		\autocite{mayhewCongressElectoralConnection1974}.
		Then a vote for or against the candidate
		entails the voter's consideration of
		the incumbent,
		the incumbent's party,
		or a combination of both.
		This provides a natural way of dividing the support for the incumbent at the polls
		into to two components:
		personal vote and party vote.
		%		Definition module begins
		%		\\
		%		Definition module begins
		\subsubsection*{Personal Vote and Party Vote}
		\import{./modules/scope}{personal_vote_definition}
		
		
		
%		\subsubsection*{Personal Vote and Party Vote}
%			Relationship module begins
%			\\
%			Relationship module begins
		
%		\subsubsection*{Relationship between Personal and Party Votes}
%			\import{./modules/scope}{personal_vs_party}
%			To discuss the personal vote versus party vote tradeoff
%			is like going down a rabbit hole, best not discuss it.
			
			
			personal vote in recall begins
			\\
			personal vote in recall begins
			
		
		\subsubsection*{Personal and Party Votes in Recall Elections}
		
			\import{./modules/scope}{personal_recall}
			
%			personal vote in electoral system begins
%			\\
%			personal vote in electoral system begins

		
		\subsubsection*{Personal and Party Votes across Electoral System}
			
			\import{./modules/scope}{personal_electoral_system}
		
		\subsubsection*{Scope Condition of this Study}
		
%			\import{./modules/scope}{scope_condition}
	\subsection*{Main Theory: Policy-Driven Model}
		
		I have indeed posited that recall induces personal vote,
		but I have not specified how the mechanism works.
		The primary goal,
		which mainly seeks to answer the question of whether
		recall induces personal vote seeking behavior in a partisan setting,
		is more exploratory in nature.
		Given that the reason for pursuing personal vote is complex,
		it is difficult to say for certain which mechanism is at work
		and which is not.
		
		Nevertheless, I would still like to propose a sketch of a model
		that describes how recall induces personal vote.
		The model begins with a simple model where
		the implementation of party policy alone across multiple periods
		determines the electoral fate of the copartisan incumbents.
		I will demonstrates necessity of personal vote
		by illustrating that introducing recall in the simple model
		precludes the implementation of certain extreme policies.
		
		
		
		\subsubsection*{Primacy Effect}
		
		Primacy factor
		
		experimental evidence
		
		political business cycle
		
	
		\subsubsection*{A Game of Policy Sequencing}
%			\import{./modules/agree}{personal_vote_definition}
		
		\subsubsection*{Policy Sequencing without Recall}
%			\import{./modules/agree}{personal_vote_definition}
		
		\subsubsection*{Policy Sequencing with Recall}
%			\import{./modules/agree}{personal_vote_definition}
		
		
	\subsection*{Stylized institution of Recall}
		There are multiple ways to model the threshold,
		I will not give a clear model of how to determine the threshold,
		but I will by giving a stylized model presents
		the hypotheses of how this threshold
		can be increased or decreased
	
	
		However, it must be noted that the above policy sequencing game is not active
		unless the threshold for recall is equal to threshold for reelection.
		
		The threshold itself will be modeled sometime later,
		in this stylized institution of recall,
		I will identify components that
		affects the threshold of reeelection/retention threshold.
		
		
		This assumption 
		
		\subsubsection*{Description of Stylized Institution}
		%			\import{./modules/scope}{personal_vote_definition}
		signature gatehering
		recall election campaign
		special election campaign
		
		\subsubsection*{Credibility of Recall Threats}
		
%		Signature threshold
		length of signature gathering,\\
		The longer the length the lower the threshold.
		
		percentage of signature,\\
		The lower the percentage,
		the lower the threshold
		
		absolute number of signature
		the lower the number
		the lower the threshold
		
		Absolute versus relative threshold
		Both probably matter
		At least
		\textit{ceteris paribus}
		threshold decrease in each of them
		
		
		threshold of recall/reelection
		recall election threshold
		special election threshold
		this is tricky, it depends on the electoral system
		
%		Recall threshold
		winning criteria
		
		
	
	\subsection*{Alternative Theory}
	
		\subsubsection*{Agreement Model}
		Composition of recall electorate different,
		may make sense to rely more on party vote.
		
		
		\subsubsection*{Conflict Model}
		
		
		Recall depletes resource,
		then incumbents could want to spend more to
		prevent recall which is more costly.
		COmpetition for resource
		
		Prevention of recall may derail more
		
		
		
		\subsubsection*{Miscellaneous}
		These mechanisms make work independently of or interact with the main theory,
		which posit that the party and individual incumbents should both agree that
		individual incumbents need to cultivate a stronger personal vote.
		The main point of departure here is that
		the agreement is no longer guaranteed
		as both party and individual incumbents
		are incentivized to pursue party vote and personal vote respectively.
		The incentive effect observed will be the
		outcome of more context-specific competition between
		party leadership and their members for resources and influence
		and as such does not travel well across different institutional settings
		where the power distribution vary.
	
		\subsection*{Signal of Unpopularity}
		
		Adverse selection problem
		
		A candidate wants to be seen as popular, capable of winning election,
		a recall may just help party/consituents to find out whether
		the candidate remain capable.
		
%			\import{./modules/agree}{personal_vote_definition}
	
		\subsection*{Depleteion of Resources}
%			\import{./modules/agree}{personal_vote_definition}

	
		
	\newpage
	\printbibliography
	
	
\end{document}

