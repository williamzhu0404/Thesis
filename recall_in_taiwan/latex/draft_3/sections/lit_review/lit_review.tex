% !TeX root = lit_review.tex

\documentclass[hyphens, crop=false]{standalone}
\usepackage{import}
\usepackage[subpreambles=true]{standalone}

\usepackage[
reflist=true,
%noreprint, 
natbib=true,
autocite=inline,
style=windycity]
{biblatex}


\usepackage[american]{babel}
\usepackage{csquotes}

\usepackage{url}


\usepackage[multiple]{footmisc}

\usepackage{setspace}
\doublespacing



\addbibresource{lit_review.bib}


\begin{document}
	[
	Instead of attempting to demonstrate that accountability always exists for all recall election.
	The purpose of this study is to use Taiwan as a baseline case in support
	of the theory that accountability driven by the incumbent's attempt to
	cultivate a personal vote must exist to a significant degree
	in for at least some institutions where electoral competition is largely partisan.
	]
	
	
	
	
%	\subsection*{Roadmap}
%	\begin{enumerate}
%		\item 
%		General Theory
%		
%		Recall is just another election
%		
%		Voting theory in elections
%		\begin{enumerate}
%			\item 
%			Theory of voting behavior
%			\item 
%			Theory of personal vote and party vote
%			\item 
%			Theory of recall
%		\end{enumerate}
%		\item 
%		Scope Condition
%		\begin{enumerate}
%			\item 
%			Prediction of individual politician behavior theoretically more straightforward
%			\item 
%			Observation of incentive effect on individual behavior more likely to be significant
%			\item 
%			Focus on recall of individuals instead of collectivities
%			\item 
%			Recall of individuals compatible only with candidate-based electoral system
%		\end{enumerate}
%		\item 
%		Stylized instittutional arrangement of recall
%		\begin{enumerate}
%			\item 
%			Similar to institutional arrangement of recall in Taiwan
%			\item 
%			No extraneous elements
%			\item 
%			Separate component
%			\item 
%			Similar to institutional arrangement of recall elsewehere
%		\end{enumerate}
%		\item 
%		Main Theory
%		\begin{enumerate}
%			\item 
%			Recall causes parties and individuals to agree that personal vote is needed
%			\item 
%			Parties need policies and seats, individuals need reelection
%			\item 
%			Recall makes policy more difficult to implement
%			as recall could reduce the size of legislative party.
%			\item 
%			Without recall, parties could maintain their size through buying off voters with policy closer to the election
%			\item 
%			With recall, members of legislative parties are subject to swift retribution through recall.
%		\end{enumerate}
%		\item 
%		Alternative Theory
%		\begin{enumerate}
%			\item 
%			Recall leads to conflict between parties and individual politicians, relative strengths of parties and individuals would matter
%			\item 
%			Fear of resource depletion
%			\item 
%			Fear of credible signal of unpopularity
%			\item 
%			Fear of losing election
%			\item 
%			We unfortunately cannot determine which mechanism at work, we may only exclude certain mechanisms if our statistical asssessment may ascertain whether recall has incentive effect
%		\end{enumerate}
%	\end{enumerate}
%	\newpage
	
	\subsection*{Some Definitions and Stylized Facts}
		I would like to begin this inquiry by making a simple observation that
		recall can be treated as just another election
		an incumbent needs to survive.
		This allows for appropriation of a number of findings and techniques
		that have been long established in electoral behavior literature,
		the first one of which being a simple stylized fact that
		each candidate with a realistic chance of winning an election belongs to a party.
		Then when constituents cast a vote for a candidate,
		one part of the decision is based on the candidate's personal tratis,
		and the other part depends on the performance of the candidate's party. 
		One may thus say that there is both
		a \textit{personal vote}
		and a \textit{party vote} for the candidate.
		
		
		%		\\
		%		Definition module begins
		\subsubsection*{Personal Vote and Party Vote}
		\import{./modules/stylized}{personal_vote_definition}
		
		\import{./modules/stylized}{stylized_fact}
		
		
	\subsection*{Setting the Baseline}
		
		\import{./modules/baseline}{baseline_intro}
		
		
		\subsubsection*{Recall in Taiwan as a Baseline}
		
		\import{./modules/baseline}{tw_electoral_reform}
		
		
		\subsubsection*{Sensitivity to Incentive Effect}
		
		
		
		\subsubsection*{Institutional Arrangements of Taiwan's Legislative Recall}
		
		\import{./modules/baseline}{scope_stylized}
		
		
		\subsubsection*{Stability of Party Vote}
		
		
		%		\subsubsection*{Personal Vote and Party Vote}
		%			Relationship module begins
		%			\\
		%			Relationship module begins
		
		%		\subsubsection*{Relationship between Personal and Party Votes}
		%			\import{./modules/baseline}{personal_vs_party}
		%			To discuss the personal vote versus party vote tradeoff
		%			is like going down a rabbit hole, best not discuss it.
		
		
		%		Definition module begins
			
			
%			personal vote in recall begins
%			\\
%			personal vote in recall begins
%			
%		
%		\subsubsection*{Personal and Party Votes in Recall Elections}
%		
%			\import{./modules/baseline}{personal_recall}
			
%			personal vote in electoral system begins
%			\\
%			personal vote in electoral system begins

		
		
		
		
%			\import{./modules/baseline}{scope_condition}
	\subsection*{Main Theory}
		\import{./modules/recency}{recency_vs_alternative}
		
		\subsubsection*{Recency Bias}
			\import{./modules/recency}{recency_bias}
		
	
		\subsubsection*{A Game of Policy Sequencing}
			% model written with words
			\import{./modules/recency}{policy_sequencing_informal}
%			% model written with maths
%			\import{./modules/recency}{policy_sequencing_formal}
			
			\import{./modules/recency}{personal_vote_definition}
		
%		\subsubsection*{Policy Sequencing without Recall}
%%			\import{./modules/agree}{personal_vote_definition}
%		
%		\subsubsection*{Policy Sequencing with Recall}
%%			\import{./modules/agree}{personal_vote_definition}
		
	\subsection*{Alternative Theories}
		
		\subsubsection*{Lacking Credible Threat of Recall}
			\import{./modules/alternative}{lacking_credible_threat}
			
		\subsubsection*{Fear of Resource Depletion}
			\import{./modules/recency}{resource_depletion}
		
	
		
	\newpage
	\printbibliography
	
	
	
	%		\subsubsection*{Agreement Model}
	%		Composition of recall electorate different,
	%		may make sense to rely more on party vote.
	%		
	%		
	%		\subsubsection*{Conflict Model}
	%		
	%		
	%		Recall depletes resource,
	%		then incumbents could want to spend more to
	%		prevent recall which is more costly.
	%		COmpetition for resource
	%		
	%		Prevention of recall may derail more
	%		
	%		
	%		
	%		\subsubsection*{Miscellaneous}
	%		These mechanisms make work independently of or interact with the main theory,
	%		which posit that the party and individual incumbents should both agree that
	%		individual incumbents need to cultivate a stronger personal vote.
	%		The main point of departure here is that
	%		the agreement is no longer guaranteed
	%		as both party and individual incumbents
	%		are incentivized to pursue party vote and personal vote respectively.
	%		The incentive effect observed will be the
	%		outcome of more context-specific competition between
	%		party leadership and their members for resources and influence
	%		and as such does not travel well across different institutional settings
	%		where the power distribution vary.
	%	
	%		\subsection*{Signal of Unpopularity}
	%		
	%		Adverse selection problem
	%		
	%		A candidate wants to be seen as popular, capable of winning election,
	%		a recall may just help party/consituents to find out whether
	%		the candidate remain capable.
	%		
	%%			\import{./modules/agree}{personal_vote_definition}
	%	
	%		\subsection*{Depleteion of Resources}
	%%			\import{./modules/agree}{personal_vote_definition}
	%		Within the scope condition
	%		and
	%		based on the stylized recall device sketched above,
	%		I now seek to develop a theory of how recall
	%		induces politicians,
	%		more specificially legislators associated with parties,
	%		to change their behavior
	%		with a view to cultivate stronger personal vote.
	%		In addition to Mayhew's classical explanation
	%		whereby
	%		incumbents,
	%		unable to influence their
	%		parties' popularity among the entire electorate,
	%		are left with but one option
	%		to focus on cultivating personal vote,
	%		I shall argue
	%		that recall further incentivizes this type of behavior by
	%		its ability to bring in
	%		voters' judgment much faster than
	%		general elections could,
	%		especially when the incumbent or the incumbent's party
	%		move away from the aggregate preference.
	%		
	%		Recognizing
	%		the theoretical objection that
	%		personal vote may not matter to legislators
	%		as
	%		parties' influence on
	%		crafting policies and waging campaigns
	%		is generally dominant,
	%		I will seek to demonstrate that
	%		recall,
	%		punishes the legislative party
	%		by voting out more vulnerable 
	%		
	%		 and increasingly so in the
	%		US where partisan polarization is on the rise,
	%		I will illustrate,
	%		that a purely partisan
	%		
	%		
	%		in the same spirit of Down's voting paradox
	%		which demonstrates the deficiency of purely strategic voting behavioral model,
	%		that party vote alone would be insufficient to deal with
	%		recall's pressure on parties.
	%		Parties 
	%		
	%		
	%		Toward that end,
	%		I would like to demonstrate it in a,
	%		shall I say,
	%		semi-formal model
	%		that illustrates not only the 
	%		
	%		
	%		
	%		I have indeed posited that recall induces personal vote,
	%		but I have not specified how the mechanism works.
	%		The primary goal,
	%		which mainly seeks to answer the question of whether
	%		recall induces personal vote seeking behavior in a partisan setting,
	%		is more exploratory in nature.
	%		Given that the reason for pursuing personal vote is complex,
	%		it is difficult to say for certain which mechanism is at work
	%		and which is not.
	%		
	%		Nevertheless, I would still like to propose a sketch of a model
	%		that describes how recall induces personal vote.
	%		The model begins with a simple model where
	%		the implementation of party policy alone across multiple periods
	%		determines the electoral fate of the copartisan incumbents.
	%		I will demonstrates necessity of personal vote
	%		by illustrating that introducing recall in the simple model
	%		precludes the implementation of certain extreme policies.
	
%	
%	
%	
%	One of the most substantial differences concerns
%	whether it is a collectivity or an individual that can be recalled.
%	Though recall of collectivity, such as an entire legislative or executive council,
%	is rather rare indeed
%	\autocite{welpRecallReferendumWorld2020}
%	and may be simply excluded for this reason alone,
%	I would also like to argue that the scope condition should exclude
%	recall of collecitivty to create a more tractable theoretical foundation
%	for the rest of the paper.
%	%		I will develop a theory of recall's incentive effect that is
%	%		specific to the scope condition
%	%		that only includes mechanisms ofrecall of individuals
%	%		Although I hope a theory applicable to all recall mechanisms
%	%		may be formulated and supported,
%	%		the nature of one of the key motivations for this study
%	%		makes it the most tractble approach.
%	As the reader may recall,
%	one of my goals
%	is to understand the net incentive effect of recall institution
%	on incumbents in a partisan environment,
%	the implicit assumption being that
%	the incumbent's party and constituency
%	may differ in their preferences and that
%	the differences matter to the incumbent.
%	It will soon become apparent that this assumption does not hold
%	for recall of collectivity and thus justifies a different theory
%	that cannot be developed here due to the word limit constraint.
%	
%	To begin understanding how incumbents respond to recall elections
%	in a partisan environment,
%	I deem it uncontroversial enough to assume that
%	the incumbent
%	is associated with a party
%	in addition to the conventional assumption that the incumbent
%	single-mindedly pursues reelection
%	\autocite{mayhewCongressElectoralConnection1974}.
%	Then a vote for or against the candidate
%	entails the voter's consideration of
%	the incumbent,
%	the incumbent's party,
%	or a combination of both.
%	This provides a natural way of dividing the support for the incumbent at the polls
%	into to two components:
%	personal vote and party vote.
	
	
	
	
	
	
	
\end{document}

