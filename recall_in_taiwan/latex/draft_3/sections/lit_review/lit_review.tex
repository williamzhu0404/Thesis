% !TeX root = lit_review.tex

\documentclass[hyphens, crop=false]{standalone}
\usepackage{import}
\usepackage[subpreambles=true]{standalone}

\usepackage[
reflist=true,
%noreprint, 
natbib=true,
autocite=inline,
style=windycity]
{biblatex}


\usepackage[american]{babel}
\usepackage{csquotes}

\usepackage{url}


\usepackage[multiple]{footmisc}

\usepackage{setspace}
\doublespacing



\addbibresource{lit_review.bib}


\begin{document}
	[
	Instead of attempting to demonstrate that accountability always exists for all recall election.
	The purpose of this study is to use Taiwan as a baseline case in support
	of the theory that accountability driven by the incumbent's attempt to
	cultivate a personal vote must exist to a significant degree
	in for at least some institutions where electoral competition is largely partisan.
	]
	
	
	
	
	\subsection*{Roadmap}
	\begin{enumerate}
		\item 
		General Theory
		
		Recall is just another election
		
		Voting theory in elections
		\begin{enumerate}
			\item 
			Theory of voting behavior
			\item 
			Theory of personal vote and party vote
			\item 
			Theory of recall
		\end{enumerate}
		\item 
		Scope Condition
		\begin{enumerate}
			\item 
			Prediction of individual politician behavior theoretically more straightforward
			\item 
			Observation of incentive effect on individual behavior more likely to be significant
			\item 
			Focus on recall of individuals instead of collectivities
			\item 
			Recall of individuals compatible only with candidate-based electoral system
		\end{enumerate}
		\item 
		Stylized instittutional arrangement of recall
		\begin{enumerate}
			\item 
			Similar to institutional arrangement of recall in Taiwan
			\item 
			No extraneous elements
			\item 
			Separate component
			\item 
			Similar to institutional arrangement of recall elsewehere
		\end{enumerate}
		\item 
		Main Theory
		\begin{enumerate}
			\item 
			Recall causes parties and individuals to agree that personal vote is needed
			\item 
			Parties need policies and seats, individuals need reelection
			\item 
			Recall makes policy more difficult to implement
			as recall could reduce the size of legislative party.
			\item 
			Without recall, parties could maintain their size through buying off voters with policy closer to the election
			\item 
			With recall, members of legislative parties are subject to swift retribution through recall.
		\end{enumerate}
		\item 
		Alternative Theory
		\begin{enumerate}
			\item 
			Recall leads to conflict between parties and individual politicians, relative strengths of parties and individuals would matter
			\item 
			Fear of resource depletion
			\item 
			Fear of credible signal of unpopularity
			\item 
			Fear of losing election
			\item 
			We unfortunately cannot determine which mechanism at work, we may only exclude certain mechanisms if our statistical asssessment may ascertain whether recall has incentive effect
		\end{enumerate}
	\end{enumerate}
	\newpage
	
	\subsection*{Setting the Baseline}
		Having laid out the context of recall in Taiwan,
		it may help to zoom out for a moment by noting that
		recall mechanisms around the world can be very dissimilar to each other,
		and as such should limit the ambition of any theory of recall.
		In fact,
		unless accounted for in full detail,
		variations in institutional arrnagement
		could very well prevent the discovery of any net incentive effect
		as parties learn to adapt and counteract the direct incentive effect of recall it might have.
		Such an endeavor has to await another time
		as the length constraint of this study does not permit it.
		Instead of searching for a theory that explains every kind of extant recall institution,
		I will attempt to use legislative recall in Taiwan as a baseline case that demonstrates
		that recall's incentive effect can hold incumbents more accountable
		by inducing them to cultivate a stronger \textit{personal vote}
		and then seek to explain why it may not be the case when
		certain components of recall's institutional arrangements change.
		%		\\
		%		Definition module begins
		\subsubsection*{Personal Vote and Party Vote}
		\import{./modules/scope}{personal_vote_definition}
		
		\subsubsection*{Recall in Taiwan as a Baseline}
		The conflict between the pursuit of personal vote and party vote
		constitutes the major difficulty underlying the key motivation of this study
		- to determine recall's net incentive effect on incumbents in a partisan electoral environment.
		Incumbents may well want to pursue personal vote to their hearts' content,
		but they may not do so when their party
		may prevent the full extent of their pursuit of personal vote since
		the goal of implementing policy and thus their ultimate goal of winning reelection
		requires each member to contribute to the public good their party supplies
		by submitting to the party discipline.
		Consequently, one may very well fail to observe any incentive effect of institutions
		when parties counteracts it by incentivizing members to behave otherwise.
		Though accountability may still be achieved indirectly through party,
		it is by no means guaranteed considering that
		partisan vote rests in large part on voters' partisan identification
		which is remarkably stable against the performance of the incumbent or the incumbent's party
		resulting in no observation of any net incentive effect.
		
		A way to counteract this difficulty resulting from theory and
		the attending measurement issues is to
		find a recall mechanism that encourages pursuit of personal vote and discourages that of party vote to a great extent.
		Such an approach stands the best chance
		of measuring signficiant net incentive effect that
		reveals the existence of recall's direct incentive effect as the baseline
		and potentially allow one to work out exactly what is counteracting this effect if the effect exists.
		This is where Taiwan's legislative recall comes in as one of the best possible baseline case.
		I will demonstrate that
		in addition to its credible threat already seen in the above description of recall's context and practice in Taiwan,
		recall elections,
		with a
		design that all but maximizes incentives for cultivating a personal vote,
		operating in a candidate-centered electoral system which
		features a stable party system,
		provides some of the best baseline case.
		
		\subsubsection*{Sensitivity to Incentive Effect}
		
		To measure the direct incentive effect of recall requires the incumbents,
		it is imperative that legislators not only pursue personal vote
		but also remain sensitive to insitutitonal changes such that
		their personal-vote seking behavior adjusts accordingly.
		Such concern is especially prevalent in studies of institutional reforms
		where a frequently leveled criticism is that
		institutional inertia may prevent one from efficiently measuring any behavioral changes
		due to institutional reforms until years later or at all
		especially when parties discourage such behavioral changes.
		
		\
		
		To illustrate that such conditions are very likely to hold,
		I'd like to direct reader to the evolution of the electoral system of
		Taiwan's Legislative Yuan.
		When democratization first took place,
		Legislative Yuan's members were elcted by the single non-transferable vote (SNTV) system.
		where seats are awrded to candidates winning the most votes
		in multi-member districts (MMDs), some of which return as many as 10 candidates to the legislature.
		This electoral system promotes personal vote to a far greater extent than
		other candidate-centered electoral system,
		yet simultaneously undermines the legislators' accountability to
		the local consitituencies who are represented by multiple legislators,
		making it rather difficult to effectively reward and punish incumbents
		at the poll.
		The 2008 electoral reform
		
		and thus
		electioneering for personal vote is on adrenaline,
		though potentially at the expense of accountability.
		
		To bolster confidence that personal vote seeking behavior exists
		and adjusts with institutional changes in Taiwan,
		
		
		to be sensitive to their local constituencies in the first place
		before recall reform is introduced.
		
		
		Taiwan's Legislative Yuan
		where all the members representing local constituencies
		are elected by the first-past-the-post (FPTP) system
		induces exactly this type of behavior.
%		In fact, before the electoral reform of 2008,
%		they are all elected 
		
		\subsubsection*{Institutional Arrangements of Taiwan's Legislative Recall}
		
		\import{./modules/scope}{scope_stylized}
		
		
		\subsubsection*{Stability of Party Vote}
		
		
		%		\subsubsection*{Personal Vote and Party Vote}
		%			Relationship module begins
		%			\\
		%			Relationship module begins
		
		%		\subsubsection*{Relationship between Personal and Party Votes}
		%			\import{./modules/scope}{personal_vs_party}
		%			To discuss the personal vote versus party vote tradeoff
		%			is like going down a rabbit hole, best not discuss it.
		
		
		
		One of the most substantial differences concerns
		whether it is a collectivity or an individual that can be recalled.
		Though recall of collectivity, such as an entire legislative or executive council,
		is rather rare indeed
		\autocite{welpRecallReferendumWorld2020}
		and may be simply excluded for this reason alone,
		I would also like to argue that the scope condition should exclude
		recall of collecitivty to create a more tractable theoretical foundation
		for the rest of the paper.
%		I will develop a theory of recall's incentive effect that is
%		specific to the scope condition
%		that only includes mechanisms ofrecall of individuals
%		Although I hope a theory applicable to all recall mechanisms
%		may be formulated and supported,
%		the nature of one of the key motivations for this study
%		makes it the most tractble approach.
		As the reader may recall,
		one of my goals
		is to understand the net incentive effect of recall institution
		on incumbents in a partisan environment,
		the implicit assumption being that
		the incumbent's party and constituency
		may differ in their preferences and that
		the differences matter to the incumbent.
		It will soon become apparent that this assumption does not hold
		for recall of collectivity and thus justifies a different theory
		that cannot be developed here due to the word limit constraint.
		
		To begin understanding how incumbents respond to recall elections
		in a partisan environment,
		I deem it uncontroversial enough to assume that
		the incumbent
		is associated with a party
		in addition to the conventional assumption that the incumbent
		single-mindedly pursues reelection
		\autocite{mayhewCongressElectoralConnection1974}.
		Then a vote for or against the candidate
		entails the voter's consideration of
		the incumbent,
		the incumbent's party,
		or a combination of both.
		This provides a natural way of dividing the support for the incumbent at the polls
		into to two components:
		personal vote and party vote.
		%		Definition module begins
			
			
%			personal vote in recall begins
%			\\
%			personal vote in recall begins
%			
%		
%		\subsubsection*{Personal and Party Votes in Recall Elections}
%		
%			\import{./modules/scope}{personal_recall}
			
%			personal vote in electoral system begins
%			\\
%			personal vote in electoral system begins

		
		
		
		
%			\import{./modules/scope}{scope_condition}
	\subsection*{Main Theory}
		\import{./modules/recency}{recency_vs_alternative}
		
		
		
		\subsubsection*{Recency Bias}
			\import{./modules/recency}{recency_bias}
		
	
		\subsubsection*{A Game of Policy Sequencing}
			% model written with words
			\import{./modules/recency}{policy_sequencing_informal}
%			% model written with maths
%			\import{./modules/recency}{policy_sequencing_formal}
			
%			\import{./modules/recency}{personal_vote_definition}
		
%		\subsubsection*{Policy Sequencing without Recall}
%%			\import{./modules/agree}{personal_vote_definition}
%		
%		\subsubsection*{Policy Sequencing with Recall}
%%			\import{./modules/agree}{personal_vote_definition}
		
	\subsection*{Alternative Theories}
		
		\subsubsection*{Lacking Credible Threat of Recall}
			\import{./modules/recency}{policy_sequencing_informal}
			
		\subsubsection*{Fear of Resource Depletion}
			\import{./modules/recency}{policy_sequencing_informal}
		
	
		
	\newpage
	\printbibliography
	
	
	
	%		\subsubsection*{Agreement Model}
	%		Composition of recall electorate different,
	%		may make sense to rely more on party vote.
	%		
	%		
	%		\subsubsection*{Conflict Model}
	%		
	%		
	%		Recall depletes resource,
	%		then incumbents could want to spend more to
	%		prevent recall which is more costly.
	%		COmpetition for resource
	%		
	%		Prevention of recall may derail more
	%		
	%		
	%		
	%		\subsubsection*{Miscellaneous}
	%		These mechanisms make work independently of or interact with the main theory,
	%		which posit that the party and individual incumbents should both agree that
	%		individual incumbents need to cultivate a stronger personal vote.
	%		The main point of departure here is that
	%		the agreement is no longer guaranteed
	%		as both party and individual incumbents
	%		are incentivized to pursue party vote and personal vote respectively.
	%		The incentive effect observed will be the
	%		outcome of more context-specific competition between
	%		party leadership and their members for resources and influence
	%		and as such does not travel well across different institutional settings
	%		where the power distribution vary.
	%	
	%		\subsection*{Signal of Unpopularity}
	%		
	%		Adverse selection problem
	%		
	%		A candidate wants to be seen as popular, capable of winning election,
	%		a recall may just help party/consituents to find out whether
	%		the candidate remain capable.
	%		
	%%			\import{./modules/agree}{personal_vote_definition}
	%	
	%		\subsection*{Depleteion of Resources}
	%%			\import{./modules/agree}{personal_vote_definition}
	%		Within the scope condition
	%		and
	%		based on the stylized recall device sketched above,
	%		I now seek to develop a theory of how recall
	%		induces politicians,
	%		more specificially legislators associated with parties,
	%		to change their behavior
	%		with a view to cultivate stronger personal vote.
	%		In addition to Mayhew's classical explanation
	%		whereby
	%		incumbents,
	%		unable to influence their
	%		parties' popularity among the entire electorate,
	%		are left with but one option
	%		to focus on cultivating personal vote,
	%		I shall argue
	%		that recall further incentivizes this type of behavior by
	%		its ability to bring in
	%		voters' judgment much faster than
	%		general elections could,
	%		especially when the incumbent or the incumbent's party
	%		move away from the aggregate preference.
	%		
	%		Recognizing
	%		the theoretical objection that
	%		personal vote may not matter to legislators
	%		as
	%		parties' influence on
	%		crafting policies and waging campaigns
	%		is generally dominant,
	%		I will seek to demonstrate that
	%		recall,
	%		punishes the legislative party
	%		by voting out more vulnerable 
	%		
	%		 and increasingly so in the
	%		US where partisan polarization is on the rise,
	%		I will illustrate,
	%		that a purely partisan
	%		
	%		
	%		in the same spirit of Down's voting paradox
	%		which demonstrates the deficiency of purely strategic voting behavioral model,
	%		that party vote alone would be insufficient to deal with
	%		recall's pressure on parties.
	%		Parties 
	%		
	%		
	%		Toward that end,
	%		I would like to demonstrate it in a,
	%		shall I say,
	%		semi-formal model
	%		that illustrates not only the 
	%		
	%		
	%		
	%		I have indeed posited that recall induces personal vote,
	%		but I have not specified how the mechanism works.
	%		The primary goal,
	%		which mainly seeks to answer the question of whether
	%		recall induces personal vote seeking behavior in a partisan setting,
	%		is more exploratory in nature.
	%		Given that the reason for pursuing personal vote is complex,
	%		it is difficult to say for certain which mechanism is at work
	%		and which is not.
	%		
	%		Nevertheless, I would still like to propose a sketch of a model
	%		that describes how recall induces personal vote.
	%		The model begins with a simple model where
	%		the implementation of party policy alone across multiple periods
	%		determines the electoral fate of the copartisan incumbents.
	%		I will demonstrates necessity of personal vote
	%		by illustrating that introducing recall in the simple model
	%		precludes the implementation of certain extreme policies.
	
	
\end{document}

