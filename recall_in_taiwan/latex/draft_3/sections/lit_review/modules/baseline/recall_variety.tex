% !TeX root = personal_vs_party.tex

\documentclass[hyphens, crop=false]{standalone}

\usepackage[
reflist=true,
%noreprint, 
natbib=true,
autocite=inline,
style=windycity]
{biblatex}


\usepackage[american]{babel}
\usepackage{csquotes}

\usepackage{url}


\usepackage[multiple]{footmisc}

\usepackage{setspace}
\doublespacing



\addbibresource{lit_review.bib}


\begin{document}
	
	
		Having laid out the context of recall in Taiwan,
		it may help to zoom out for a moment by noting that
		recall mechanisms around the world can be very dissimilar to each other,
		and as such should limit the ambition of any theory of recall.
		In fact,
		unless accounted for in full detail,
		variations in institutional arrnagement
		could very well prevent the discovery of any net incentive effect
		as parties learn to adapt and counteract the direct incentive effect of recall it might have.
		Such an endeavor has to await another time
		as the length constraint of this study does not permit it.
		Instead of searching for a theory that explains every kind of extant recall institution,
		I will attempt to use legislative recall in Taiwan as a baseline case that demonstrates
		that recall's incentive effect can hold incumbents more accountable
		by inducing them to cultivate a stronger \textit{personal vote}
		and then seek to explain why it may not be the case when
		certain components of recall's institutional arrangements change.
	
	
%	\newpage
%	\printbibliography
	
	
\end{document}

