% !TeX root = scope_stylized.tex

\documentclass[hyphens, crop=false]{standalone}

\usepackage[
reflist=true,
%noreprint, 
natbib=true,
autocite=inline,
style=windycity]
{biblatex}


\usepackage[american]{babel}
\usepackage{csquotes}

\usepackage{url}


\usepackage[multiple]{footmisc}

\usepackage{setspace}
\doublespacing



\addbibresource{lit_review.bib}


\begin{document}
	What further improves our chances of observing recall induce Taiwanese legislators
	to cultivate more personal vote and thus be more accountable to their consituents
	may be deduced from the institutional arrangements of recall in Taiwan
	To simplify analysis, a stylized institutional arrangement of recall in Taiwan is presented below.

	Any recall in Taiwan includes at most four events.
	\begin{enumerate}
		\item 
		Announcement of the intention to circulate recall petition;
		\item 
		(Dis)approval of the recall petition;
		\item 
		Holding the recall election;
		\item
		Holding the special election to fill the vacancy created by the recall.
	\end{enumerate}
	Not all events would take place in a recall attempt except the first two.
	Each intervening period between the consecutive events in this sequence
	lasts for a statutorily predetermined amount of time.
	In this stylized institutional arrnagment,
	recall cannot begin unless the organizers file their intention to circulate recall petition
	with the relevant electoral agency
	which grants ministerial approval of circulation as long as the application is in order.
	After the intention is filed and announced,
	unless
	the organizers successfully gather signatures in the incumbent's constituency
	numbering above a statutorily determined threshold on their recall petition
	within a statutorily determined amount of time,
	the recall petition will not be approved.
	If and when the petition is approved,
	a recall campaign period will follow before the recall election takes place.
	Should the recall be successful,
	a final special election campaign will be scheduled on a date after the recall election,
	allowing people to file candidacy and campaign for the vacated position.
	
	As the above description illustrates,
	the question put to the constituency is actually twofold.
	The explicit question that is always asked in a recall election is
	whether to retain the incumbent;
	whereas the implicit question that is not always asked at the polling station is
	who should fill the vacancy if the incumbent is recalled.
	Then putting the two questions to the voters at the same time,
	requires voters to obtain more information about other candidates
	in addition an answer to the simple ``how was my life under the incumbent" question
	if they are left to their own device.
	This creates an opportunity for politial entrepreneurship
	through which parties may coordinate on whom to support to
	fill the vacancy.
	Consider the gubernatorial California recalls
	
	
	
	
	
	
	As the above description illustrate,
	For the voters, the goal is less simple.
	The explicit question asked first is whether to retain the incumbent,
	the implicit question that is not always asked is who to elect if recall succeeds.
	If answering explicit question leads to answering the implicit question automatically,
	it requires voters to think ahead and creates a need for people to coordinate
	if recall decides who would replace them.
	California Gubernatorial recall asks voters who they want to elect should the recall succeeds
	and San Francisco recall of school board members invest mayor the power to appoint new members,
	essentially asking people to consider compare the utility of incumbents versus mayor choices.
	Personal vote seeking behavior would be increased
	if incumbents and voters focus on the incumbent's performance 
	and then coordinate for special election later.
	Taiwan's recall puts special election later,
	no need for political entrepreneurship, less need for party.
	
	As such, the recall institutional
	arrangement in Taiwan
	presents a chain of events in a sequential fashion
	without any overlap in time,
	which helps minimize the need for voters to think strategically ahead.
	As the above description of recall institution makes clear,
	recall actually poses two questions,
	the first one being whether to retain the incumbent in power,
	and the second one being who should fill the vacancy if the incumbent is recalled.
	While the first question may be simple enough,
	the second question less so.
	

	The benefits of doing so could be illustrated with the famous California gubenatorial recall of 2003.
	In the run-up to that recall election,
	Republicans and Democrats coordinated quickly by throwing their support
	respectively for Arnold Schwarzenneger and Cruz Bustamente,
	in large part because voters, in addition to the question of the recall,
	must also decide whom to fill the vacancy should the recall succeed,
	which might lead to a different outcome
	had the incumbent Gray Davis been the only name on the ballot.
	By contrast,
	this institutional arrangement makes the incentive to cultivate personal vote incentive front and center
	as the incumbent's job is the only thing on the line.
	
	Finally and perhaps most importantly,
	the insitutional arrangement presents a clear direct incentive effect
	that encourages politicians to
	pursue personal vote.
	While this is in many aspect is a fortuitous result of Taiwan's own
	institutional design that will hopefully provide baseline evidence that
	recall's net incentive effect does induce more
	activities aimed at cultivating a stronger personal vote,
	it can also serve a basis on which
	students of recall may be able to model
	other recall institutions that encourage personal vote to a lesser extent.
	
	

	
		
%	\newpage
%	\printbibliography
	
	
\end{document}

