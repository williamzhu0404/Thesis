% !TeX root = baseline_intro.tex

\documentclass[hyphens, crop=false]{standalone}

\usepackage[
reflist=true,
%noreprint, 
natbib=true,
autocite=inline,
style=windycity]
{biblatex}


\usepackage[american]{babel}
\usepackage{csquotes}

\usepackage{url}


\usepackage[multiple]{footmisc}

\usepackage{setspace}
\doublespacing



\addbibresource{lit_review.bib}


\begin{document}
	
		To defend the view that recall does offer the potential of holding
		politicians accountable to their constituents in a partisan electoral environment,
		I propose to begin with a baseline case
		where such effect is not only credibly attributable to recall
		but also more likely to be measurable,
		and then work toward explaining the harder cases.
		Taiwan offers a perfect baseline case to begin this line of inquiry.
		As the threat of recall in Taiwan is already established,
		I shall demonstrate here that
		there is a good chance of measuring the effect of recall
		on the accountability of incumbents
		by studying
		legislative recall in Taiwan.
		Since the island's democratization,
		Taiwanese legislators not only engage in a high volume of activities
		to cultivate personal vote,
		but also change their personal vote seeking behavior
		in response to institutional reform.
		Furthermore,
		the institutional arrangement of legislative recall
		encourages personal vote seeking behavior
		while limiting the need for partisan coordination,
		thus providing one of the best opportunities to observe recall
		holding incumbents more accountable to their constituencies.
	
	
%		The conflict between the pursuit of personal vote and party vote
%		constitutes the major difficulty underlying the key motivation of this study
%		- to determine recall's net incentive effect on incumbents in a partisan electoral environment.
%		Incumbents may well want to pursue personal vote to their hearts' content,
%		but they may not do so when their party
%		may prevent the full extent of their pursuit of personal vote since
%		the goal of implementing policy and thus their ultimate goal of winning reelection
%		requires each member to contribute to the public good their party supplies
%		by submitting to the party discipline.
%		Consequently, one may very well fail to observe any incentive effect of institutions
%		when parties counteracts it by incentivizing members to behave otherwise.
%		Though accountability may still be achieved indirectly through party,
%		it is by no means guaranteed considering that
%		partisan vote rests in large part on voters' partisan identification
%		which is remarkably stable against the performance of the incumbent or the incumbent's party
%		resulting in no observation of any net incentive effect.
%		
%		A way to counteract this difficulty resulting from theory and
%		the attending measurement issues is to
%		find a recall mechanism that encourages pursuit of personal vote and discourages that of party vote to a great extent.
%		Such an approach stands the best chance
%		of measuring signficiant net incentive effect that
%		reveals the existence of recall's direct incentive effect as the baseline
%		and potentially allow one to work out exactly what is counteracting this effect if the effect exists.
%		This is where Taiwan's legislative recall comes in as one of the best possible baseline case.
%		I will demonstrate that
%		in addition to its credible threat already seen in the above description of recall's context and practice in Taiwan,
%		recall elections,
%		with a
%		design that all but maximizes incentives for cultivating a personal vote,
%		operating in a candidate-centered electoral system which
%		features a stable party system,
%		provides some of the best baseline case.


%		Measuring the direct incentive effect of recall requires the incumbents,
%		it is imperative that legislators not only pursue personal vote
%		but also remain sensitive to insitutitonal changes such that
%		their personal-vote seking behavior adjusts accordingly.
%		Such concern is especially prevalent in studies of institutional reforms
%		where a frequently leveled criticism is that
%		institutional inertia may prevent one from efficiently measuring any behavioral changes
%		due to institutional reforms until years later or at all
%		especially when parties discourage such behavioral changes.



%	\newpage
%	\printbibliography
	
	
\end{document}

