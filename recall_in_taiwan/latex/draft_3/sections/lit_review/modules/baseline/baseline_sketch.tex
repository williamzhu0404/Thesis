% !TeX root = personal_vs_party.tex

\documentclass[hyphens, crop=false]{standalone}

\usepackage[
reflist=true,
%noreprint, 
natbib=true,
autocite=inline,
style=windycity]
{biblatex}


\usepackage[american]{babel}
\usepackage{csquotes}

\usepackage{url}


\usepackage[multiple]{footmisc}

\usepackage{setspace}
\doublespacing



\addbibresource{lit_review.bib}


\begin{document}
	Recall gives local constituency the right to recall the incumbent.
	Hence,
	recall is supposed to improve dyadic accountability, period.
	
	In general, electoral incentives for seeking personal vote improves dyadic accountability
	and given that recall is just another election the incumbent needs to survive,
	holding or threatening to hold a recall election to remove the incumbent should improve dyadic accountability
	but partisan forces do not always encourage things to go in this direction.
	
	I believe that recall does enhance dyadic accountability achieved by shaping incentive structures
	that encourage personal vote should be true.
	To make sure that this theory
	we want personal vote seeking behavior and incentives to vary and likely feature in great quantity
	we want personal vote seeking behavior to respond to institutional changes
	we want party vote seeking behavior to decrease as much as possible or increase as little as possible
	we want recall institution to be effective insofar as it generates credible threat of triggering it
	so that that changes in personal vote seeking behavior can be measured and
	we can be sure that it is responsive to institutional change
	
	Taiwan has always featured a lot of personal vote seeking behavior.
	It varies a lot from one
	Taiwan's transition from SNTV to FPTP
	Lots of personal vote seeking behavior to decrease personal vote seeking behavior
	Responsiveness of pvsb to past institutional reform gives us reason to believe that the same may happen again.
	
	taiwan's recall institution itself also encourages personal vote seeking behavior
	Yes recall is a referendum on the incumbent and the survival of incumbent hinges greatly on how
	the incumbent's personal vote is strong
	
	For the incumbent, the goal of incumbent is simple, survive recall.
	This is espeically true in Taiwan, being recalled means disqualification from standing for the position again for five years

	For the voters, the goal is less simple.
	The explicit question asked first is whether to retain the incumbent,
	the implicit question that is not always asked is who to elect if recall succeeds.
	If answering explicit question leads to answering the implicit question automatically,
	it requires voters to think ahead and creates a need for people to coordinate
	if recall decides who would replace them.
	California Gubernatorial recall asks voters who they want to elect should the recall succeeds
	and San Francisco recall of school board members invest mayor the power to appoint new members,
	essentially asking people to consider compare the utility of incumbents versus mayor choices.
	Personal vote seeking behavior would be increased
	if incumbents and voters focus on the incumbent's performance 
	and then coordinate for special election later.
	Taiwan's recall puts special election later,
	no need for political entrepreneurship, less need for party.
	
	
	signature gathering period lasts for a definite time
	what does it help
	to help reduce thrate of recall
	to focus the campaign in 
	
	let's see there is also the credibility of recall threat
	
	Baseline evidence for recall's improvement of dyadic representation/accountability
	
	Credible threat of recall
	personal vote responsive to the credible threat
	response measurable
	demand for party vote increase less or decrease
	
	Party and incumbents agree that incumbents need more personal vote
	Party implement policies in order that its incumbents stay reelected
	Party position is generally different incumbent's policy position
	Paty delivers policy by giving incumbents side payment
	It also delivers policy by implementing extreme policies early and moderate policies late
	Doing so helps improve incumbents in marginal constituency
	Recall means that you cannot hide extreme policies at the beginning of a term,
	doing so would cause legislators to be subject to recall
	Then 
	
	
	
	
	
	
	Let's organize the thought
	\begin{enumerate}
		\item
		Incentivizing incumbents to cultivate a personal vote generally improves dyadic accountability
		\item
		FPTP elections generally incentivize personal vote
		\item
		Recall is essentially an FPTP election for the incumbent, should incentivize personal vote
		\item
		This theory is not yet borne out, looking for some minimal evidence that this theory is correct
		\item
		Baseline
		\begin{enumerate}
			\item
			Credible threat of recall, actively working
			\item
			threat of recall causes personal vote seeking behavior to increase
			\item
			personal vote seeking behavior exists with great range and variability
			\item
			party does not disincentivize personal vote to a great extent
		\end{enumerate}
		\item
		We further back this up with a theory that party and incumbents agree that incumbents need more personal vote
		\begin{enumerate}
			\item
			Voters have recency bias, attach more importance to later events
			\item
			Party passes policy to further its incumbents' reelection
			\item
			Party implements extreme policy earlier, pass moderate policy later
			\item
			Recall prevents doing so by allowing voters to remove incumbents early on
			\item
			Party can give up extreme policies
			\item
			Party can allow incumbents to distance themselves from extreme polciies
			\item
			Party can allow incumbents to devote more resources toward constituency service
		\end{enumerate}
		\item
		\begin{enumerate}
			\item
			Agreement: recall voter largely partisan, party matters for both
			\item
			Disagreement: unpopularity signal, can party agree to back it up
			\item
			Disagreement: resource depletion, can party agree to preemptively help incumbents.
			\item
			
			\item
			
		\end{enumerate}
		\item
		Research Design
		\begin{enumerate}
			\item
			mixed-member electoral system/did
			\item
			roll call vote
			\item
			cosponsorship
			\item
			identifying assumption
		\end{enumerate}
		\item
		
		\item
		
		\item
		
		\item
		
		\item
		
		\item
		
		\item
		
	\end{enumerate}
	
	
	
	
	
	
	
	
%	\newpage
%	\printbibliography
	
	
\end{document}

