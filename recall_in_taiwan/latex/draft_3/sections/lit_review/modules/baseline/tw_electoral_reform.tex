% !TeX root = personal_vs_party.tex

\documentclass[hyphens, crop=false]{standalone}

\usepackage[
reflist=true,
%noreprint, 
natbib=true,
autocite=inline,
style=windycity]
{biblatex}


\usepackage[american]{babel}
\usepackage{csquotes}

\usepackage{url}


\usepackage[multiple]{footmisc}

\usepackage{setspace}
\doublespacing



\addbibresource{lit_review.bib}


\begin{document}
	
		Cultivating a personal vote has always been a priority for legislators,
		even before voters got a chance to pick them in free and fair elections.
		Under the one-party rule of Kuomintang,
		both politicians within and without the ruling party,
		have to skirt around the ban on forming political parties
		by establishing their own personal brands and/or factions.
		When free and fair elections did arrive,
		the single nontransferable vote (SNTV) electoral system was adopted.
		Under this sytem,
		virtually all constituencies are multi-member districts (MMD)
		where each constituent gets only one vote to choose one candidate,
		and a constituency of size $n$
		would be represented by the $n$ candidates receiving the most votes
		in the Legislative Yuan.
		This electoral system promotes personal vote to a far greater extent than
		other candidate-centered electoral system like FPTP,
		yet simultaneously undermines the legislators' accountability to
		the local consitituencies who are represented by multiple legislators,
		making it rather difficult to effectively reward and punish incumbents
		who can survive
		at the poll
		with varying level of personal vote.
		
		The 2008 electoral reform
		saw the Legislative Yuan eliminate all MMDs and replaced SNTV with the FPTP system
		for the legislative elections,
		which incentivizes cultivating a personal vote
		to a lesser extent than before.
		Though many conventional methods of cultivating a personal vote,
		including constituency service,
		remain alive and well,
		the electoral reform saw a dramatic decline in 
		dissenting votes against their party leadership on the legislative floor.
		These behavioral patterns serve underline
		the high volume and pliability of personal vote seeking behavior in the legislative arena
		in response to institutional constraints
		in the past and hopefully in the case of recall reforms as well.
		
%	\newpage
%	\printbibliography
	
	
\end{document}

