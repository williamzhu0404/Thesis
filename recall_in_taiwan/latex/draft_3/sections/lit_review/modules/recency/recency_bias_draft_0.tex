% !TeX root = recency_bias.tex

\documentclass[hyphens, crop=false]{standalone}

\usepackage[
reflist=true,
%noreprint, 
natbib=true,
autocite=inline,
style=windycity]
{biblatex}


\usepackage[american]{babel}
\usepackage{csquotes}

\usepackage{url}


\usepackage[multiple]{footmisc}

\usepackage{setspace}
\doublespacing



\addbibresource{lit_review.bib}


\begin{document}
%	[Note to class: 
%	This section is finished only recently to incorporate insight from a literature the review of
%	I have yet to complete.
%	There is another section that seeks to justify the stylized facts you shall see below.
%	In order to present a concise outline of my theory,
%	I have omitted the justificatiions and asked you to take them for granted.
%	I would appreciate any feedback on the adequacy of the theory itself
%	especially regarding its logical coherence
%	and ability to yield the hypotheses
%	outlined in this section.
%	As a result, some of the assertions in this sample lacks citation which will be
%	provided in due course.
%	]
	
	
	
	
	
	
	In summary,
	the previous section has established the following
	three stylized facts:
	\begin{enumerate}
		\item
		An incumbent receives
		two distinct kinds of support,
		one based on the personal reputation, which is called \textit{personal vote}, and
		the other based on the incumbent's party's reputation, which is called \textit{party vote}.
		\item
		For any incumbent, there is a tradeoff between pursuing personal vote and party vote.
		\item
		An incumbent cultivates a personal vote
		by providing more services and porks to the constituency and
		representing the aggregate preference of a constituency.
	\end{enumerate}
	which implies that
	institutional arrangements that induce an incumbent to pursue more personal vote than party vote
	\textit{generally} improves the incumbent's accountability to the constituency.
	As the previous section's argument goes,
	when parties are taken out of the equation
	legislative recall in Taiwan seems to do exactly that
	by putting the name of the incumbent alone on the ballot for an up or down vote.
	These conclusions provide a nice segue to the next question:
	would the same prediction hold when parties are put back into the equation again?
	
	To this question,
	my answer is a qualified yes.
	Just like electoral systems,
	institutional arrangements of recall are so diverse that it is difficult to
	argue that parties would always allow incumbents to seek more personal vote
	in response to the threat of recall elections.
	Fortunately,
	the mechanism sketched here,
	though specific to the legislative recall
	from which the evidence is drawn,
	is general enough
	that it should be applicable to all recall institutions around the world
	upon minor modification to account for specific details of institutional arrangements of recall.
	
	Simply put,
	when recall is not feasible,
%	recall undermines the government party's ability to manipulate
%	the policymaking process in the legislature
%	through a mechanism that is also responsible for political business cycles.
%	When recall is unavailable or infeasible due to certain institutional arrangement,
	the government party,
	tentatively defined as the party who controls the executive branch,
%	to which the legislature submit bills for approval,
	anticipating voters 
	implements its more extreme policies early in the term,
	since it anticipates voters to reward and punish its members
	to reward and punish its members
	based on its more recent performance,
	hoping to win back voters' support with porks and other moderate policies
	toward the end of the term,
	thus creating an electoral cycle of policymaking.
	
	However,
	when recall becomes feasible,
	especially in marginal constituencies represented by members of the government party,
	the electoral cycle policymaking may take on a different shape.
	Voters, seeking to prevent the government party
	from implementing extreme policies,
	threaten to punish the government party
	by recalling its members in the legislature
	at the earliest juncture.
	The credible threat is real in the marginal constituencies,
	and the more seats it may lose,
	the more the government party has to compromise
	to present a less extreme bills
	which defeats the purpose of imposing
	discipline on its members to pass the government bill in the first place.
	Therefore, the government party
	will choose to implement policies that are less extreme,
	instead of being forced to do so as a result of loss its members
	through recall.
	The government party will also allow its members in the legislature,
	especially those representing marginal constituencies,
	to cultivate a stronger personal vote
	to prevent recall elections from happening at all
	not only to ensure its majority throughout the legislative term
	which is essential to its ability to implement policies.
	
	
	In addition,
	I further predict that the above predicted effect
	will be stronger when the probability of recall is stronger.
	It makes intuitive sense that at the beginning of the term,
	a legislator may be given more benefit of doubt
	and thus may be allowed to deviate more from constituency preference.
	Similarly, recall also makes less sense when the election date
	is approaching and incumbents can be simply punished at the general election.
	Therefore, recall should also generate its own
	electoral cycle of policymaking and personal vote activities.
	One should expect that more extreme policies will still be introduced toward the beginning of the term
	but less so than before.
	More importantly, one should expect more moderate policies
	and more activities aimed at cultivating a stronger personal vote
	around the middle of the term than before,
	which is not predicted to happen if recall is not feasible
	but very likely to happen when recall is feasible.
	
%	In addition, to the above prediction about the government party
%	and its legislative members
%	across the legislative term,
%	the same prediction would also modify the electoral cycle of policymaking
%	when the probability of recall varies throughout the term.
%	This has two possible outcomes
%	one outcome is that 
%	A party would then seek to introduce policies
%	that are less extreme
%	The effect will be more acute in the case
%	where probability of recalling a government party
%	changes throughout the legislative term.
%	This means the higher the probability of recall is,
%	the government party to
%	dole out more pork and moderate policies 
	
	
	
	
	
	Consequently,
	this theory predicts that recall's effect on the government party is contingent
	on the recall's threat to its majority.
	The more likely a recall reduces a government party's
	majority in the legislature,
	the more likely recall causes the said government party to
	\textit{ceteris paribus}
	\begin{enumerate}
		\item
		introduce policies that are less extreme 
		throughout the legislative term;
		\item
		introduce policies that provide more pork
		for marginal constituencies,
		especially those held by the government parties.
	\end{enumerate}
	The theory also predicts that 
	the less their margins of victory were in the last election,
	the more likely
	legislators,
	especially those belonging to the government party,
	may be recalled,
	causing them, 
	to
	\begin{enumerate}
		\item
		take a policy position that is further from its party's position;
		\item
		devote more time and resource to constituency service.
	\end{enumerate}
	The each of the above prediction is also,
	per the discussion in the previous paragraph
	more likely to happen in the middle of the legislative term
	than the rest of the legislative term.
	
	
	that belong to the government party will
	Note that the
	``period of the term where recall is more feasible"
	is intentionally unspecified.
	The phrasing implies an understanding of an underlying
	probability distribution over the entire legislative term
	as delineating the period requires not only
	consideration of statutory rules of when recall is possible
	The more marginal seats
	a government party holds,
	the more likely it would seats
	should it attempt to pass extreme policies.
	The threat's credibility increases in the number of marginal constituencies,
	the government party stands to lose in a recall,
	thus hampering its goal of passing policies and ensuring its members' reeelction.
	Ultimately,
	the government party passes less extreme policies,
	provides more pork,
	and encourages its members to cultivate a stronger personal vote,
	and recall will not happen save at the disequlibrium where the government party
	misjudges the preferences of the marginal consituencies.
	Consequently, the electorl cycle of legislative policymarking is smoothened,
	if not completely eliminated.
	
	\subsection*{Hypotheses on Recall's Incentive Effect in Taiwan}
	
	The national legislature of Taiwan, which is called Legislative Yuan,
	provides a good testing ground for the theory.
	Each legislative term lasts for four full years,
	which should have provided ample opportunities for
	the government party to manipulate the timing of bill introduction
	when recall is not feasible,
	giving us an opportunity to see how recall disrupts this sort of strategic timing.
	More importantly,
	since it consists of two tiers of legislators
	who operate under distinct incentive structures
	that are largely independent from each other.
	There is a nominal tier of legislators
	who are elected from single-member districts (SMDs)
	and take up about two-thirds of the seats in the Legislative Yuan;
	then there is also a party tier of legislators
	who are elected from a national at-large district
	based on the closed party list voters pick in addition to the
	candidates they choose to represent their SMDs
	and take up about one-third of the seats.
	Though technically speaking,
	recall was and is always legally possible for nominal-tier legislators
	and impossible for party-tier legislators,
	the statutory threshold set for recall was so demanding that
	recall was effectively impossible for all legislators until the reform
	that made it possible \textit{only} for nominal-tier legislators,
	thus creating a natural experiment where
	party-tier legislators are the control group
	and
	nominal-tier legislators the treatment group.
	
	Furthermore,
	one provision in Taiwan's recall institutions
	makes it easier to test the hypothesis that recall institution generates its own electoral cycle.
	In Taiwan,
	it is impossible to recall an incumbent until one year after the incumbent assumes office.
	While this stipulation may be justified on the normative ground
	that it prevents frvivolous recall that does nothing other than seeking a rerun of the last election,
	it may also be seen as an attempt to continue to the practice of hiding bad policies at the beginning of the term.
	Nevertheless,
	it creates a clear-cut point in time
	after which the electoral cycle generated by recall should be set in motion,
	allowing for the testing of t
	The above institutional arrangements of legislative recall in Taiwan
	lead to the following
	hypotheses about the government party's behavior in Legislative Yuan.
	\begin{enumerate}
		\item
		When recall becomes feasible, the government bills, i.e. bills introduced by the government party, will become less extreme.
		\item
		When recall becomes feasible, the frequency of government bill introduction initially drops
		after one year into the legislative term
		but eventually increases when the end of the term approaches.
	\end{enumerate}
	
	As the government party needs its members to hold on to their seats,
	especially those who represent marginal constituencies where they are elected on a thin margin (hence the term marginal constituency),
	I make the following hypotheses about the individual members of the legislative party.
	\begin{enumerate}
		\item
		When recall becomes feasible, the less the margin
	\end{enumerate}
	
	
	
	
	Hypothesis 1:
	Recall  induces the government bills, i.e. bills introduced by the government, to become less extreme.
	
	Hypothesis 2:
	Party in government avoids introducing policies during the period where
	recall is possible.
	
	Hypothesis 3:
	
	Party in government allows incumbents to take distinct positions from the party
	more often, especially those representing marginal constituencies.
	
%	Prediciton 4:
%	
%	
%	Party in government gives incumbents more time and resources to perform constituency service
%	
%	Hypothesis 5:
%	
%	Party in government doles out more pork, especially to marginal constituents.
%	
%	Hypothesis 6:
%	
%	Party in government permits more position taking against the leadership on the legislative floor.
%	
%%	Policy, Position Taking: dissent from party, Credit Claiming: pork and constituency service
	
asdfasdfasdfdsfaddfd	
%	This mechanism
%	ressembles the familiar tale of the political business cycle that is first formalized by Nordhaus.
%	The model begins its life as a theory that connects business cycles of the economy
%	with electoral calendars in democracies.
%	Given the mixed evidence for a political business cycle in its original conception,
%	the literature expands its purview by examining
%	electoral cycles of fiscal policies and budget spendings.
%	Decades of 
%	
%	
%	
%	The mixed evidence for actual business cycles causes people to
%	examine the cyclical implementation of fiscal policies.
%	
%	Two offshoots of political business cycle: electoral cycle of policymaking and electoral calendar's effect on political business cycle.
%	Political business cycle vanishes when elections are too frequent or elections never occur at all.
%	Hide extreme policies toward the beginning of the term,
%	court voters with favorable policies toward the end of the term.
%	
%	There is no general theory of electoral cycle of policymaking
%	outside of policies that are not economically significant.
%	Difficult to measure extreme versus moderate across jurisdictions
%	unless the policy has a clear economic dimension.
%	
%	
%	
%	

%	This classical political business cycle begins with the assumption that
%	\begin{enumerate}
%		\item
%		Election dates are exogenously determined by statutes.
%		\item
%		Incumbents only seek reelection
%		\item
%		Party seeks to help its members win reelection.
%		\item
%		Constituents vote based on economic conditions,
%		reward party in government in good times and punish it in bad times.
%		\item
%		Constituents attach greater importance to 
%	\end{enumerate}
%	voters 
%	
%	
%	Voters are retrospectively assess the performance of the party in government
%	and then decide whether to vote for its members.
%	
%	
%	
%	
%	
%	To illustrate why the party would agree that,
%	in response to recall,
%	individual incumbents need to cultivate a stronger vote,
%	I posit that recall reduces the party's ability
%	to manipulate voters' perception of its performance
%	which depends on a well-understood effect in political pyschology:
%	recency bias.
%	To put it simply,
%	people,
%	\textit{ceteris paribus},
%	attach greater significance to events that occur more recently.
%	Incumbents
%	that want to implement more radical policies
 %	may then
%	seek to take advantage of this bias by
%	introducing them
%	at the beginning of their term
%	and then seek to ``make amends" for the radical policies
%	by doling out more moderate policies and porks
%	toward the end of the term,
%	thereby securing reelection even after passing policy that deviates
%	from the aggregate preference of the constituency.
%	
%	This phenomenon has an eminent tradition that
%	and was first formalized by Nordhaus
%	who famously argued that governments manipulate inflation rate to reduce unemployment
%	temporarily
%	At the same time,
%	evidence supporting existence of political business cycle
%	has been rather tenuous
%	despite the persistent interest in this phenomenon
%	the content of which has also evolved in large part due to
%	the available evidence that supports a cyclical movement in certain metrics.
%	This article contributes to this eminent literature
%	by positing that recall has,
%	like many other institution that promotes transparency,
%	limited the ability of incumbents to exploit the recency bias
%	that off
%	
%	The tradition of political business cycle
%	is to focus on the interface between politics and economics
%	A new strand of research
%	has now begun to explore the cyclical nature
%	of legislative policymaking process as well.
%	And 
%	
%	
%	
%	
%	
%	The position of this paper in the tradition of the political business cycle
%	is that it explores the institutional constraints
%	that inhibits political business cycle
%	through adjustment of the electoral calendar where
%	the threat of election becomes a perpetual
%	instead of a periodic threat such that
%	extreme policies are no possible.
%	While there has been a lot of study about how endogenous election timing
%	could be related to political business cycle,
%	no research has focused on
%	how the threat of recall,
%	which usually exist for terms with exogenously
%	determined length,
%	may disincentivize cyclical patterns of political behavior
%	around the political calendar.
%	
%	One possibility is that recall introduces constant level of threat
%	throughout the term
%	such that political business cycle is completely erased.
%	
%	Another possibility is that recall
%	is tantamount to holding a midterm election
%	for all incumbents.
%	only introduces certain degree of threat
%	
%	The more likely scenario would be situated somewhere
%	between constant level of threat at all times 
%	and merely effectively halving the incumbent's term.
%	
%	
%	Voters attach greater importance to recent events.
%	
%	Without recall, voters pay attention to policies and events in the run-up to next election.
%	
%	Voters punish incumbents who underperform in the run-up to the next election.
%	
%	Incumbents seek to avoid punishment,
%	implement policies preferred by the voters toward the end of the term.
%	
%	Incumbents who have to pass unpopular policies.
%	implement policies toward the beginning of the term.
%	Sometimes, it can be justified by blaming the predecessors.
%	
%	With recall institutiion in place,
%	voters may seek to punish the incumbents immediately.
%	
%	Can voters predict that incumbents will be incompetent
%	without signs of incompetence?
%	No.
%	Do voters punish incumbents through recall to prevent future damage?
%	Do voters punish incumbents through recall to undo recent damage?
%	
%	If voters punish transgressions that occurred earlier in term
%	with great accuracy and severity,
%	recall will not be necessary, election can hold 
%	
%	
%	size of majority 
%	
%	size of margin
%	
%	ideological position of policy
%	
%	ideological distance between parties
%	
%	ideological distance between incumbent and consituency
%	
%	
%	
%	
%	
%	
%	
%	
%	
%	
%	
%	
%	Theoretical Implication is that
%	party will yield to recall pressure by introducing less extreme policies
%	thereby flattening the cycle so to speak
%	
%	
	
	
	
	
	
	
	
	
	
	
	
	
	
	
	

		
%	\newpage
%	\printbibliography
	
	
\end{document}

