% !TeX root = policy_sequencing_informal.tex

\documentclass[hyphens, crop=false]{standalone}

\usepackage[
reflist=true,
%noreprint, 
natbib=true,
autocite=inline,
style=windycity]
{biblatex}


\usepackage[american]{babel}
\usepackage{csquotes}

\usepackage{url}


\usepackage[multiple]{footmisc}

\usepackage{setspace}
\doublespacing



\addbibresource{lit_review.bib}


\begin{document}
	
	In a nutshell,
	the main theory is built on a well-established
	psychological phenomenon called recency bias
	which leads individuals to attach more importance to more recent events.%
	\footnote{While there exist both recency bias and primacy bias, which means individuals attach more importance to early events, recency bias is generally stronger.}
%	Summarize recency bias evidence
		
	Building on the assumption that recency bias is operative,
	economists and political scientist have built a large literature of political business cycle/policy cycle
	where periodic implementation of policies around election days is predicted.
%	Summarize political business cycle evidence

	A recency factor is created to encapsulate this recency bias.
	

	
	
	
	
	
	
	
	
	
	
	
	
	
	
	
	
	
	
	
	
	
	

		
%	\newpage
%	\printbibliography
	
	
\end{document}

