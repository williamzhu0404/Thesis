% !TeX root = recency_bias_1.tex

\documentclass[hyphens, crop=false]{standalone}

\usepackage[
reflist=true,
%noreprint, 
natbib=true,
autocite=inline,
style=windycity]
{biblatex}


\usepackage[american]{babel}
\usepackage{csquotes}

\usepackage{url}


\usepackage[multiple]{footmisc}

\usepackage{setspace}
\doublespacing



\addbibresource{lit_review.bib}


\begin{document}
%	[Note to class: 
%	This section is finished only recently to incorporate insight from a literature the review of
%	I have yet to complete.
%	There is another section that seeks to justify the stylized facts you shall see below.
%	In order to present a concise outline of my theory,
%	I have omitted the justificatiions and asked you to take them for granted.
%	I would appreciate any feedback on the adequacy of the theory itself
%	especially regarding its logical coherence
%	and ability to yield the hypotheses
%	outlined in this section.
%	As a result, some of the assertions in this sample lacks citation which will be
%	provided in due course.
%	]
	
	
	
	
	
	
%	In summary,
	To establish a theory of recall,
	it halpes to begins with
%	the previous section has established the following
	three stylized facts:
	\begin{enumerate}
		\item
		An incumbent receives
		two distinct kinds of support,
		one based on the personal reputation, which is called \textit{personal vote}, and
		the other based on the incumbent's party's reputation, which is called \textit{party vote}.
		\item
		For any incumbent, there is a tradeoff between pursuing personal vote and party vote.
		\item
		An incumbent cultivates a personal vote
		by providing more services and porks to the constituency and
		representing the aggregate preference of a constituency.
	\end{enumerate}
	which implies that
	institutional arrangements that induce an incumbent to pursue more personal vote than party vote
	\textit{generally} improves the incumbent's accountability to the constituency.
	As the previous section's argument goes,
	holding more frequent elections under candidate-based electoral system,
	such as FPTP,
	incentivizes incumbents to pursue a stronger personal vote.
	Legislative recall in Taiwan,
	by putting the name of the incumbent alone on the ballot for an up or down vote,
	effectively constitutes yet another FPTP election for incumbents.
	Then for the incumbent,
	the lesser the margin of victory is in the last election,
	the more likely the incumbent may be recalled in a hypothetical recall.
	This implies that people who are more likely to be recalled,
	cannot rely on the party vote alone and are thus
	more likely to need to cultivate a stronger personal vote;
	and when recall becomes feasible,
	they have an even stronger reason to do so.
	This raises an important question:
	will the party stand for it?
	
	To this question,
	my answer is a qualified yes.
%%	Though
%%	institutional arrangements of recall around the world
%%	are so diverse that
%%	it is difficult to
%%	argue that parties would always allow incumbents to seek more personal vote
%%	in response to the threat of recall elections.
%%	Fortunately,
%%	the mechanism sketched here,
%%	though specific to the legislative recall
%%	from which the evidence is drawn,
%%	is general enough
%%	that it should be applicable to all recall institutions around the world
%%	upon minor modification to account for specific details of institutional arrangements of recall.
%	Though the theory I will be proposing
%	may be heavily dependent
%	on the theoretical assumptions
%	and the particularities of the political landscape
%	of the baseline case of Taiwan,
%	but it serves to illustrate that recall
%	\textit{can} hold the incumbents possible more accountable to their constituents,
%	at least in the baseline case,
%	leaving a more detailed investigation of the mechanism for another time.
	First,
	I shall assume that personal vote has no role in elections
	and elections is simply a result of the sampling of the constituents'
	partisan identification and policy evaluation
	and the sampling error accounts for the election results.
	Accordingly, each party becomes a unitary actor
	and its members but interchangeable cyphers mindlessly executing the party's instruction.
	Second,
	I shall illustrate in the next subsection that before recall is feasible,
	policymaking making process in Taiwan
	follows the same pattern and mechanism of political business cycle that allow
	the legislative members of the government party,
	which is
	defined here as
	the party that controls the executive branch of the government,
	to avoid electoral punishment for passing extreme policies toward the beginning of their term.
	Third,
	I shall demonstrate how
	the recall changes the calculus of the government party,
	whose members can now be punished early on for such opportunistic timing of policy implementation,
	causing the government party to introduce less extreme policies,
	Finally,
	I shall put the personal vote back into the equation
	to show that giving the incumbents more personal vote,
	which enhances their accountability to their local constituencies,
	actually helps the party to implement their policies,
	thus completing the theory.
	
	
%	2.1 Here begins the literature review of political business cycle
	
	Political business cycle
	was intended as a way to explain economic conditions
	with political institutions,
	more specifically the exogenously fixed election dates.
	There are several assumptions underlying this mechanism:
	\begin{enumerate}
		\item
		Voters make their choice at the poll based on the macroeconomic conditions
		which signals government competence.
		More specifically, they do so by voting for government party candidates
		when unemployment rate is low
		and voting for opposition party candidates
		when unemployment rate is high.
		\item
		Voters have rather limited memory,
		consequently they attach greater importance to the macroeconomic condition
		when the election day is drawing closer.
		\item
		Incumbents seek reelection.
		
		\item
		Incumbents can manipulate the economy by increasing the inflation rate
		that lowers the unemployment rate in the short run as predicted by the Philips curve.
		\item
		At the general election,
		voters decide the composition of both the executive and legislative branch of the government.
		\item
		The dates of general election are exogenously set by the statute
		and cannot be altered.
	\end{enumerate}
	It follows that,
	come next election day,
	voters' memory of the macroeconomic conditions right after the last election
	is much weaker than that in the run up to the next election day.
	Consequently,
	voters will reward the government party for low unemployment rate
	before the election day
	even if the unemployment rate is high after the last election.
	Then, unable to move the date of general election,
	the government party
	realizes that it can win the next election by
	raising the inflation rate to temporarily
	lower the unemployment rate before the election day
	without losing the election after the next
	for rising unemployment right after the election day.
	Hence, the political business cycle becomes the equilibrium result,
	as long as the assumption holds.
	
	Much as
	it offers a simple and compelling mechanism,
	political business cycle in its original conception
	has received its fair share of criticism.
%	from both theoretical and empirical perspective.
	Some point out the flaws in the behavioral assumptions of voters and incumbents
	which may prevent the emergence of an electoral cyclical pattern in macroeconomic conditions;
	Others question whether incumbents
	are capable effecting changes in macroeconomic conditions
	that make such manipulation possible.
	Still others investigate how changing the institutional arrangements
	stipulated in the assumptions may modify the cyclical pattern.
	These lines of inquiry
	have contributed to the development of a vast literature
	where a more nuanced
	sketch of a conditional electoral cycle
	of both policy instruments
	and economic conditions
	has emerged.
	Given that not all the studies in this literature
	do not all concern cyclical patterns of macroeconomic indicators,
	it would be inappropriate to consider all of them
	part of the political business cycle literature.
	However, for lack of a better term,
	this literature will hereinafter be referred to as PBC literature
	where PBC, though usually an abbreviation of political business cycle
	or political budget cycle,
	is meaningless here except as a label.
%	for all studies
%	that concern cyclical pattern
%	of both policy instruments
%	and economic conditions
%	induced by political institutions.
	
	
	
	
	
	
	In
%	this vast
	the PBC
	literature
%	of political business cycle
%	which attempt to answer follow these lines of inquiry
	two developments are especially noteworthy for this study.
	One offshoot of
%	political business cycle
	this
	literature
	trains its aim on
	political legislation cycle,
	which refers to
	the cylical pattern of legislative process in general
	instead of focusing on one particular type of policy instruments
	like fiscal policies, bugeting among other policies.
	Most work along this line of inquiry focuses on
	the volume of legislative activity at different points in time
	throughout the legislative term.
	Another branch of the PBC literature investigates how
	changes to the timing of elections
	yields changes to the cyclical pattern of the policy or economic outcomes.
	This
	study of recall contributes to the these two branches of the
	PBC
	litertature
	by sketching a theory illustrating that,
	policies implemented by the government party in the legislature
	tend to be more extreme after the election days
	and less extrene before the election days
	under the typical assumptions that general elections are held on exogenously fixed dates
	(usually every four years)
	and that
	by threatening to trigger a legislative recall,
	which is effectively just another election for the incumbent legislator,
	in the middle of the legislative term,
	voters can incentivize government party legislators
	to cultivate a stronger personal vote
	\textit{throughout the legislative term}
	than before
	where more effort is exerted when election day draws near.
	
%	2.2 Political Legislation Cycle Reworked
	
	To begin with,
	I first sketch a theory of political legislation cycle
	when recall is not feasible.
	This theory
	preserves
	most of the fundamental ingredients would stay in place.
	That incumbent legislators single-mindedly seek reelection
	has been a longstanding assumption in political science
	and will continued to be the case here.
	Voters also continue to
	retain a recency bias whereby they
	discount the utility of incumbents' past policies
	attach greater importance to events
	closer to the present
	is also well-established.
	These assumptions are central to the
	prediction that voters discount 
	past job performance
	such that doing a good job when the election is coming
	would \textit{more than} compensate
	a mistake committed in the more distant past.
	
	
	
	However,
	this theory will make one crucial change
	to the conventional assumptions made
	by many studies in the PBC literature.
	I propose a different assumption about voter motivation
	which stipulates
	voters electorally reward the government 
	for a high degree of policy congruence,
	and punish the government
	for a low degree of policy congruence.
	Strangely enough,
	the issue of policy congreuence
	has largely been sidestepped
	in PBC literature,
	even though it is
	central to the
	spatial theory of voting.
	Many studies
	in the PBC literature
	avoid discussing policy congreuence
	by focusing on one type of policy instrument
	for which voter preference is relatively straightforward.
	Another way to avoid doing so
	is to assume that high volume of policymaking activities
	by itself can improve the government party's chances of
	winning reelection.
	Such assumptions could very well become untenable
	as in the case of the original political business cycle theory
	which is frequently criticized for ignoring the possibility that
	one party prefers low unemployment whereas the other party low inflation,
	leading to a competing theory that the poilitical business cycle
	revolves around partisan changeovers
	instead of electoral calendar.
	Therefore,
	if a theory of cyclical pattern of policies
	that revolves around electoral calendar is to remain
	robust against heterogeneity or change over time of preferences among the electorate,
	policy congruence should be treated as the ultimate
	goal for voters.
	
%	Theoretically defensible as they might be,
%	these approaches miss out the point that voters ultimately want policy outcomes
%	and incumbents' competence is merely in service of this goaal.
%	If competence alone counts,
%	it may lead to an absurd conclusion that
%	a pro-life voter would reelect a governor
%	\textit{because} the governor signed a law that bans abortion.
%	Intuition suggests that the said voter would only reelect the governor
%	\textit{notwithstanding} that the governor approved the abortion ban.
	
	
	
	
	
%	2.3 Here begins the sketch of policymaking process without recall
	
	Finally,
	this theory further assumes additional institutional arrangements
	on which this theory will be built.
	It is assumed that,
	with or without recall institution,
	both the chief executive and the legislators normall serve a fixed term
	bookended general elections held on exogenously fixed election day
	where voters choose choose both the chief executive and the legislator.
	Finally, both the chief executive and the legislators are chosen
	by constituents in single-member districts (SMDs) under
	the first-past-the-post (FPTP) electoral system.
	Naturally, the chief executive is elected by a jurisdiction-wide SMD
	and the legislators are elected by SMDs
	which are geographically bounded constiuencies that partition the entire jurisdiction.
	

	Now that all the assumptions about voters and incumbent legislators
	are in place,
	the prediction of the political legislation cycle can be spelled out.
	As voters attribute blames and praise of policy outcomes
	to the government party which is
	in charge of implementing them,
	they electorally reward and punish government party incumbents
	based on these policies as well.
	Given that the government party is very unlikely to lose safe seats
	but very likely to lose swing seats in marginal constituencies,
	its main goal is to help reelect their members
	standing in the marginal constituency elections.
	Knowing
	that voters discount the utility of policy congruence
	by the amount of time elasped since the bill was passed,
	the government party in the legislature
	can pass extreme policies early in the legislative term,
	and dole out moderate policies in the run up to the next election
	to win the retain their seats in \textit{marginal constituencies}
	needed to maintain a workable majority for government bills
	in the legislature.
	Hence,
	I make the first hypothesis that
	
	Hypothesis 0.0: When recall is not feasible,
	the government bills passed earlier in the legislative term are on average  
	more extreme than those passed later in the legislative term,
	especially the last year of the legislative term.
	
	In the language of spatial theory of voting,
	Hypotheis 0.0 can also be expressed as thus:
	
	Hypothesis 0.0: When recall becomes feasible,
	the distances between government bills passed earlier in the legislative term
	and those passed later in the legislative term,
	espeically the last year of the legislative term,
	in the policy space are on average positive.
	
	Furthermore,
	since
	this political legislation cycle theory is predicated on
	the government party's ability to pass extreme policies
	and its need to win elections
	in marginal constituencies
	in order to maintain a majority for government bills,
	such behavior is conditional on
	the size of the government party size in the legislature,
	the number of marginal constituencies
	represented by the members of the government party.
	Strobl, Bäck, Müller (2021),
	argue that the government is less likely to pass austerity policy
	early in the legislative term
	both when it does not hold a majority,
	which gives the party in opposition the
	power to forestall it or prevent it altogether
	and when the government party has a large majority,
	which makes it easier to appease voters.
	Following
	I make the analogous hypothesis that
	
	Hypothesis 0.1: When recall is not feasible,
	the closer the size of the government party is to the minimum-winning majority,
	the more likely its extreme bills will be passed early in the legislature term.
	
	One word of caution may be required
	to note that
	Hypothesis 0.1 does not predict that
	the government party will pass the most extreme policy
	when it controls a minimum-winning majority.
	It simply predicts that
	the government party will most likely to opportunistically
	time their policy implementation
	which intends to maximize the distance between
	policies introduced before and after the election day,
	which helps the government party to exploit the recency bias
	to the fullest extent.
	
	This political legislation cycle thus
	far does not bode well for moderate voters.
	It suggest that,
	given sufficient amount of time,
	the government party can bounce back from
	any dip in its popularity come election day,
	by passing policies that appease voters in the marginal constituencie.
	Consequently,
	legislators, especially those belonging to the government party,
	are effectively unaccountable for their policies
	passed early in their legislative term.
	
%	It follows that
%	the need to court voters in marginal constituencies
%	should have led to a relatively more moderate policy outcome.
%	Voters have little opportunity to punish the government party
%	early in its term.
	
	
%	3.1 Here begins the sketch of policymaking process with recall
	
	This is precisely where the recall comes in
	to enhance legislators' accountability to their constituents
	throughout the legislative term.
	Voters no longer have to wait until the election day
	to punish the government party in the legislature for bad policy outcome.
	Instead,
	they can do that immediately after the government party passes extreme bills or,
	better yet,
	threaten to punish the government party if it passes the bills.
	This time around,
	the recency bias works in favor of voters
	seeking to hold legislators in the middle of their term,
	as it results in a voter choice at tge recall election
	based for most part on the asssessment of the recent policies.
	and
	constitutes a threat the credibility of which decreases in the
	legislator's margin of victory.
	Consequently,
	voters in the marginal constituency are in a better position to
	demand the government party to pass bills that
	are closer to their aggregate preferences
	by wielding the threat of recall leading to this first hypothesis
	about recall's incentive effect:
	
	Hypothesis 1.0: When recall becomes feasible,
	the maximum distance between government bills
	and the preferences of the marginal constituencies will be lower
	than that before recall becomes feasible.
	
	In addition,
	if recall can be triggered at anytime,
	it no longer makes sense
	for the government party
	to pass extreme policies
	\textit{in the hope that}
	voters can forgive them for passing moderate policies
	that are more congruent with their preferences.
	Instead,
	the government party now expects the threat of recall 
	of its members representing marginal constituencies
	to materialize very soon if the extreme policies are to be passed
	at anytime,
	which defeats the purpose of engineering a political legislation cycle in the first place -
	to pass extreme policies \textit{without} losing swing seats.
	This leads to the following hypothesis about legislative recall:
	
	Hypothesis 1.1: When recall becomes feasible,
	the distance between government bills passed earlier in the legislative term
	and those passed later in the legislative term,
	espeically the last year of the legislative term,
	in the policy space decreases.
	
%	Hence, making recall feasible flips the recency effect on its side
%	and thus disincentivizes the government party from engineering
%	a political legislation cycle.
	
	In addition,
	since
	the threat of recall is more credible in marginal constituencies,
	its effect on the government party's policies
	will also be the most pronounced when the government party
	has the strongest need for marginal seats.
	Per the argeuments of
	Strobl, Bäck, Müller (2021)
	sketched above,
	the need for marginal seats is the strongest
	when the government has the minimum-winning majority.
	As a result, recall's ability to undermine the political business cycle
	is at its strongest when the government has the minimum-winning majority.
	This leads to the supplementary hypotheses below:
	
	Hypothesis 1.0.0: When recall is feasible,
	further than that before recall becomes feasible.
	the closer the size of the government party is the minimum-winning majority,
	the closer
	the maximum distance between government bills
	and the preference of the marginal constituencies will be.
	
	Hypothesis 1.1.0: When recall is feasible,
	the closer the size of the government party is the minimum-winning majority,
	the lesser the distance is between government bills passed earlier in the legislative term
	and those passed later in the legislative term,
	espeically the last year of the legislative term,
	in the policy space.
	
	Of course,
	even if
	the foregoing hypotheses hold,
	recall may not completely eliminate political legislation cycle.
	Government parties in legislatures around the world
	often have to implement unpopular policies for various reasons,
	and elections may still be capable of generating
	a window of opportunities for passing extreme policies,
	even when recall is in the picture.
	Voters' positive feeling toward the chief executive
	ate the beginning of the executive term,
	which, under the previously stated assumptions,
	coincide with the legislative term,
	may well spill over into the legislative arena,
	thus allowing the government party in the legislature
	to pass some extreme policies in the legislature during that period
	without losing many of its government party legislators
	at recall or general elections,
	thus continuing the political legislation cycle.
	Thus, it is entirely possible that
	recall
	may simply produce a damping effect on political legislation cycle,
	without eliminating it.
	
	Suppose the political legislation cycle remains,
	would recall generate other incentive effect
	that distorts the cycle by encouraging the government party
	to \textit{intentionally} pass more extreme policies
	toward the middle of the term?
	Naturally,
	negotiation within the government party and
	that between the government party and the opposition
	may prevent bill passage to take place as soon as possible,
	but that should remain the goal if the government party operates on
	the assumption that voter has recency bias.
	Would recall introduce a new ideal timing for passing extreme policies?
	
	While this question deserves
	more detailed treatment in the future,
	my intuition suggests
	the answer is most likely no.
	The above argument based on the existence of honeymoon period,
	suggests that providing for recall throughout the legislative term
	would not prevent the government party from
	considering it ideal to implement extreme policies
	during the honeymoon period at the beginning of the term
	as soon as possible.
	Any institutional arrangements
	that prevent voters from recalling the incumbents
	at the beginning of their term would only reinforce this tendency.
	
	Technically speaking,
	passing extreme policies in the middle of the term.
	Suppose voters are forbidden from
	recalling incumbents
	during some idiosyncratic time period in the middle of their term,
	say the second year of a four-year term,
	then it may make sense to predict that
	extreme bills would be more like
	to be passed in the second year of a four-year term.
	That sort of institutional arrangement,
	though,
	is difficult to justify and still more difficult to implement.
	Even if recall is permissible throughout the legislative term.
	it still makes sense to pass extreme policies
	as soon as possible after the election is over.
	
	Ultimately,
	the ideal political legislation cycle desired by the government party
	should see the passage of extreme policies
	right after the election day
	and the passge of moderate policies right before the election day.
%	To carry the anaology of cycle further,
%	if the political legislation cycle is to remain
%	after the recall becomes feasible,
%	one shouldn't predict any translational effect of the cycle of legislation
%	which moves the peak and trough to a different point in time,
%	despite the incentive effect of recall.
	
	
%	4.1 Personal vote is back, baby.	
	
	
	
	However,
	while the ideal timing for passing extreme and moderate policies
	would most likely remain the same even
	when recall becomes feasible,
	the actual timing for passing extreme policies in the legislature,
	may actually be distorted as a result of recall institutional arrangements.
	In the hypothesis about legislative recall
	the government party with a minimum winning
	has the greatest incentive to pass the extreme policies as soon as possible,
	but will it be able to do so?
	
	This question offers a good opportunity
	to revisit the previous assumption that personal vote is not important.
	Under this assumption,
	chossing the timing of bill passage at any point during the legislative term 
	poses no difficulty as long as the government party controls the majority,
	and a minimum winnning majority would allow the government
	to pass extreme policies right at the beginning of the legsilative term.
	This is where the assumption truly stretches credulity
	given that passing extreme bills at a great speed
	does not seem like a strong suit of the government party
	holding only a minimum-winning majority in the legislature.
	In fact, protracted bargaining where the government party tries
	to win the support of legislators on the fence seems to be the norm
	
	Though there may be a number of ways to
	explain away the protracted bargaining process,
	consideration of personal vote presents an easy path.
	When the government with a minimum winning majority
	seeks to pass an extreme bill which conflicts with
	the constituency preferences,
	a legislator has an incentive
	to kill the bill
	unless the party doles out
	enough side payment for supporting the bill.
%	
%	
%	Unfortunately,
%	the legislative bargaining process is complex and difficult
%	to predict even on a case-to-case basis.
	Here, I
%	only
	suggest one way
%	out of many mechanisms
	recall institutional arrangement may affect
	the actual timing of the actual timing.
	Remember that
	when recall is assumed to be feasible throughout the legislative term,
	there is no reason particular reason
	to pass an extreme bill at anytime other than the beginning of the legislative term
	since voters can always recall legislators
	immediately after they pass it.
	That assumption is violated
	in some jurisdictions, Taiwan included,
	where there exist legal prohibition against recalling incumbents
	until a period of time after the term began.
	While this type of legal protection against early recall
	has the potential to preserve the political legislation cyle,
	it also means that there is a clear disincentive
	against passing extreme bills
	\textit{after} a certain point in time during the legislature.
	This means that it is possible to observe a sharp drop in
	the policy extremeness after that point in time.
	Such drop is especially likely to be observed
	when the size of the government party
	in the legislature is close to minimum-winning majority
	where means the government party members' incentives
	to haggle and to avoid recall
	are both at its strongest.
%	Nevertheless, such a long drop in bill extremeness
%	cannot be guaranteed to happen all the time.
%	Suppose there is a long period during which
%	incumbents are immune from recall,
%	incumbents may well have enough time to
%	finish all the bargaining in time to pass extreme policies.
%%	Furthermore, 
%%	recall may well be unfeasible,
%%	when the period during which incumbents are legally immune from recall
%%	for too long,
%%	which means that none of the mechanism
%%	specified above will ever kick in.
%	Investigation of the effect of the \textit{length} of period
%	during which incumbents are immune from recall
%	will unfortunately have to wait for another time.
	This offers an important test of recall's
	incentive effect on political legislation cycle
	which will be explored further later
	when discussing how to put the theories to test
	in the baseline case of Taiwan.
	
	More importantly,
	bringing the personal vote back in
	not only sheds light on the policy timing
	but also other legislative behavior
	that is directly tied to 
	cultivating a personal vote
	and ultimately their accountability to voters.
	Before the recall sets in,
	government party legislators' need for reelection
	can be somewhat reliably met by the
	political legislation cycle
	which generates more positive views toward them
	in the run-up to the election.
	Unfortunately,
	the government party legislators no longer enjoy this
	benefit when extreme government bills.
	Consequently,
	if the government party ever wants to pass extreme government bills
	when recall is possible,
	it needs its members in the legislature to
	cultivate a \textit{stronger} personal vote
	to ward off recall or threat thereof.
	
	First, the need for a stronger personal votes
	required to pass extreme government bills
	without losing its members representing marginal constituencies
	induces the government party,
%	I specify two reasons the government party wants
%	its members to cultivate a stronger personal vote
%	as a result of the recall.
%	First,
	to convince the voters that its members 
	are not merely vehicles
	by allowing them to take dissenting positions
	that is costly to the government
	including votes against party.
	Otherwise,
	the voters represented by the government party legislator,
	believing that their legislator will never oppose the government bills,
	will always threaten to recall the legislator,
	to prevent extreme government bills from passing,
	ultimately threatening to deprive a workable majority for government bills.
	leading to the hypothesis that
	
	Hypothesis 2.0:
	Recall causes
	the position of the government party members
	that represent marginal constituencies to move away from the party position. 
	
	
	Second,
	the government party also wants to encourage
	its members in marginal constituencies to cultivate a stronger personal vote
	by giving them even more pork for which they could claim credit than they received before,
	which entails the following hypothesis
	
	Hypothesis 2.1:
	Recall causes
	the marginal constituencies to receive a greater amount of pork
	than it otherwise would have.
	
	There are other ways the government party legislators
	could cultivate a stronger personal vote.
	However, since these methods may be more specific
	to the how incumbents did it in the past
	before recall becomes feasible,
	I will provide more hypotheses about them in the next section
	where I discuss how to test the more general hypotheses
	and context-specific hypotheses in Taiwan's legislative arena.
	
	
	
	
	
	
	
	
	
%	To understand the effect of recall,	
%	it helps to begin by understanding the policymaking process in Taiwan
%	when recall is in no way feasible
%	through the perspective of political business cycle.
%	The theory of political business cycle
%	was first formalized by Nordhaus to explain economic cycles
%	with political institution,
%	more specifically that of fixed-term election.
%	The theory,
%	however,
%	has been subject to a myriad of criticism
%	from both theoretical and empirical angles.
%	Nevertheless,
%	the fundamental insights that eletions matter
%	to the timing of policy remains compelling.
%	Details of the mechanisms have been modified,
%	and focus of studies have shifted from
%	the electoral cycle's effect on macroeconomic conditions
%	to policy instruments,
%	yielding a more refined and conditional mechanism of
%	electoral cycle of various policy instruments.
%	
%	
%	
%	
%	
%	
%	
%	In Taiwan,
%	both the president and the legislators
%	are elected in the same general election
%	and serve the same fixed term of four years.
%	Hence, the president and the legislature
%	can wor
	
	
	
	
	
	
	
	
	
	
	
	
	
	
	
	
	
	
	
	
	
	
	
	
	
	
	
	
	
	
	
	
	
	
	
	
	
	
	
	
	
	
	
	
	
	
	
	
	
	
	
	
	
	
	
	
	
	
	
	
	
	
	
	
	
	
	
	
	
	
	
	
	
	
	
	
	
	
	
	
	
	
	
	
	
	
	
	
	
	
	
	
	
	
	
	
	
	
	
	
	
	
	
	
	
	
	
	
	
	
	
	
	
	
	
	
	
	
	
	
	
	
	
	
	
	
	
	
	
	
	
	
	
	
	
	
	
	
	
	
	
	
	
	
	
	
	
	
	
	
	
	
	
	
	
	
	
	
	
	
	
	
	
	
	
	
	
	
	
	
	
	
	
	
	
	
	
	
	
	
	
	
	
	
	
	
	
	
	
	
	
	
	
	
	
	
	
	
	
	
	
	
	
	
	
	
	
	
	
	
	
	
	
	
	
	
	
	
	
	
	
	
	
	
	
	
	
	
	
	
	
	
	
	
	
	
	
	
	
	
	
	
	
	
	
	
	
	
	
	
	
	
	
	
	
	
	
	
	
	
	
	
	
	
	
	
	
	
	
	
	
	
	
	
	
	
	
	
	
	
	
	
	
	
	
	
	
	
	
	
	
	
	
	
	
	
	
	
	
	
	
	
	
	
	
	
	
	
	
	
	
	
	
	
	
	
	
	
	
	
	
	
	
	
	
	
	
	
	
	
	
	
	
	
	
	
	
	
	
	
	
	
	
	
	
	
	
	
	
	
	
	
	
	
	
	
	
	
	
	
	
	
	
	
	
	
	
	
	
	
	
	
	
	
	
	
	
	
	
	
	
	
	
	
	
	
	
	
	
	
	
	
	
	
	
	
	
	
	
	
	
	
	
	
	
	
	
	
	
	
	
	
	
	
	
	
	
	
	
	
	
	
	
	
	
	
	
	
	
	
	
	
	
	
	
	
	
	
	
	
	
	
	
	
	
	
	
	
	
	
	
	
	
	
	
	
%	While spatial theory of voting
%	frequently begins with a set of policies
%	in a policy space
%	and voters decide which policies to support
%	by measuring the distance of the policies to their ideal points,
%	or in other words,
%	how extreme these policies are too the voters,
%	it poses measurement issues as to where to locate the policy on a policy space.
%	Many studies sidestep these measurement by focusing on one kind of policies
%	where measurement of policy position is more inferrable from its economic impact
%	or on other measurements of policymaking process that does not
%	pertain to the the extremeness of policies but still contain information
%	political business cycle sidesteps the need to 

		
%	\newpage
%	\printbibliography
	
	
\end{document}

