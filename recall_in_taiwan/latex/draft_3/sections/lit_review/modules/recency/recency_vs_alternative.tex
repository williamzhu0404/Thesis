% !TeX root = recency_vs_alternative.tex

\documentclass[hyphens, crop=false]{standalone}

\usepackage[
reflist=true,
%noreprint, 
natbib=true,
autocite=inline,
style=windycity]
{biblatex}


\usepackage[american]{babel}
\usepackage{csquotes}

\usepackage{url}


\usepackage[multiple]{footmisc}

\usepackage{setspace}
\doublespacing



\addbibresource{lit_review.bib}


\begin{document}

	As I have previewed in the previous section,
	recall on its own directly incentivizes
	incumbents to cultivate a stronger personal vote,
	as party vote is much harder to influence
	and much less useful in a recall where
	parties feature less prominently
	than they do in general elections.
	It remains to see how recall affects parties,
	which in turn respond by
	imposing their own set of incentive structures
	that together with the direct incentive effect of recall
	and determine the net incentive effect of recall,
	the main motivation of this study.
	Therein also lies the key difficulty as a party,
	created by its individual members for their own reelection,
	are tasked with providing public goods for its members
	and thus must determine how much ``tax" to exact from them
	in the form of disciplinary action.
	Thus,
	when recall is instituted,
	the party must decide whether to impose more or less tax/discipline. 
	If past experimentation of anti-party reform is any indication,
	such decision is very difficult to predict.
	
	In the interest of promoting transparency
	while staying true to the exploratory nature of this study,
	I will provide in this section the main theory and in the next section alternative theories.
%	before I undertake to examine the data.
%	What separates the main theory from alternative ones
%	is whether the theory is falsifiable by the data available.
	Reader shall soon see that the main theory predicts that
	a party, especially that in government,
	will \textit{always} agree with their members that
	more personal vote of some sort will be in the interest of
	both the incumbents and their party,
	resulting in an increase in activities done toward that end. 
	Given that in the baseline case of Taiwan,
	incentives encouraging incumbents to a strong personal vote is very strong,
	there is a good chance of falsifying it with
	data of incumbent policy positions and performance alone.
	Alternative theories,
	on the other hand,
	require more variables to predict whether a net increase or decrease in
	activities aimed at cultivating personal vote will result.
	Testing the alternative theories then
	require more variables
	and thus deserve more in-depth treatment in the future.
	

		
%	\newpage
%	\printbibliography
	
	
\end{document}

