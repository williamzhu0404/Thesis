% !TeX root = personal_vs_party.tex

\documentclass[hyphens, crop=false]{standalone}

\usepackage[
reflist=true,
%noreprint, 
natbib=true,
autocite=inline,
style=windycity]
{biblatex}


\usepackage[american]{babel}
\usepackage{csquotes}

\usepackage{url}


\usepackage[multiple]{footmisc}

\usepackage{setspace}
\doublespacing


\addbibresource{lit_review.bib}


\begin{document}
	
	The above definitions of the personal vote and party vote provide a nice segue
	to the next stylized fact:
	incentive structures that encourages incumbents to pursue a strong personal vote
	generally improves their accountability to their constituents.
	Given that recall elections are essentially
	up or down votes on the incumbents
	and thus greatly ressemble elections held under the
	first-past-the-post (FPTP) electoral system,
	which is known for incentivizing incumbents to cultivate a strong personal vote,
	one should expect recall elections
	to improve incumbents' accountability to their constituents.
	
	Unfortunately the second stylized fact on its own does not carry the argument very far
	in large part due to the fact that an incumbent is usually
	unable to the discussion above which has established that
	pursuing a strong personal vote generally conflicts with
	retraining the party vote.
	One may virtually rest assured that parties would adapt to
	the new institution by creating new incentive structures on top of it
	if doing so suits the need of the parties.
	As a result,
	it is difficult to assert that recall
	always enhances accountability of incumbents to their constituencies
	without qualification.
	
	
%	\newpage
%	\printbibliography
	
	
\end{document}

