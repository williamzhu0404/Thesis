% !TeX root = personal_vote_definition.tex

\documentclass[hyphens, crop=false]{standalone}

\usepackage[
reflist=true,
%noreprint, 
natbib=true,
autocite=inline,
style=windycity]
{biblatex}


\usepackage[american]{babel}
\usepackage{csquotes}

\usepackage{url}


\usepackage[multiple]{footmisc}

\usepackage{setspace}
\doublespacing



\addbibresource{lit_review.bib}


\begin{document}
		
		%		<!-- Definition -->
		Cain, Ferejohn and Fiorina
		\autocite*{cainPersonalVoteConstituency1987}
		define the personal vote as the ``portion of a candidate's electoral support which originates in his or her
		personal qualities,
		qualifications,
		activities,
		and record." 
		\autocite*[9]{cainPersonalVoteConstituency1987}
		The significance of the personal vote has long been recognized,
		though the employment of this term is not so universal as it manifests itself throughf a different name or an analogous concept
		(electoral connection
		\autocites[home style][]{fennoHomeStyleHouse1978}[electoral connection][]{mayhewCongressElectoralConnection1974}[personal reputation][]{careyIncentivesCultivatePersonal1995}[dyadic representation][]{millerConstituencyInfluenceCongress1963}{weissbergCollectiveVsDyadic1978}{ansolabehereDyadicRepresentation2011}[personal representation][]{colomerPersonalRepresentationNeglected2011}[local vote][]{pattieWinningLocalVote1995}.
		To clarify their defintion of the personal vote,
		Cain, Ferejohn and Fiorina
		\autocite*[9]{cainPersonalVoteConstituency1987}
		further emphasize the personal vote does not include
		\begin{quotation}
			support for the candidate based on his or her partisan affiliation, fixed voter charactersistics such as class, religion, and ethnicity, reactions to national conditions such as the state of the economy and performance evaluations centered on the head of the governing party
%			\autocite*[9]{cainPersonalVoteConstituency1987}
		\end{quotation}
		which will serve as my working defintion of ``party vote" are distinct types of support.
		
		
		While
		these definitions offered by
		\citeauthor{cainPersonalVoteConstituency1987}
		\autocite*{cainPersonalVoteConstituency1987}
		who coined and defined the term ``personal vote"
		clearly demarcate two different kinds of support
		received by a candidate at the poll,
		they have not prevented other political scientists from
		putting forward their own definitions of these terms or their respective analogs.
		More importantly,
		to differentiate these two kinds of support
		do suggest that purusing both of them simultaneously present a conflict,
		which,
		as
		\citeauthor{carseyRethinkingNormalVote2017}
		\autocite*{carseyRethinkingNormalVote2017}
		posit,
		need not be a real one.
%		
%		More specifically,
		\citeauthor{carseyRethinkingNormalVote2017}
		\autocite*{carseyRethinkingNormalVote2017}
		argue that,
		though a candidate cultivating a personal vote based on policy orientation
		would almost inevitably conflict with seeking a party vote based on party platform,
		it is theoretically possible to
		seek electoral support through non-policy work like providing constituency service
		without undermining the incumbent's pursuit of party vote.
		Whether such proposition is true,
		\citeauthor{carseyRethinkingNormalVote2017}
		\autocite*{carseyRethinkingNormalVote2017}
		argue,
		should only be determined empirically.
%		
		Though I do believe that such conflict is real, even in non-policy areas,
		it seems prudent
		to recognize this validity of this critique 
		by explicitly dividing both party vote and personal vote into
		a
		policy component
		and a
		catch-all
		non-policy component
		and
		then
		focus on their conflict in policy component for the time being.
		
%		Even though this appears to exceed the scope portion of the support driven by the concept of partisan identification as defined by
%		\citeauthor{campbellAmericanVoter1960}
%		\autocite*[121]{campbellAmericanVoter1960},

%		it does mirror the national partisan electoral swing which,
%		per
%		Mayhew
%		\autocite*[28,32]{mayhewCongressElectoralConnection1974},
%		cannot be reasonably be expected to controlled by
%		individual legislators,
%		especially those representing marginal constituencies,
%		and thus should be treat as acts of God.
%		It should become apparent here that
%		the key distinction here between the personal vote and the party vote is
%		the degree to which an individual legislator may influence.
%		It follows that the definition ties in with
%		the portions of the policies and economic condition
%		for which individual legislators and parties may credibly seek credit,
%		and only parties may credibly claim to be the state of economy and the success of policies that impact the entire electorate
%		\autocite{mayhewCongressElectoralConnection1974}.
%		Therefore,
%		the definitions of the personal vote and the party vote clearly correspond to the issues for which voters hold them responsible. Similarly,
%		the party vote is also manifested through a different name or analogous concept
%		\autocites[partisan representation][]{hurleyPartisanRepresentationRealignment1991}[collective representation][]{weissbergCollectiveVsDyadic1978}[normal vote][]{converseConceptNormalVote1966}[partisan reputation][]{careyIncentivesCultivatePersonal1995}
		
		%		Note to self: converse's concept of normal vote needs better citation --> under many circumstances.
	
		
%	\newpage
%	\printbibliography
	
	
\end{document}

