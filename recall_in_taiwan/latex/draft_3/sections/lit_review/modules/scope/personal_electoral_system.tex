% !TeX root = personal_electoral_system.tex

\documentclass[hyphens, crop=false]{standalone}

\usepackage[
reflist=true,
%noreprint, 
natbib=true,
autocite=inline,
style=windycity]
{biblatex}


\usepackage[american]{babel}
\usepackage{csquotes}

\usepackage{url}


\usepackage[multiple]{footmisc}

\usepackage{setspace}
\doublespacing



\addbibresource{lit_review.bib}


\begin{document}
		
		Having specified the scope of recall under consideration,
		I now proceed to describe the broader institutional arrangements associated with recall within the scope of study.
		As it turns out,
		one of
		the most pertinent arrangement
		- electoral system -
		is actually tightly connected to the scope.
		Recall is usually justified as a way for people to retract their approval of incumbent officials,
		whose approval is formally granted through electoral system.
		Many electoral system differ as to who gets the approval
		and
		may be thus divided into categories where
		one category named candidate-centered electoral system gives voter the right to elect an individual to an office
		and
		another category, usually labeled party-centered electoral system, grants the electorate the right to send a group of politicians to
		a number of positions at a time, usually in the legislature,
		typically by the strength of their parties in the election.
		%		and the third category, usually employed in legislative elections,
		%		creates two tiers elected positions, where
		%		one tier, usually called nominal tier, consists of individual politicians elected based on the votes they win individually
		%		and the other, often referred to as party tier, politicians elected from a party list based on the votes their parties win.
		%		The first category is called candidate-based electoral system whereas the second one party based electoral system and the third one mixed-member electoral system.
		It wouldn't take long to see that recalling individual incumbents who take their office based on a party list is hard to justify
		as it contradicts the purpose of insitituting a party list, i.e. giving the party,
		instead of individual legsilators,
		the direct responsibility for, and thus the control over, policies and personnel in the legislature.
		Some make the even bolder claim that recall of individuals is incompatible with party-centered electoral systems.
		For these reasons, it is safe to assume that only incubents elected under an candidate-centered electoral system may be recalled.
	
		Furthermore,
		the candidate-centered electoral system used for returning congresspeople encourages them to pursue more personal votes which are more valuable for legislators competing in party-centered electoral systems.
		These findings have similarly been corroborated across a varieties of institutional settings
		\autocite{careyIncentivesCultivatePersonal1995}.
		%		<!-- universal occurrence of pursuit of personal vote  --> 
		That is not to say,
		however,
		that legislators contesting under party-based electoral systems do nothing to enhance their electoral independence from the party.
		Even in some of the most party-based electoral systems,
		efforts have been devoted to shield candidates from the uncertain electoral fate of the party by developing their personal brands,
		sometimes at the behest of party leadership
		\autocite{shugartLookingLocalsVoter2005},
		suggesting that activities geared toward cultivating a personal vote is quite universal,
		even though the intensity of such pursuit varies.
		
		
		
%	\newpage
%	\printbibliography
	
	
\end{document}

