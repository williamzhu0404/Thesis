% !TeX root = personal_recall.tex

\documentclass[hyphens, crop=false]{standalone}

\usepackage[
reflist=true,
%noreprint, 
natbib=true,
autocite=inline,
style=windycity]
{biblatex}


\usepackage[american]{babel}
\usepackage{csquotes}

\usepackage{url}


\usepackage[multiple]{footmisc}

\usepackage{setspace}
\doublespacing



\addbibresource{lit_review.bib}


\begin{document}
	Given that party vote and personal vote generally conflict in their respective policy components,
	it will become apparent that,
	incentive effect of
	recall of collecitivities
	is harder to predict than
	that of recall of individual.
	When recall of individual is in place,
	incumbents are generally incentivized to pursue single-mindedly personal vote
	and only seek to pursue party vote when doing so helps maximize personal vote
	in order
	to prevent a recall altogether as
	the party might very well abandon the incumbent
	should the threat of recall materialized;
	whereas when recall of collectivity is in place,
	the incentive effect becomes harder to predict
	given that party may very well
	hope to ward off the recall threat by cultivating
	party vote and personal vote across all consitituencies
	since both kinds of support help the party.
	Therefore,
	to predict how incumbents change their behavior
	when recall of collectivity
	requires an examination of how party
	as a whole balances the pursuit of personal vote and party vote,
	which is too complex to be treated here
	and thus cannot and will not be included in the scope condition of the thoery
	proposed in this study.
	
	

	
		
%	\newpage
%	\printbibliography
	
	
\end{document}

