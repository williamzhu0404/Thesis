% !TeX root = personal_vs_party.tex

\documentclass[hyphens, crop=false]{standalone}

\usepackage[
reflist=true,
%noreprint, 
natbib=true,
autocite=inline,
style=windycity]
{biblatex}


\usepackage[american]{babel}
\usepackage{csquotes}

\usepackage{url}


\usepackage[multiple]{footmisc}

\usepackage{setspace}
\doublespacing



\addbibresource{lit_review.bib}


\begin{document}
		While
		these definitions offered by
		\citeauthor{cainPersonalVoteConstituency1987}
		\autocite*{cainPersonalVoteConstituency1987}
		who coined and defined the term ``personal vote"
		clearly demarcate two different kinds of support
		received by a candidate at the poll,
		they have not prevented other political scientists from
		putting forward their own definitions of these terms or their respective analogs.
		These definitions
		understand the relationship between support for party and support for candidate and,
		as
		\citeauthor{carseyRethinkingNormalVote2017}
		\autocite*{carseyRethinkingNormalVote2017}
		put it,
		may lead to divergent theoretical prediction
		of whether cultivating more personal vote conflicts with the goal of pursuing more party votes.
		More specifically,
		\citeauthor{carseyRethinkingNormalVote2017}
		\autocite*{carseyRethinkingNormalVote2017}
		argue that,
		when support for the candidate
		based on policy orientation is excluded from both party and personal votes,
		the relationship between party vote and personal vote
		should only be determined empirically.
		%		<!-- position taking issues -->
%		\citeauthor{carseyRethinkingNormalVote2017}
%		\autocite*{carseyRethinkingNormalVote2017}
%		first divide personal vote into two components:
%		one is ideological and the other non-ideological 
%		\autocite*[467-468]{carseyRethinkingNormalVote2017}.
%		Then they argue that since voters identified with the politician's party and those not identified with that party evaluate these two components differently
%		and it is impossible to know how they do it except through assessment of empirical evidence,
%		it follows that it is also impossible to know whether devoting more effort to pursuing personal vote may cost them the party vote,
%		that is,
%		whether there is a real tradeoff between party vote and personal vote.

		Though
		I do not contest that how this tradeoff works depend on empirical assessment of voters' response to legislative behavior,
		I do contend that the tradeoff must exist at some point,
		and a simple thought experiment would suffice to demonstrate it,
%		Assume that in a constituency with a $100\%$ turnout rate,
%		a candidate wins all the votes.
%		Because the personal vote and party vote
%		consist of support that are mutually exclusive by definition,
%		one may, through an oracle, find out that a person has $a\%$ of the support is personal vote
%		$(100 -a)\%$ of voters of the support is party vote.
%		Then anytime
%		 
%		To suggest that there may be no tradeoff implies that a strategy that maximizes personal vote
%		maximizes party vote
%		as well.
%		And the only way that can work rests on an assumption that
%		the party imposes no discipline at all.
%		Under such a circumstance,
%		the party requires no effort to maintain
%		and operates 
%		
%		
%		Indeed
%		\citeauthor{carseyRethinkingNormalVote2017}
%		\autocite*{carseyRethinkingNormalVote2017}
%		admit that as long as ideology remains a substantial part of the contributing factors to personal vote,
%		choosing a more liberal position always carries the risk of losing the support of more conservative voters regardless of the voter's party identification
%		\autocite*[467]{carseyRethinkingNormalVote2017}.
		even if the personal vote can be conceptualized in a non-ideological/non-policy fashion and focuses only the vote gained by delivering constituency services,
		pork barrels,
		and other non-ideologically-motivated activities.
		Imagine a recluse who communicates with no one and lives alone and in a forest in the constituency.
		The same recluse decides to break the silence just once by sending an email to the legislator's office,
		and the staffer returns a boilerplate response without ever reviewing the request.
		Yet,
		the recluse,
		who originally supports the incumbent only because of the incumbent's party
		is now touched by the response from the incumbent's office and decides to support the incumbent more because of the casework the moment the recluse finishes reading the response.
		One may argue that at that moment,
		the personal vote for the incumbent increases and the party vote for the incumbent decreases,
		unless
		the personal vote for the incumbent decreases by exactly that much.
		
		
		who simply 
		one can imagine at the split-second when a legislator's office
		completed a casework for a recluse who used to vote for the incumbent based purely on party vote
		such that recluse now votes
		such that the recluse now
		
		
		it is still improbable that devoting the maximal effort toward cultivating personal vote does not conflict with their pursuit of party vote.
		Imagine a candidate who spends all the time doing casework and demanding pork for
		the constituents
		and never
		does anything
		the party asks for
		like endorsing the party
		A party does many things in addition to implementing policies.
		It
		distributes positions of power and influence,
		coordinates electoral strategy,
		demands endorsement and fund from its members
		for their copartisans in electoral campaigns.
		To say that pursuing personal vote and 
		
		Under time and resouce constraint,
		one penny and one second spent on heeding the needs of the constituents is one penny and one second not spent on party-building activities,
		say party strategy sessions,
		fundraising for the party's common war chest,
		and campaign activities endorsing copartisan candidates.
		Then to maximize pursuit of personal personal vote
		Common sense and political folklore would predict that evading all calls on their time and purse to be a good team player in the party in order to grow a personal following does not bode well for their reputation among the party leadership and local activists,
		which would undermine their chances of renomination and,
		if still successfully renominated,
		reelection.
		%		Tradeoff between two kinds of activities
		Therefore,
		I hold the conventional view that at a certain point,
		a candidate must face the tradeoff between the personal vote and the party vote which is not only grounded in classical microeconomic theory but also empirically supported
		\autocite{ansolabehereOldVotersNew2000,primoPartyStrengthPersonal2010}.
		%		 <!-- .
		A candidate's time and resources are limited and devoting effort toward cultivating a personal vote,
		though not necessarily detracting the candidate's party vote,
		necessarily crowds out time and resources that could have been spent on cultivating a party vote,
		leading to a trade-off between activities geared toward cultivating the personal vote and those intended for building the party vote.
		%		 <!-- hypothesis --> 
		This discussion also provides a nice segue to my central hypothesis regarding the behavioral consequence of the threat of recall: 
		\textbf{the time and resources a legislator expends toward cultivating personal vote,
			\textit{ceteris paribus,}
			increases in the legislator's vulnerability to recall attempts.}
		
%	\newpage
%	\printbibliography
	
	
\end{document}

