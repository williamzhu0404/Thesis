% !TeX root = lit_review_2.tex

\documentclass[hyphens, crop=false]{standalone}
\usepackage{import}
\usepackage[subpreambles=true]{standalone}

\usepackage[
reflist=true,
%noreprint, 
natbib=true,
autocite=inline,
style=windycity]
{biblatex}


\usepackage[american]{babel}
\usepackage{csquotes}

\usepackage{url}


\usepackage[multiple]{footmisc}

\usepackage{setspace}
\doublespacing



\addbibresource{lit_review.bib}


\begin{document}
%	\section*{Introudction}
	
	\section*{Some Definitions and Stylized Facts}
		I would like to begin this inquiry by making a simple observation that
		recall can be treated as just another election
		an incumbent needs to survive.
		This allows for appropriation of a number of findings and techniques
		that have been long established in electoral behavior literature,
		the first one of which being a simple stylized fact that
		each candidate with a realistic chance of winning an election belongs to a party.
		Then when constituents cast a vote for a candidate,
		one part of the decision is based on the candidate's personal tratis,
		and the other part depends on the performance of the candidate's party. 
		One may thus say that there is both
		a \textit{personal vote}
		and a \textit{party vote} for the candidate.
	
		\subsection*{Personal Vote and Party Vote}
			\import{./modules/stylized}{personal_vote_definition}
			
			\import{./modules/stylized}{stylized_fact}
				
	\section*{Setting the Baseline}
		
		\import{./modules/baseline}{baseline_intro}
		
		
		\subsection*{Recall in Taiwan as a Baseline}
		
			\import{./modules/baseline}{tw_electoral_reform}
			
			
	
	\section*{Main Theory}
	
	
	\import{./modules/recency}{recency_vs_alternative}
	
		\subsection*{Recency Bias}
			\import{./modules/recency}{recency_bias}
		
		
		\subsection*{A Game of Policy Sequencing}
			\import{./modules/recency}{policy_sequencing_informal}
		
		\subsection*{Predictions}
			
	\section*{Alternative Theories}
		
		\import{./modules/alternative}{lacking_credible_threat.tex}
		
		\import{./modules/alternative}{resource_depletion.tex}
		
%		
%	\section*{Dicussion of Empirical Strategy}
%		\subsection*{Research Design}
%		\subsection*{Measurements of Legislative Behavior}
%		
%	\section*{Tentative Conclusion}
			
\end{document}

