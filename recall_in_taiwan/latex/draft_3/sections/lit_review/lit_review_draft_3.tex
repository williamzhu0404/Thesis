% !TeX root = lit_review_draft_3.tex

\documentclass[hyphens, crop=false]{standalone}

\usepackage[
reflist=true,
%noreprint, 
natbib=true,
autocite=inline,
style=windycity]
{biblatex}


\usepackage[american]{babel}
\usepackage{csquotes}

\usepackage{url}


\usepackage[multiple]{footmisc}

\usepackage{setspace}
\doublespacing



\addbibresource{lit_review.bib}


\begin{document}
	\subsection*{Roadmap}
	\begin{enumerate}
		\item 
		General Theory
		
		Recall is just another election
		
		Voting theory in elections
		\begin{enumerate}
			\item 
			Theory of voting behavior
			\item 
			Theory of personal vote and party vote
			\item 
			Theory of recall
		\end{enumerate}
		\item 
		Scope Condition
		\begin{enumerate}
			\item 
			Prediction of individual politician behavior theoretically more straightforward
			\item 
			Observation of incentive effect on individual behavior more likely to be significant
			\item 
			Focus on recall of individuals instead of collectivities
			\item 
			Recall of individuals compatible only with candidate-based electoral system
		\end{enumerate}
		\item 
		Stylized instittutional arrangement of recall
		\begin{enumerate}
			\item 
			Similar to institutional arrangement of recall in Taiwan
			\item 
			No extraneous elements
			\item 
			Separate component
			\item 
			Similar to institutional arrangement of recall elsewehere
		\end{enumerate}
		\item 
		Main Theory
		\begin{enumerate}
			\item 
			Recall causes parties and individuals to agree that personal vote is needed
			\item 
			Parties need policies and seats, individuals need reelection
			\item 
			Recall makes policy more difficult to implement
			as recall could reduce the size of legislative party.
			\item 
			Without recall, parties could maintain their size through buying off voters with policy closer to the election
			\item 
			With recall, members of legislative parties are subject to swift retribution through recall.
		\end{enumerate}
		\item 
		Alternative Theory
		\begin{enumerate}
			\item 
			Recall leads to conflict between parties and individual politicians, relative strengths of parties and individuals would matter
			\item 
			Fear of resource depletion
			\item 
			Fear of credible signal of unpopularity
			\item 
			Fear of losing election
			\item 
			We unfortunately cannot determine which mechanism at work, we may only exclude certain mechanisms if our statistical asssessment may ascertain whether recall has incentive effect
		\end{enumerate}
	\end{enumerate}
	
	\subsection*{Sketch of General Theory}
	
		\subsubsection*{Theory of Voting}
		To sketch a theory of recall's incentive effect on elected officials
		working in a partisan political landscape, I would like to begin with an observation that,
		paraphrasing Mayhew
%		\citep{mayhewCongressElectoralConnection1974},
		\autocite*{mayhewCongressElectoralConnection1974},
		recall is just another election that incumbents need to survive.
		This observation suggests that,
		despite the minute size of literature of recall election,
		one may draw on the vast literature of voting behavior,
		especially that in legislative elections,
		to build the theory. 
		
		Unfortunately, the vastness of voting behavior literature has
		yet to produce a consensus on which model has the best predictive power.
		This is in large part due to
		these models' capability of explaining and predicting the same voting behavior,
		thus
		providing a slimmer of comfort in the fact that the choice of model wouldn't matter just yet.
		Currently, the attitudinal Michigan model
		that began with the eminent tradition of \textit{The American Voter}
		\autocite{campbellAmericanVoter1960}
		and the sptial model spearheaded by
		\citeauthor{downsEconomicTheoryDemocracy1957}
		\autocite*{downsEconomicTheoryDemocracy1957}
		are some of the most popular models of voting behavior.
		Given that to understand the role of party in shaping recall's incentive effect is
		a key goal of this study,
		\citeauthor{campbellAmericanVoter1960}'s
		\autocite*{campbellAmericanVoter1960}
		model will be the starting point of the study,
		as it recognizes the role of party from the get go
		whereas the party is frequently absent from spatial models,
		though the spatial model will remain a crucial tool
		and be employed from time to time.
		
		
		Following the tradition of \textit{The American Voter}
		\autocite*{campbellAmericanVoter1960},
		I begin by identifying three categories of factors that affect a voter's decision in an election:
		partisan identification, policy orientation, and candidate evaluation
		\autocites[59]{campbellAmericanVoter1960}[28]{lewis-beckAmericanVoterRevisited2008}.
		Though voters' decisions have long been dominated by voters' partisan identification,
		policy orientation and evaluation remain important components of their decision calculus such that electoral swings hinge on them.\footnote{For a more thorough review of the voting behavior literature, see
			Bartels \autocite*{bartelsStudyElectoralBehavior2010}
			and
			Jacoby \autocite*{jacobyAmericanVoter2010}.
		}
		Therefore,
		the confluence of these different kinds of factors means that the message of each election is frequently far from clear, notwithstanding widespread claims in mass media otherwise.
		From the vantage point of this classical theory of voting behavior,
		it does not take many extra steps to view institutions of direct democracy as means to allow the electorate to send a clearer message
		by filtering only the pertinent factor from the rest when they are called to the polling stations.
		Referenda and initiatives allow voters to express and enact their policy preferences directly by a simple yes or no an each proposition,
		potentially overcoming partisan divides and incumbents' stranglehold on popular measures.
		%	start examining recall 
		Similarly,
		recalls afford voters with a clean-cut fashion of evaluating their elected officials.
		When the recall election is held, one archetypal question is put to the voters:
		do you want to remove the incumbent officeholder from the office immediately?
		In this question,
		there is neither partisan cue nor policy questions,
		thus isolating voters from the influence of party identification and policy evaluation as much as possible.
		If the recall attempt fails, the incumbent gets to stay in the office.
		It is only when the recall is successful that voters might be given a chance to decide how to fill the vacancy.
		These institutional arrangements common to most if not all recalls mean that voters may to still vote for their own party's candidate should the recall succeeds,
		further removing a significant chunk of partisan identification from their decision calculus at the recall election.
		
		Nevertheless,
		theoretical potential afforded by the institution of direct democracy to express their preference for policies and incumbents may not materialize fully.
		Parties and popular elected officials have successfully mobilized the electorate to vote along party lines in referenda and initiatives
		\autocite{bowlerDirectDemocracyUnited2010,brantonExaminingIndividualLevelVoting2003}.
		Recall elections certainly are not immune to partisan coordination either
		\autocite{masketCaliforniaRecallSprint2016}.
		Nevertheless,
		the potential is always there,
		and arguably voters have repeatedly recalled elected officials
		that would have been reelected had forces of partisan identification been given a chance to prevail on the usual election day
		\autocite{gerstonRecallCaliforniaPolitical2004}
		Furthermore,
		relying on the forces of partisan identification to tide through a recall attempt is a particularly risky choice
		for elected officials fighting for their job in marginal constituencies where even a mild swing against them may give them the boot.
		It thus stands to reason that even when partisan coordination is taken into account or even assumed to be perfect,
		demographic shift,
		among other swing factors would reduce their job security,
		thus contributing the credible threat of recall,
		which,
		as the foregoing discussion suggests,
		causes voters to weigh their evaluation of the incumbent more than they they would in a regular election.
		To put it succinctly, recalls should make politicians realize that the security of their positions hinge more on their personal labels than before.
		Consequently,
		incumbents,
		%		<!-- Mayhew's assumption: single-minded seeker of reelections -->
		who are assumed to seek reelections single-mindedly
		\autocite{mayhewCongressElectoralConnection1974},
		demand more electoral independence from their parties
		\autocite{cainPersonalVoteConstituency1987},
		which is best achieved by cultivating personal votes for themselves.
		
		
		%		SubSubtitle
		\subsubsection*{Definition of Personal Vote}
		
		%		<!-- Definition -->
		Cain, Ferejohn and Fiorina
		\autocite*{cainPersonalVoteConstituency1987}
		define the personal vote as the "portion of a candidate's electoral support which originates in his or her
		personal qualities,
		qualifications,
		activities,
		and record." 
		\autocite*[9]{cainPersonalVoteConstituency1987}
		The significance of the personal vote has long been recognized,
		though the employment of this term is not so universal as it manifests itself throughf a different name or an analogous concept
		(electoral connection
		\autocites[home style][]{fennoHomeStyleHouse1978}[electoral connection][]{mayhewCongressElectoralConnection1974}[personal reputation][]{careyIncentivesCultivatePersonal1995}[dyadic representation][]{millerConstituencyInfluenceCongress1963}{weissbergCollectiveVsDyadic1978}{ansolabehereDyadicRepresentation2011}[personal representation][]{colomerPersonalRepresentationNeglected2011}[local vote][]{pattieWinningLocalVote1995}.
		To clarify their defintion of the personal vote,
		Cain, Ferejohn and Fiorina
		\autocite*[9]{cainPersonalVoteConstituency1987}
		further emphasize the personal vote does not include
		\begin{quotation}
			support for the candidate based on his or her partisan affiliation, fixed voter charactersistics such as class, religion, and ethnicity, reactions to national conditions such as the state of the economy and performance evaluations centered on the head of the governing party
%			\autocite*[9]{cainPersonalVoteConstituency1987}
		\end{quotation}
		which will serve as my working defintion of "party vote." Even though this appears to exceed the scope portion of the support driven by the concept of partisan identification as defined by
		\citeauthor{campbellAmericanVoter1960}
		\autocite*[121]{campbellAmericanVoter1960},
		it does mirror the national partisan electoral swing which,
		per
		Mayhew
		\autocite*[28,32]{mayhewCongressElectoralConnection1974},
		cannot be reasonably be expected to controlled by
		individual legislators,
		especially those representing marginal constituencies,
		and thus should be treat as acts of God.
		It should become apparent here that
		the key distinction here between the personal vote and the party vote is
		the degree to which an individual legislator may influence.
		It follows that the definition ties in with
		the portions of the policies and economic condition
		for which individual legislators and parties may credibly seek credit,
		and only parties may credibly claim to be the state of economy and the success of policies that impact the entire electorate
		\autocite{mayhewCongressElectoralConnection1974}.
		Therefore,
		the definitions of the personal vote and the party vote clearly correspond to the issues for which voters hold them responsible. Similarly,
		the party vote is also manifested through a different name or analogous concept
		\autocites[partisan representation][]{hurleyPartisanRepresentationRealignment1991}[collective representation][]{weissbergCollectiveVsDyadic1978}[normal vote][]{converseConceptNormalVote1966}[partisan reputation][]{careyIncentivesCultivatePersonal1995}
		%		Note to self: converse's concept of normal vote needs better citation --> under many circumstances.
		
		%		Subtitle
		\subsubsection*{Relationship between Personal and Party Votes}
		
		%		<!-- alterantive definition discussion -->
		
		Before I further detail my theoretical prediction of recall's effect on an incumbent's demand for personal vote,
		it behooves me to clarify the relationship between the personal vote and the party vote which,
		as
		\citeauthor{carseyRethinkingNormalVote2017}
		\autocite*{carseyRethinkingNormalVote2017}
		put it,
		may lead to divergent theoretical prediction when this relationship is understood differently and thus should only be determined empirically.
		%		<!-- position taking issues -->
		\citeauthor{carseyRethinkingNormalVote2017}
		\autocite*{carseyRethinkingNormalVote2017}
		first divide personal vote into two components:
		one is ideological and the other non-ideological 
		\autocite*[467-468]{carseyRethinkingNormalVote2017}.
		Then they argue that since voters identified with the politician's party and those not identified with that party evaluate these two components differently
		and it is impossible to know how they do it except through assessment of empirical evidence,
		it follows that it is also impossible to know whether devoting more effort to pursuing personal vote may cost them the party vote,
		that is,
		whether there is a real tradeoff between party vote and personal vote.
		
		While I do not contest that how this tradeoff works depend on empirical assessment of voters' response to legislative behavior,
		I do contend that the tradeoff must exist at some point.
		Indeed
		\citeauthor{carseyRethinkingNormalVote2017}
		\autocite*{carseyRethinkingNormalVote2017}
		admit that as long as ideology remains a substantial part of the contributing factors to personal vote,
		choosing a more liberal position always carries the risk of losing the support of more conservative voters regardless of the voter's party identification
		\autocite*[467]{carseyRethinkingNormalVote2017}.
		Even if the personal vote can be conceptualized in a non-ideological/non-policy fashion and focuses only the vote gained by delivering constituency services,
		pork barrels,
		and other non-ideologically-motivated activities,
		it is still improbable that devoting the maximal effort toward cultivating personal vote does not conflict with their pursuit of party vote.
		One penny and one second spent on heeding the needs of the constituents is one penny and one second not spent on party-building activities,
		say party strategy sessions,
		fundraising for the party's common war chest,
		and campaign activities endorsing copartisan candidates.
		Common sense and political folklore would predict that evading all calls on their time and purse to be a good team player in the party in order to grow a personal following does not bode well for their reputation among the party leadership and local activists,
		which would undermine their chances of renomination and,
		if still successfully renominated,
		reelection.
		%		Tradeoff between two kinds of activities
		Therefore,
		I hold the conventional view that at a certain point,
		a candidate must face the tradeoff between the personal vote and the party vote which is not only grounded in classical microeconomic theory but also empirically supported
		\autocite{ansolabehereOldVotersNew2000,primoPartyStrengthPersonal2010}.
		%		 <!-- .
		A candidate's time and resources are limited and devoting effort toward cultivating a personal vote,
		though not necessarily detracting the candidate's party vote,
		necessarily crowds out time and resources that could have been spent on cultivating a party vote,
		leading to a trade-off between activities geared toward cultivating the personal vote and those intended for building the party vote.
		%		 <!-- hypothesis --> 
		This discussion also provides a nice segue to my central hypothesis regarding the behavioral consequence of the threat of recall: 
		\textbf{the time and resources a legislator expends toward cultivating personal vote,
			\textit{ceteris paribus,}
			increases in the legislator's vulnerability to recall attempts.}
		\newpage
%	
%	[Note: This section comes after the background section which details the development of
%	institutions in Taiwan. As a result, some part of the text may reference
%	features of Taiwan. I apologize for any inconvenience this may cause.]
		\subsection*{Setting Scope Conditions}
%		While Taiwan's
%		unique institutional setting
%		gives us a chance to ascertain
%		the incentive effect of Taiwan's recall election on the behavior of the members of Taiwan's legislature,
%		it wouldn't be of much use beyond a policy evaluation unless
%		Taiwan's example could inform us of the property of recall elections
%		across a broader set of institutional arrangements.
		\subsubsection*{Cultivating a personal vote against recall threats as a legislator}
		
		%		<!-- In the foregoing analysis,
		%		I have invoked the conventional assumption the legislators are single-minded seekers of reelection [@mayhew1974]. Although an oversimplification of the goals of legislators, this assumption has been deemed not only plausible in the American context but also across a wide varieties of democratic assemblies. To facilitate analysis, I shall continue to follow this assumption until there is sufficient ground to challenge it since reelection is not only an important goal for legislators but also instrumental in, if not essential to their attainment of other goals in the legislative arena and beyond, such as policy demands and positions of power. -->
		
		As I have mentioned before,
		the recall is effectively yet another election incumbents need to survive to stay in the office,
		which is conventionally assumed to be the legislator's only goal.
		Toward that end,
		cultivating a personal vote can be an important strategy for all legsialtors.
		Of course,
		the extent to which this strategy is useful varies greatly depending on the broader institution in which legislators contend for survival.
		The pursuit of personal vote has long been documented and analyzed in the US where candidates belong to a party simply because they the candidates say so and the parties' ability to constrain their members' behavior is limited,
		providing more leeway for members of Congress to craft their own policies and positions
		\autocite{mayhewCongressElectoralConnection1974, coxLegislativeLeviathanParty2007, cainPersonalVoteConstituency1987}.
		The relative penury of the parties and the laxity of campaign finance regulations also allow individual candidates to build their own potent campaign machine independent of the party's support and devoted mostly to cultivating the candidates' personal bases of support.
		All told,
		the aforementioned factors that have helped the candidates to cultivate personal votes not only in the US but also in other democracies where weak parties or loose campaign finance regulations also persist
		\autocite{careyLegislativeVotingAccountability2008,crispCapturingVoteseekingIncentives2021,reedDemocracyPersonalVote1994}.
		
		
		Moving beyond the aforementioned direct indicators of party strengths,
		institutionalists have identified constitutional distribution of power and electoral systems
		\autocite{careyIncentivesCultivatePersonal1995,	finocchiaroMakingWashingtonWork2018}
		as a more fundamental cause of the greater utility of activities in aid of seeking personal vote.
		Again,
		the starting point of this line of query can be traced to the US Congress where those activities is extremely prevalent.
		%		<!-- constitutional distribution of power -->
		The US Constitution provides for a presidential system of separation of power where members of Congress cannot serve in the executive branch; whereas in a parliamentary system,
		a prime minister must also be a member of parliament (MP),
		whose government,
		consisting also of MPs,
		retains the confidence of a majority of the MPs.
		Perhaps because the rewards at the disposal of party leadership in Congress appear less attractive than those offered by their counterparts in other parliaments,
		congresspeople may feel that they have less to gain from voting with their copartisans than parliamentarians in general
		\autocite{cainPersonalVoteConstituency1987}.
		Furthermore,
		the candidate-centered electoral system used for returning congresspeople encourages them to pursue more personal votes which are more valuable for legislators competing in party-centered electoral systems.
		These findings have similarly been corroborated across a varieties of institutional settings
		\autocite{careyIncentivesCultivatePersonal1995}.
		%		<!-- universal occurrence of pursuit of personal vote  --> 
		That is not to say,
		however,
		that legislators contesting under party-based electoral systems do nothing to enhance their electoral independence from the party.
		Even in some of the most party-based electoral systems,
		efforts have been devoted to shield candidates from the uncertain electoral fate of the party by developing their personal brands,
		sometimes at the behest of party leadership
		\autocite{shugartLookingLocalsVoter2005},
		suggesting that activities geared toward cultivating a personal vote is quite universal,
		even though the intensity of such pursuit varies.
		
		
		To specify how legislators are incentivized by the threat of recall to devote more time and resources toward cultivating personal votes,
		I would like to point to another observation about recall devices to facilitate analysis.
		To date,
		no mechanisms provides for recalls of individuals elected on the basis of a party list.
		%		<!-- a collectivity,
		%		e.g.
		%		a party.
		%		Voters are only allowed to recall individual elected officials in Taiwan as is the case for all other jurisdictions where recalls are allowed.
		%		 --> 
		Even though such a mechanism is technically conceivable that parties of individual legislators elected on the basis of party list,
		e.g.,
		closed-list proportional representation (CLPR),
		open-list proportional representation (OLPR),
		can be recalled by the electorate,
		such practice,
		to my knowledge,
		has never existed,
		possibly because it defeats the point of instituting a party list electoral system where
		parties are the vehicles of representation as the name of the category suggests and thus
		given the final say as to which members of their parliamentary parties get to keep their job.
		This means that,
		realistically,
		only legislators chosen on the basis of candidate-centered electoral systems,
		e.g.
		first-past-the-post (FPTP),
		single transferable vote (STV),
		single nontransferable vote (SNTV),
		alternative vote (AV),
		can ever be recalled since voters technically return individuals instead of parties to the elected office,
		and are thus given the opportunity to recall them.
		Therefore,
		I believe that it is safe to assume that only legislators chosen on the basis on candidate-centered electoral systems can ever be recalled.
		%		<!-- Of course it is certainly possible that,
		%		due to my omission or some future institutional innovation,
		%		a legislator elected on the basis of the party list may be subject to recall attempts as well.
		%		-->
		%		<!-- As the foregoing discussion suggests,
		%		 a party or legislators elected from the party list devote significant efforts to court personal votes and would not be immune to pressure to redouble their efforts should they be subject to recalls.
		%		 One simply need to reconsider to whom those activities should be attributed as both party leadership and individual candidates on the party list might be responsible.
		%		 -->
		
		On the assumption that legislators facing the threat of recall contest in candidate-centered electoral systems,
		I now proceed to sketch the conditions incentivizing legislators to devote more efforts to cultivate a personal vote in response to the latent threat of recall.
		\citeauthor{cainPersonalVoteConstituency1987}
		\autocite*{cainPersonalVoteConstituency1987}
		suggest that seniority and electoral uncertainty are key predictors of personal vote.
		Considering the personal vote as a stockpile of legislators accumulate over time and expend to counter electoral uncertainty,
		\citeauthor{cainPersonalVoteConstituency1987}
		\autocite*{cainPersonalVoteConstituency1987}
		argue that legislators' demand for personal vote demand for personal vote decline in both seniority and electoral uncertainty.
		Later studies have 
		%		<!-- not only  -->
		corroborated this key finding
		\autocite{hibbingCongressionalCareersContours1991,ansolabehereOldVotersNew2000}.
		%		<!-- but also specified elements contributing to electoral uncertainty, e.g. party dealignment [], constituency complexity [@ensley2009], coattail effects [], as well as how they gain personal votes through various positions of power, e.g. (sub)committee appointment, government portfolio, thereby decreasing their demand for more personal votes and thus their activities to that end.  -->
		As recall constitutes an electoral uncertainty, it follows that similar pattern should be observed among legislators who can be recalled.
		
		As for the activities pursued for the purpose of cultivating personal votes,
		legislators have a number of options which can be categorized into position taking,
		%		<!-- (which is usually ideological) -->
		and particularistic benefit provision.
		%		<!-- (which is usually nonideologlical) -->
		Members of the US Congress have a long history of enegaging in both kinds of activities to cultivate personal votes.
		However,
		taking an ideological position different from that of the rest of the party is a luxury for legislators who are more incentivized to toe the party line
		\autocite{kamPartyDiscipline2014}.
		\citeauthor{mayhewCongressElectoralConnection1974}
		\autocite*{mayhewCongressElectoralConnection1974}
		argues that a legislator's demand for personal vote should also decline in the amount of campaign resources controlled by the parties.
		Empirical evidence from US Congress and other legislative assemblies,
		especially those operating under parliamentary systems bears out this theory
		\autocite{rohdePartiesLeadersPostreform1991,coxLegislativeLeviathanParty2007,careyLegislativeVotingAccountability2008,depauwLegislativePartyDiscipline2009}.
		Nonetheless,
		party strength should not offset legislators' demand for personal vote as a result of the latent threat of recall.
		Keep in mind that even perfect partisan coordination and mobilization of voters is not foolproof against electoral swing against the incumbents,
		especially for the less experienced ones or those in marginal constituencies.
		Hence,
		one should still expect to observe a net increase in the amount of activities pursued for the purpose of cultivating personal votes.
		Under this condition,
		copartisans,
		desiring the same,
		if not lower,
		level of electoral uncertainty for themselves,
		must be willing to reduce some aspects of their intraparty cohesion.
		Drawing inspiration from Martin and Vanberg's
		\autocite*{martinCoalitionGovernmentLegislative2020}
		arguments with regard to bargaining among the coalition partners,
		I also contend that the activities legislator undertake to pursue personal votes as a range of goods which are not equally "visible" among the whole electorate and thus make for good side payments that parties use to maintain their solidarity platform-wise
		\autocite
		{coxSettingAgendaResponsible2005,jenkinsBuyingNegativeAgenda2012}.
		Some goods,
		such as records of roll call vote and Minister's Question Time are highly visible and potentially damaging to a party's national reputation and thus the party vote across the board; other goods,
		such as constituency service and pork barrels outside the legislator's constituency and thus far less of an immediate concern for the party as a whole.
		Naturally,
		individual legislators may as well desire both the highly visible and the poorly visible goods,
		depending on their capacity for collective action.
		Since the party as a whole benefit from not demanding so many highly invisible goods for each individual legislator,
		the demand for highly visible goods would decline in the party's capacity for engaging in collective action.
		Following these arguments,
		I hypothesize that
		\textbf
		{
			\textit
			{
				credible threat of recall increases legislators' demand for personal votes.
				And when parties are endowed more with more resources,
				personal votes will be more likely to pursued in a fashion that is less visible among the national electorate.
			}
		}
		
		



		Ideally, it would be great to develop a theory that applies to all mechanisms of recall in existence.
		However, such a theory would have to contend with the immense diversity of 
		institutional arrangements for recalls, which
		almost rivals that of electoral system.
		In addition to individual elected officials,
		an entire legislative or executive branch is also subject to recall
		\autocite{welpRecallReferendumWorld2020},
		which in this case is effectively an early general election called by constituents.
		If the literature of electoral system is any indication, it would be wise not to impute the
		incentive effect,
		if there is any,
		on any recall election without understanding how 
		its components,
		on their own and interated with each other or broader institutitional settings,
		may causes a politician's incentive structure to change.
		
		Therefore, I will spcify an institutional arrangement of recall that serves several purposes.
		This institutional arrangment will,
		above all else,
		seek to capture the essential elements of
		how Taiwan, where I shall draw my empirical evidence,
		provides for recall of the incumbents legislators in its Legislative Yuan,
		which helps to improve this study's internal validity.
		In addition, the institutional arrangement I propose will seek to
		strive to separate the effect of each component of recall,
		thereby paving the path for a more nuanced understanding of how these components may interact with each other.
		Lastly,
		I will attempt to convince the reader that
%		far from bogged down with the particularities of recall insitutiton in Taiwan,
		this institutional arrangement I sketch is general enough to
		serve as a good jumpboard for understanding recall institutions around the world such that
		even in certain cases where the institutional setup I propose does not resemble the recall instituion under examination in certain important aspects,
		only a few modifications will suffice to model the points of divergence.
%		Therefore, to buttress the externality validity of the finding,
%		I will specify a tentative boundary of its scope condition
%		that will include the recall mechanism in Taiwan which is under examination,
%		and propose a prototypical recall mechanism
%		that not only strips away the less relevant features of recalls
%		but also elucidates the effect of each feature individually.
%		This prototype will serve as a starting point for building more sophisticated models of
%		the incentive effects of recall elections,
%		which may be adjusted in the future as long as
%		the component of recall mechanism under study justifies the adjustment.
		
		\subsection*{Recall of Individuals versus Recall of Collectivities}
		For this article, I shall direct all my attention to recall that remove individual.
		Although it is easy to justify this decision on the sole basis that
		recall of a collectivity,
		say a legislative or executive council,
		is already rarely around the world and have not been activated for nearly a century,
		I shall take the extra step of explaining
		how recall of individuals and collectivities
		introduce qualitatively different incentive effects
		such that recall of collectivity deserves its own treatment which cannot be answered here.

%		To take the first step toward specifying this scope condition,
		To being with,
		I shall retrace my step to the central puzzle of understanding why
		recall appear to have no effect on,
		or, worse, undermine the accountability of politicians to their constituents.
		Though
%		Gordon and Yntiso
		\citeauthor{gordonIncentiveEffectsRecall2021a}
		\autocite*{gordonIncentiveEffectsRecall2021a}
%		citation needed
		demonstrate that elected judges do respond to constituent pressure,
		the resultant behavioral change may not be consistent with consituency preferences .
		it does not sufficiently solve the puzzle as
		it does not tell us how politicians respond to partisan pressure in addition to constituent pressure.
		Thus,
		understandning how recall election works for partisan offices helps to examine how
		the crosswinds coming from the both the party and the consittuency,
		which in turn
		could help unravel this central puzzle
		in addition to the other goal of
		determining the
		the scope condition of
		Gordon and Yntiso's
		finding.
		
		However,
		in a setting where only
		the entire
		legislative or executive council may be recalled,
		it is difficult to argue that it directly alters the incentives of legislators.
		Should a recall of the entire legislature
		be activated,
		it is most likely to be triggered by
		the broader political conditions.
		This is potentially problematic as
		Mayhew
		\autocite*{mayhewCongressElectoralConnection1974}
		points out that since
		it is practically impossible for the vast majority of
		individual legislators,
		particularly those who are not charged with crafting party platforms or national policies,
		to alter the broader political conditions,
		it is better for them to treat the broader political conditions,
		those that can cause the entire legislature to be recalled included,
		as acts of God and do nothing about them.
		Hence, it stands to reason that a recall of the entire legislative or executive council
		provides no direct incentive for the vast majority individual legislative or executive council member
		to alter their behavior,
		justifying the decision to exclude this kind of recall from the finding's applicable scope for theoretical reasons.

		While legislators may not be directly motivated by the recall of the entire legislature,
		their parties may be, introducing a different issue as to predicting the recall's incentive effect.
		Suppose parties, especially those in government, do feel that the threat of recall of the entire legislature,
		they very well may attempt to influence their legislators' behavior
		as parties
		are far more capable of changing the broader political condition
		which is the main source of the threat of recall of an entire legislature and thus better at
		protecting themselves from such threat.
		What parties may do, however,
		to bolster the electoral security is more difficult to predict as it is far from clear whether giving members
		more leeway to attract voters with their own message does more good than harm to the parties.
		Therefore, as the recall of a collectivity is rare and its incentive effect hard to deduce theoretically,
		I will not consider it in my discussion of recall unless it is specifically referenced.
		Then there remains only the recall of individuals within the scope of discussion.
		Unless, otherwise indicated, recall will hereinafter simply be synonymous with recall of indviduals.
		
		\subsection*{Recall Election and Electoral Systems} 
		Having specified the scope of recall under consideration,
		I shall proceed to describe the broader institutional arrangements associated with recall within the scope of study.
		As it turns out,
		one of
		the most pertinent arrangement
		- electoral system -
		is actually tightly connected to the scope.
		Recall is usually justified as a way for people to retract their approval of incumbent officials,
		whose approval is formally granted through electoral system.
		Many electoral system differ as to who gets the approval
		and
		may be thus divided into categories where
		one category gives voter the right to elect an individual to an office
		and
		another grants the electorate the right to send a group of politicians to
		a number of positions at a time, usually in the legislature,
		typically by the strength of their parties in the election;
%		and the third category, usually employed in legislative elections,
%		creates two tiers elected positions, where
%		one tier, usually called nominal tier, consists of individual politicians elected based on the votes they win individually
%		and the other, often referred to as party tier, politicians elected from a party list based on the votes their parties win.
%		The first category is called candidate-based electoral system whereas the second one party based electoral system and the third one mixed-member electoral system.
		It wouldn't take long to see that recalling individual incumbents who take their office based on a party list is hard to justify
		as it contradicts the purpose of insitituting a party list, i.e. giving the party,
		instead of individual legsilators,
		the direct responsibility for, and thus the control over, policies and personnel in the legislature.
		Some make the even bolder claim that recall of individuals is incompatible with party-based electoral systems.
		For these reasons, it is safe to assume that only incubents elected under an candidate-based electoral system may be recalled.
		
		\subsection*{A Prototypical Recall}
		Now that the major groundworks specifying the institutional arrangements is in place,
		I shall first put together a stylized institutional arrangement of recall,
		or a prototypical recall, so to speak,
		and then justify the decision to include or omit certain components.
		As the word ``stylized" suggests, this prototype neccessarily differs from the actual recall device in place around the world,
		including that found in Taiwan.
		However, as with most attempts to build a model of the real world,
		it strives to elucidate the effect of the relevant components of the institution.
		Toward that end, I shall justify my sketch of the prototype's components
		with both their prevalence around the world  
		and their purpose of elucidating the role in structuring the incentives of incumbents.
		
		In this stylized arrangement, a recall include at most four events.
		\begin{enumerate}
			\item 
			Announcement of the intention to circulate recall petition;
			\item 
			(Dis)approval of the recall petition;
			\item 
			Holding the recall election;
			\item
			Holding the special election to fill the vacancy created by the recall.
		\end{enumerate}
		Not all events would take place in a recall attempt except the first two.
		Each intervening period between the consecutive events in this sequence
		lasts for a statutorily predetermined amount of time.
		In this prototype,
		recall cannot begin unless the organizers file their intention to circulate recall petition
		with the relevant electoral agency
		which grants ministerial approval of circulation as long as the application is in order.
		After the intention is filed and announced,
		unless
		the organizers are gather signatures in the incumbent's constituency
		numbering above a statutorily determined threshold on their recall petition
		within a statutorily determined amount of time,
		the recall petition will not be approved.
		If and when the petition is approved,
		a recall campaign period will follow before the recall election takes place.
		Should the recall be successful,
		a final special election campaign will be scheduled on a date after the recall election,
		allowing people to file candidacy and campaign for the vacated position.
		
		The first rationale for prototype is derived from the prevalence
		of the four events and how it completes the whole process of recall.
		Any recall attempt would begin with petitions which must gain enough support to merit a recall election.
		And when the recall election is successful, the vacancy must be filled in some way.
		Although the details of their institutional arrangements differ from one jurisdiction to another,
		this underlying structure remains the same across the board and does not leave the position vacant.
		
		The second rationale for this prototype
		is that each potential stage of a recall attempt,
		once set in motion by filing the intention to circulate recall petition,
		must complete within a constant time frame.
		Such rigidity of each intervening time period greatly simplifies the game structure
		as
		it does not permit to any actors involved in this game to
		act strategically play with the timing of the last three events
		therby simplifying the game structure.
		Hence, only the strategic part about the timing is the organizer's timing of when to initiate a recall attempt.
		
%		The third rationale for this prototype is that it disincentivizes the actors to plan strategically ahead
%		when by introducing nature as a player in every stage of the game.
%		Therefore, when the outcome becomes uncertain at any stage, the dominant strategy is to
%		anticipate success instead of failure for each actor.
%		This allows us to simplify the goal of each actor in each stage of the recall.
		
		The third rationale for this prototype is that it eliminates a number of arrangements
		that are extraneous to the analysis of
		the incentive effects of recall.
		A good example of such kind of arrangment can be found in the test case Taiwan where an organizer 
		must gather enough signatures for a \textit{preliminary} petition for recall
		before they can gather signatures for the actual petition for recall.
		Essentially, both preliminary and actual petition requirements
		fulfill the same purpose of demanding a minimal level of support of recall to justify the recall election
		and thus are sufficiently encapsulated by the prototype where only one petition is needed.
		Another feature, like that of requiring the organizer to present their reason for triggering a recall
		and allowing the incumbent targeted by the recall attempt to respond,
		poses such negligible obstacle to any potential recall organizers that it 
		hardly warrants any consideration and is thus not included in the prototype.
		Though these omissions of certain nuanced differences can be a strenght or weakness of considering a stylized institutional arrangement
		depends on who you ask,
		it bears repeating that it need not be the only way of analyzing recall institution
		and should always be justified when necessary.
		
		The fourth rationale for this prototype is that it explicitly makes the goal of each
		political actor clear and simple.
		Students of recall may notice that the prototype is different from how California regulates gubernatorial recall
		where each voter decide on one single ballot
		whether the governor should be recalled
		and,
		in case the recall is successful,
		who should become the new governor.
		This creates a problem where both proponents and opponents of recall may want to present their own alternative to the incumbent governor
		which is a minor issue for the organizer and the party/ies in opposition
		but a substantial issue for the party of the incumbent which is torn between
		whether it should oppose a recall altogether or accept recall as a distinct possibility and present a backup candidate
		potentially causing a rift between the party, the incumbent and the backup candidate.
		It is difficult to predict which option may prevail in such circumstances.
		Though the same Democratic Party had to face off gubernatorial recall elections twice in the 21st century,
		and attempted to throw its full weight behind the incumbent in both times,
		the attempt succeded in 2021
		but failed in 2003
		when
		Lieutenant Governor Cruz Miguel Bustamante put his hat in the ring as a viable backup candidate
		despite the warning from his fellow Democrats.
		Here lies the benefit of the prototype which not only closely resembles the test case in Taiwan
		but also make the goal of each political actor much clearer,
		further clarifying the game structure provided by the recall device.

		The fifth reationaale for this protype is that it maximizes the control of local constituents of the outcome of the recall.
		For many local recalls, such as the recall of school board members in San Francisco,
		the vacancy is filled by Mayor of San Francisco instead of San Francisco voters themselves,
		this not only introduces additional political actors in the game triggered by the recall
		but also takes away the constituents' role of actually deciding who should replace the recalled incumbents
		which is arguably the most important role consituents are supposed to play.
		For both normative and analytical considerations, such ways of filling the vacancy caused by the recall are not considered.
		
		
	
		\subsection*{Theory of Personal Vote}
		
		[
		Here's a sketch:
		Three factors determine a large part of the voting behavior:
		party identification,
		candidate evaluation,
		policy orientation.
		To simplify analysis, we only consider party identification and candidate evluation
		and merge the effect of respective policy orientation of parties and candidates to the above two factors,
		notwithstanding debates about party id and ideology.
		
		Assume incumbents strictly prefer staying in office without going through recall election
		to staying in office after surviving recall election,
		which is in turn preferred to the worst outcome, not surviving the recall election.
		
		Credibie recall threat requires:
		\begin{enumerate}
			\item 
			Sufficient support for recall election:
			$x\%$ of constituents agree
			\item 
			Sufficient support for recalling the incumbent
			$y\%$ of constituents agree
		\end{enumerate}
		If we treat both the petition circulation and recall as an election,
		an incumbent is essentially a candidate for election
		and the candidate would seek to enlarge the personal vote.
		Because the party vote is unalterable
		
		
		
		
%		Let us think this through.
%		If voters are motivated by purely strategic consideration,
%		then the number of signatures required to pass the petition wouldn't matter.
%		Because voters will only sign the petition for a recall if there is a realistic chance of recalling the incumbent.
%		Then any number of signature on the petition for recall is a surefire way of signaling that the recall has a good chance of succeding,
%		thus meriting a recall election.
%		However, that is obviously not the case.
%		The number of failed recall petitions,
%		combined with the number of recall elctions that fail to remove the incumbent,
%		suggest that many people vote not because there is a realistic chance of recalling an incumbent.
%		It has a more about personal opinion expression and identitification with a performative norm.
		
		Let us attack this problem from a different angle.
		
		\subsection*{Recall allows voter to remove incumbent earlier}
		Suppose people vote for one purpose,
		which is either party identification or policy orientation.
		A party may wishes to pass a policy
		that deviates too far from from the aggregate position of the constituency
		and requires the incumbent to toe the party line.
		When voters cannot resort to recall to punish the incumbent for
		such kind of transgression,
		the party could choose to make the incumbent enact the more radical policy
		in the early phase of the incumbent's term
		and then attempt to shore up votes for the incumbent
		by passing policy favored by the constituency.
		The constituency, attaching greater importance to the more recent policies,
		would discount the more distant policy that deviates from the constituency median voter's position,
		hence allowing the incumbent to keep the job and the party to enact the policy.
		
		However, when recall comes into play,
		the constituency could choose to
		deliver their judgement immediately by recalling the incumbent,
		thus preventing party from shoring up support for incumbents at a later time
		in order to help the incumbent get reelected.
		The incumbent, knowing
		that the party cannot be relied on to shore up support at the general election later
		as the voters' judgment comes sooner than that, will be less willing to support the party's more radical policy.
		The party thus has three choices,
		\begin{enumerate}
			\item 
			Refrain from enacting policy too radical.
			\item 
			Enact policy and shiled the incumbents from recall by giving them even more side payments.
			\item 
			Enact policy and risks recall of their members.
		\end{enumerate}
		The third choice is the worst, hence, one shall observe more personal vote than before.
		Personal vote is operative before, but through the recency effect
		one may pass radical bill without jeopardizing the reelection prospect.
		However, this will no longer be the case.
		
		\subsection*{Recall elections make relatively parties stronger, and thus incumbents avoid them at all costs}
		
		\subsection*{Recall sends credible signal that incumbent popularity is waning}
		
		Yes recall may strengthen the position of the incumbent if the outcome is extremely
		favorable to the incumbent.
		However, it remains way too great a 
		
		
		\subsubsection*{Recall depletes the resource of the incumbent that could be spent elsewhere}
%		Hence, we assume that there are
%		$l\%$
%		of people who vote for any candidate belonging to party $L$ regardless of the candidate's policy orietntation,
%		and
%		$r\%$ of people who vote for any candidate belonging to party $R$ regardless of the candidate's policy orietntation as well.
%		Then there are $(100 - l - r)\%$ of people who vote based on the distance between their ideal policy position and that of the candidate.
%		
%		Assume that the policy space is unidimensional.
%		Assume also that there are only two candidates in the previous election.
%		Suppose in the previous election the incumbent $I$ who belongs to party $L$ won
%		$(l + o_l)\%$ of votes
%		and the other candidate that belongs to party $R$ won
%		$(l + o_l)\%$.
%		Obviously,
%		$l + p_l > 50$.
%		
%		Let the position of voter $V$ be $P_V$.
%		
%		The two bills that party $L$ passes occupy position $P_{B1}$ and $P_{B2}$
%		such that $L = P_{B1} < P_{B2}$
%		
%		Allow incumbent $I$ to take two positions $P_{I1}, P_{I2}$ in a sequence.
%		
%		At the end of period 1, the utility of $V$ is
%		$-(P_{I1}- P_V)^2$.
%		
%		I am not sure anyone has done this but
%		I think there is an intuition that  both voters and the parties care more about the incumbent's positions
%		through out the term, but the more recent a position is,
%		the greater its importance is to the voter.
%		I encapsulate this intuition with recency factors for voter, party $L$ and party $R$
%		which are respectively $\rho_V$, $\rho_L$ and $\rho_R$.
%		
%		The utility for voter at the end of period 2 is
%		$-\rho_V (P_{I1}- P_V)^2 - (P_{I2}- P_V)^2$.
%		
%		The utility for party at the end of period 2 is
%		$-\rho_L (P_{B1}- P_L)^2 - (P_{B2}- P_L)^2 = - (P_{B2}- P_L)^2$.
%		
%		If $\rho_L$ is high and $\rho_V$ is low,
%		then party $L$ could propose their ideal policy $B1$ and then propose policy $B2$
%		in the hope of appeasing voters.
		
		
		
		
		
		
		
		
%		Assume further that the position of the $L$ party platform $P_L$ is to the left of incumbent $P_C$
%		and to implement the party platform, the party pushes the policy position of $P_C$ further left,
%		causing the $o_l$ to decrease and $o_r$ to increase,
%		then whenever the $P_C$ decreases to the extent that
%		$l + o_l < 50$,
%		the candidate would be immediately recalled.
%		Party $L$ has three options.
%		\begin{enumerate}
%			\item 
%			Party $L$ accepts that incumbent $P_I$ can not decrease to the level of $P_L$
%			either because $I$ does not wish to do so or $I$.
%			\item 
%			Party $L$ forces the incumbent to decrease $P_I$,
%			causing $o_l$ to decrease and $o_r$ to increase,
%			which triggers a recall election that removes incumbents $I$ and installs the opponent
%			$J$
%			\item 
%			
%		\end{enumerate}
		
		
		
%		Hence
%		The candidate
%		
%		
%		then credible threat of recall will emerge
%		whenever the support rate falls below a certain level, say 50 or 45 percent.
%		This 
%		
%		
%		Recall paradoxically improves the candidate's standing
%		
%		Yeah, I don't think the an incumbent would like that kind of risk
%		
%		Simply holding the recall carries a great risk of 
%		damaging the incumbent's propsect  of reelection either now or in the future.
%		The risk is sufficiently large that
%		Incumbents would seek to avoid a recall election altogether.
%		
%		However, fulfilling the threshold for holding a recall reelection
%		requires much fewer constituents than that
%		required to defeat the incumbent.
%		It seriously incentivizes the incumbent to cultivate a personal vote to ward off such attempt.
%		
%		To ward off such attempt:
%		politicians have an understanding of why people petition for a recall election
%		\begin{enumerate}
%			\item 
%			People who demand a recall regardless of whether it will succeed
%			\item 
%			People who demand a recall because it may succeed
%		\end{enumerate}
%		People motivated by the first cause are found everywhere
%		People motivated by the second cause are more likely to be found when the margin decreases
%		
%		If the incumbent depends on party vote alone, 
%		
%		the reputation of the incumbent,
%		whatever may happen
%		]	

	

%		<!-- To demonstrate this view,
%		 I propose to apply the technique of *reductio ad absurdium* -->
%		  <!-- first attempt to argue that the personal vote may be stripped of its position-taking component and considered as nothing but voters' response to elected officials' non-ideological activities,
%		 which,
%		 in the case of legislators,
%		 include constituency service and pork barrel.
%		 Although it is useful to consider the contribution of this aspect of politicians' activities to their electoral support,
%		 it is --> <!-- In some conceptualization where the party vote is always a constant,
%		 the candidate's job becomes even simpler as only one task remains: maximizing the personal vote under resource constraints.
%		 Such definitions and their associated predictions appear to conflict with common sense and political folklore,
%		 which,
%		 as @cain1987 [p.
%		 123] have warned us,
%		 is a cause for pause.
%		  --> 
%		  <!-- Should there be no tradeoff between a politician's attempt to cultivate a party vote and a personal vote,
%		 then  --> 
%		 <!-- the most progressive Democratic candidate could win more Republican votes than a Democrat in name only who is as conservative as any Republican simply because the former is better at delivering constituency service and pork.
%		 --> 
%		 <!-- resource spent on personal vote is not spent on party affairs --> 
	
		
	\newpage
	\printbibliography
	
	
\end{document}

