% !TeX root = sketch_of_formal_model.tex

\documentclass[hyphens, crop=false]{standalone}

\usepackage[
reflist=true,
%noreprint, 
natbib=true,
autocite=inline,
style=windycity]
{biblatex}


\usepackage[american]{babel}
\usepackage{csquotes}

\usepackage{url}


\usepackage[multiple]{footmisc}

\usepackage{amssymb}


\addbibresource{lit_review.bib}



\begin{document}
	\begin{enumerate}
		\item 
		Settings
		\begin{enumerate}
			\item 
			In the legislature $M$, there are $|M|  = m$ members, each of which returned from a single-member district (SMD) with the first-past-the-post (FPTP) electoral system.
			\item 
			There are two parties $A$ and $B$
			such that $A$ and $B$ partition the enitre electorate $E$.
			\item 
			The legislative parties of $A$ and $B$
			are respectively labeled $M_A$ and $M_B$ and $M_A$ and $M_B$ partition $M$.
			\item 
			Assume without loss of generality that $|M_A| \equiv m_A > m_B\equiv |M_B|$. Hence
			$M_A$ is the party in government.
			\item 
			Let the policy space be $\mathcal{P} = \mathbb{R}$
		\end{enumerate}
		\item 
		Hypothesis/Observation/Assumption:
		\begin{enumerate}
			\item 
			Incumbents single-mindedly pursue reelection.
			\item 
			Party $A$ seeks to implement the policy vector $\textbf{p} = [p_1, \dots, p_i, \dots, p_n]^T\subset \mathcal{P}^n$.
			\item 
			Voters are purely retrospective. 
			\item 
			Constituents represented by $M_A$
			evaluate their legislators performance
			based solely on the party's policies.
			\item 
			Let there by $n$ periods of policy making between two consecutive elections, then the utility of all the implemented policies $P$ for an individual whose recency bias factor is $\rho \in (0, 1)$ and ideal point $c\in \mathcal{P}$ is
			$$
			\sum_{i = 1}^{n} -\rho^{n-i}( p_{(i)}-c)^2.
			$$
			where $p_{(i)}$ is a sequence that rearranges the order of policies in $\mathbf{p}$.
			\item 
			Further assume that incumbent $j\in M_A$ will be reelected if and only if 
			$$
			u_{j, n} \equiv \sum_{i = 1}^{n} -{\rho_j}^{n-i}( p_{(i)}-c)^2 \geq t_j 
			$$
			where 
%			\\
			$u_{j,n}$
			is the utility function of the pivotal voter in the consitituency of $j$,
%			\\
			$\rho_j$ the recency bias factor of the pivotal voter in $j$,
			and
			$t_j$
			the reelection threshold for pivotal voter $j$. The reelection threshold is a sort of catch-all value that depends on the platform of party $B$.
%			$$
%			t_j \equiv \sum_{i = 1}^{n} -{\rho_j}^{n-i}( p_{i.j}-c)^2 
%			$$
		\end{enumerate}
		
		\item 
		Predicted outcome when recall is not allowed:\\
		Party $A$ would
		calculate the utility of median voter in each incumbent's consituency
		for each permutation of the policies and then
		maximize the number of incumbents that can get reelected.\\
		It is assumed here that all members of $M_A$ are reelected with the current sequence of policy
		
		\item 
		Predicted outcome when recall is allowed:\\
		Recall of $j$ occurs at period $r$
		if and only if
		there exists a set $R\subset \{1,\dots n\}$
		such that for all $k_1 \in R$
		$$
		u_{j, k_1} \equiv \sum_{i = 1}^{k} -{\rho_j}^{n-i}( p_{i.j}-c)^2 < t_j 
		$$
		and for all $k_2 \in \{1,\dots n\}\setminus R$
		$$
		u_{j, k_2} \equiv \sum_{i = 1}^{k} -{\rho_j}^{n-i}( p_{i.j}-c)^2 \geq t_j 
		$$
		and $r = \min(R)$.
		\\
		Party $A$ may seek to re-sequence the policy enactment order.
		However, if there are extreme policies in $P$, it will not work at all,
		potentially costing the party the required majority.
		\item 
		Consequence: \\
		When recall mechanism is introduced,
		party are much more likely to do something to maintain the current majority in addition to playing with the sequence of policy enactment.
		Options are: 
		\begin{enumerate}
			\item 
			making certain policies less radical.
			\item 
			giving members of $M_A$ more time and resources to cultivate a stronger personal vote through non-policy means like constituency service.
			\item 
			allowing members of $M_A$ to take a position different from party platform.
		\end{enumerate}
	\end{enumerate}



	
\end{document}

