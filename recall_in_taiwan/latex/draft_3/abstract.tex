% !TeX root = personal_vote_definition.tex

\documentclass[hyphens, crop=false]{standalone}

\usepackage[
reflist=true,
%noreprint, 
natbib=true,
autocite=inline,
style=windycity]
{biblatex}


\usepackage[american]{babel}
\usepackage{csquotes}

\usepackage{url}


\usepackage[multiple]{footmisc}

\usepackage{setspace}
\doublespacing



\addbibresource{lit_review.bib}


\begin{document}
	
	Recall is designed to hold elected officials more accountable,
	but the extant literature reports mixed evidence that this goal is achieved,
	espeically for incumbents operating in an electoral environment where parties may distort the effect of recall.
	I shall address this puzzle
	by outlining a mechanism where both incumbents and their party would
	agree to help incumbents cultivate a stronger personal vote and thereby holding them more accountable to their constituencies.
	I shall test this mechanism by drawing evidence from a natural experiment in Taiwan where
	electoral and recall institution of its legislature provide for some of the strongest incentives for cultivating a personal vote,
	making Taiwan a great baseline case for this mechanism.
	Establishing the validity of this mechanism in the baseline case also helps
	explain why recall devices elsewhere does not always hold politicians more accountable
	by examining where they differ from institutional arrangements of Taiwan's legislative recall.
	
%	\newpage
%	\printbibliography
	
	
\end{document}

