% !TeX root = recency_bias.tex

\documentclass[hyphens, crop=false]{standalone}

\usepackage[
reflist=true,
%noreprint, 
natbib=true,
autocite=inline,
style=windycity]
{biblatex}


\usepackage[american]{babel}
\usepackage{csquotes}

\usepackage{url}


\usepackage[multiple]{footmisc}

\usepackage{setspace}
\doublespacing



%\addbibresource{lit_review.bib}


\begin{document}
	
	
	
	I am applying to the PhD program at %name of university
	to study how democratic institutions induce politicians' behavior,
	and I believe that understanding how parties work
	is key to achieveing this end.
	The centrality of parties is forcefully
	demonstrated by the fact that anti-party institutional reforms
	that seek to achieve this end 
	by making partisan labels less meaningful to voters
	frequently fail.
	Instead of creating demand among
	low-information voters for more information about candidates,
	anti-party institutional reforms simply create demand for more meaningful partisan cues
	which political entrepreneurs are often happy to supply
	\textit{despite} the institutional constraints,
	thus defeating the purpose of the reform.
	Nevertheless,
	I do believe that anti-party institutional reforms
	create great opportunities for answering
	these questions:
	how much do voters' decisions at the poll rely on partisan cues,
	how do these cues' interaction with other cues affect those decisions,
	and how do parties respond to voters' demand for cues 
	and institutional constraints?
	Answering these questions will help provide better and more realistic
	understanding and prediction of how democratic institutions incentivize politicians.
	
	
	
	Toward that end,
	I intend to focus my research on
	anti-party institutional reforms like
	direct democracy and electoral reforms
	to study how they affect the legislative behavior
	in order
	to speak to the longstanding tradition
	in the political science
	that parties emerge in large part
	due to the needs created by
	the legislative policymaking process.
	These experiments with democratic institutions
	and often found in comparative context
	and in American states
	where institutional changes are more achievable,
	which is why my primary concentration is in
	Comparative Politics
	and my secondary concentration is in
	American Politics.
	
	While retaining interests in all of the institutional reforms mentioned above.
	I seek to contribute to improve
	the understanding of these reforms
	by focusing on one of them that receives relatively little attention
	up until now: recall election.
	Recall gives us a great insight into
	how parties seek to respond to institutional changes
	where incumbents have a good reason
	to behave in a less partisan fashion
	in order to survive a recall
	where the incumbents for most part
	can only stand on their own records.
	In the past,
	recall receives little attention as they occur so infrequently
	and most interest in this institution is generated by
	\textit{one} recall that captures the attention of the nation:
	the 2003 California gubernatorial recall
	that saw
	action movie star Arnold Schwarzenegger defeat
	both the incumbent Governor Davis Grey and Lieutenant Governor Cruz Bustamente.
	Recall, however,
	is experiencing a revival these days.
	It is not just Californians who again resort to recalls
	to throw out politicians from their office.
	On the other side of the Pacific Ocean,
	people of Taiwan are also reasserting their right to recall,
	which was inspired by the same Progressive Movement that brought recall to California and other US states,
	and put into their constitution around the same time,
	but never fully exercised until new laws were passed in 2016 that made recall easier.
	As recall becomes more frequent in California and Taiwan,
	it allows us not only to receive more quantitative data for a micro-level studies,
	but also to qualitatively compare recall in different party systems
	as California is consistently dominated by Democratic Party
	while Taiwan has a more balanced two-party system at the national level.
	My fluency in English and Mandarin Chinese,
	puts me in a great vantage point both to dive into recall at micro-level
	and to compare recall institutions across the world,
	and ultimately to put forward a theory with great external validity.
	
	To study how recall changes the incentive structures of parties
	and ultimately those of individual politicians,
	I proposed a theory in my Master's Thesis
	which is included in the writing sample,
	that the governing party in the legislature
	engineers an electoral cycle of legislation akin to political business cycle
	but introduction of recall damps
	the cycle and incentivized the governing party to 
	and encourage its members to cultivate a stronger personal vote
	thus better aligning their behavior with aggregate preferences of their constituencies.
	Simply put,
	elections held at regular interval make voters focus more
	on the governing party's performance in the run-up to the election
	due to their \textit{recency bias},
	allowing the governing party in the legislature to implement extreme policies
	early in the legislative term and to evade punishment for extreme policies
	by moderate policies toward the end of the term,
	which are more important to voter choice.
	Recall on the other hand
	manages to flip the recency bias on its head
	by allowing voters to threaten credibly to remove the legislators
	right after they pass extreme policies
	thus undermining the governing party's
	ability to make policies in the legislature.
	I have also proposed several empirical strategies
	to measure legislative behavior with 
	for testing the theory,
	which I intend to execute during my PhD
	program.
	Furthermore,
	I predict that recall
	also forces legislators to take a different policy position
	from the party
	not only through conventional methods like roll call votes
	and bill sponsorship
	but also through speech in the legislature and on social media,
	which can be systematically analyzed through NLP.
%	I can also say something about affective polarization.
	With my training in Statistics and Machine Learning,
	I believe that I am up to this task.
	
	Finally,
	I believe that in addition to studying the behavior of
	legislative parties and their members,
	it is also important to study how elites organize recall
	events and how they decide when to use partisan cues.
	This is driven in particular by the puzzle that
	organizers and campaigners in favor of recall
	take pain to avoid using explicit partisan cues
	to mobilize voters.
	It may make sense in a lot of recalls
	in California where most constituents
	are Democratic and organizers may seek to rile up
	division within Democratic political elites and voters
	to secure a majority of constituents in favor of recall;
	however, it makes much less sense
	in many recall campaigns in Taiwan
	where the incubents are voted in by marginal constituencies
	and partisan cues are among the most powerful ways
	of turning out voters.
	I suspect that it is in fact
	a response to elected officials'
	incumbency advantage
	\textit{during} their term.
	In the case of recall,
	incumbency advantage may work through two mechanisms,
	the first mechanism is that
	the electoral victory secured by the incumbent
	sends a signal that the incumbent is competent
	which should convince the voter that the incumbent remains qualified for the job.
	Another mechanism concerns the
	legitimacy bestowed by not only election law
	but also the popular mandate revealed by the election results,
	and attempting to remove the incumbent from the office
	is an illegitimate power grab
	even though it is legal.
	While the first kind of incumbency advantage
	is commonly observed at any given election,
	the second kind virtually never comes into play
	at general elections which are generally deemed legitimate
	in any democracies.
	Perhaps as a result of the latter kind of incumbency advantage
	recall organizers and campaigners may feel that
	by introducing explicit partisan cues,
	they may risk being seen as engaging in an illegitimate subversion of democratic process.
	To avoid such perception which may well eliminate the turnout power of partisan cues,
	they may hesitate to introduce any explicit partisan cues
	until there are clear signs that the recall enjoys the popular support,
	and the party can ride the wave of popular opinion.
	This effect,
	is very much contingent on the perception of recall's legitimacy.
	I hope that through research designs that employ experiments and surveys,
	it will be possible to discern the strengths of these effect.
	
	
	
	
\end{document}

